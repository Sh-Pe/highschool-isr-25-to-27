%! ~~~ Packages Setup ~~~ 
\documentclass[]{article}
\usepackage{lipsum}


% Design
\usepackage[labelfont=bf]{caption}
\usepackage[margin=0.6in]{geometry}
\usepackage{multicol}
\usepackage[skip=4pt, indent=20pt]{parskip}
\usepackage[normalem]{ulem}
\usepackage{yfonts}
\def\gothstart#1{\noindent\smash{\lower3ex\hbox{\llap{\Huge\gothfamily#1}}}
	\parshape=3 3.1em \dimexpr\hsize-3.4em 3.4em \dimexpr\hsize-3.4em 0pt \hsize}
\def\frakstart#1{\noindent\smash{\lower3ex\hbox{\llap{\Huge\frakfamily#1}}}
	\parshape=3 1.5em \dimexpr\hsize-1.5em 2em \dimexpr\hsize-2em 0pt \hsize}

\newcommand\npage {\vfil {\hfil \textbf{\textit{המשך בעמוד הבא}}} \hfil \vfil \pagebreak}


%! ~~~ Document ~~~

\date{October 28, 2025}
\author{Shahar Perets}
\title{\textit{Genesis \& Catastrophe}: Post Reading}
\begin{document}
	\maketitle
	
	\large
	
	\subsection*{\Large Question: }
	
	Choose a famous person and write their story, showing a different side to them (don't reveal who it is until later in the story, like we day about Hitler in this story). 
	
	\subsection*{\Large Story: }
	
	\noindent\frakstart{T}he doctor walked back into the room. A newly born child was in the hands of a whining mother, who gave birth to him around an hour ago, but her tears were not a result of pain, but a consequence of hopelessness.
	
	\noindent The child cried. 
	
	\noindent The doctor wasn't worried about the baby. All of the required checks have been done, and the mother appeared to be more gloomy than he's used to. 
	
	\noindent ``Are you okay?'' he asked, just to get the usual ``I'm fine, just a bit hurt'' that he got from her numerous times before. 
	
	\noindent ``Are you sure? It doesn't look like that'', he insisted. This time he got an actual reply. ``Have you been to Shlisselburg recently? Especially a week ago'. 
	
	\indent ``Yes. A friend of mine lives there. He has a profession in eye surgery.''
	
	\indent ``So you haven't been to the central city?''
	
	\indent ``The Tsar's stage of horrors? No, I'm a doctor. I cannot mentally handle it. I'm here to save lives, not to cut them off.''
	
	\indent ``My son was midtown''
	
	\indent ``Perhaps you shall teach him better.''
	
	\indent ``I don't think so. He was on stage. I haven't seen him since then.''
	
	\noindent That mother was Maria Alexandrovna Blank, a half-Jewish woman. Her other son, Sasha, was just executed a few days ago. It was an underestimate to say she loved the Tsar, Alexander III, but in her life she learn how to live alongside the monarchy, and lived a life of a comfortable, middle-class family. Sasha thought he could change that. She thought she had experienced everything when two of her children died in infancy, but seeing your own son being hanged to death, on the Tsar's stage in the middle of Shlisselburg. 
	
	\noindent The look on the doctor's face looked like a combination of shame and shock. ``I am sorry'' he replied quickly''. 
	
	\noindent ``That's fine''. A wave of revolutionaries came into our home last week. It was comforting, but I fear for my status that we maintained for the last 20 years. 
	
	\indent ``You'll be fine, I'm sure. Perhaps, that baby will live in a calmer world. At the end of the day, the Tsar is not going to live forever''. 
	
	\indent ``Their family will'', she answered pessimisticly.
	
	\indent ''Well, the previous one was great''.
	
	\indent ``I guess you're right'' she finally answered. Her tone did not indicate any agreement, rather, it was clear she decided to stop the conversation, which was clearly difficult for her.  
	
	\noindent The father came in. 
	
	\noindent The father, Ilya Ulyanov, was a descendant of a family of slaves. That fact didn't bother him, and he became a physics and mathematics teacher. In his free time, he conducted meteorological experiments. He was regarded as a multidisciplinary person with exceptional knowledge. 
	
	\noindent The child continued crying. 
	
	\noindent His mother tried to comfort him; ``Vladimir, Vladimir!'' she tiredly cried. 
	
	\noindent The father offered help and moved the child to his own hands. The child continued crying unstoppably. 
	
	\noindent The doctor tried to change the subject, as a result of the unpleasant situation he found himself in. ``and what about the name? I need to write and sign a birth certificate. You called him Vladimir, so I guess that's the name you both settled on. Am I right?''
	
	\noindent ``Yes'' they answered simultaneously. 
	
	\noindent The birth certificate was signed and given to them. They went home with the child, with hope for a better future. 
	
	\noindent That newborn changed his name in 1901 in order to hide from the authorities in Russia. The name that he chose for himself, was Lenin. Vladimir Ilyich Lenin. In his days, Shlisselburg was even busier than it was when he was born. 
	
	\noindent\dotfill
	
	\noindent\textit{Note: the chain of events was changed a bit for the sake of the story}
	\vspace{-5pt}
	
	\noindent\dotfill \\ \vfil {\begin{center}
			{\textbf{{Shahar Perets, 2025}} \\
				\scriptsize \textit{Compiled With \LaTeX\ and made exslusivly using FOSS software}}
	\end{center}} \vfil	
	
%	\ndoc
\end{document}