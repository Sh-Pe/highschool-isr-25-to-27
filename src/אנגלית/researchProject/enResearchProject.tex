\documentclass[12pt]{../../../tex/classes/styledArticle}

\usepackage{titlesec}
\usepackage{nameref}
\makeatletter
\newcommand*{\currentname}{\@currentlabelname}
\makeatother

% Titles
\titleformat{\section}       [runin]
	{\huge}
	{\Huge ({\bfseries\thesection})\ }
	{0em}
	{}
	[\dotfill]
\titleformat{\subsection}    [block]
	{\Large\itshape}
	{\normalfont\Large({\bfseries{\thesubsection}}) \,\,$\sim$\,\,}
	{0em}
	{}
\titleformat{\subsubsection} [block]
	{\normalsize\bfseries}
	{\normalfont\large\bfseries{(\thesubsubsection)}}
	{0.5em}
	{}


\author{Shahar Perets}
\title{A Compression Between The Studying Material in History as a way to Deduce State Intentions}
\begin{document}
	\maketitle
	
	\section{Introduction}\ \!\!\!
	
	\frakstart{T}his research topic is about a subject the reader is most likely greatly familiar with -- history lessons. Perhaps, the reader would have expected the existence of certain events relevant to high-school students all over the world, such as the Cold War, colonialism in the 16th-20th centuries, the First and Second World Wars (written as WWI and WWII respectively), Maoism, Nazism, and Communism. However, this is not the case; in different states and countries, there is a variety of topics being taught. 
	
	The obvious question that comes up as a result of this fact, is, why? If some events are so important to mankind, how is it that some states are completely missing them? 
	
	Our research focuses on three main questions: 
	\begin{enumerate}
		\item What is being taught in different countries as part of their history lessons? 
		\item What does it teach about the state's intentions? 
		\item How history lessons differ from Israel? 
	\end{enumerate}
	
	And it is mostly based on one of those two sources: 
	\begin{itemize}
		\item The official syllabus of the state. 
		\item External sources, in cases where the syllabus is not homogeneous (e.g. Germany) or not publicly available (e.g. China). 
	\end{itemize}
	
	
	
\end{document}