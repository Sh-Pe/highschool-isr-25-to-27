\documentclass[]{../../../../tex/classes/styledArticle}
\usepackage{../../../../tex/packages/hebrewSupport}

\usepackage{multicol,multirow}
\usepackage{tabularx,makecell}

\author{שחר פרץ}
\title{משימה ללמידה עצמאית $\sim$ מלחמת העולם השנייה לאחר מבצע ברברוסה ויהודי תוניס $\sim$ היסטוריה}
\begin{document}
	\maketitle
	\section*{השוואת הקרבות המשמעותיים במלחמת העולם השנייה}
	
	\begin{tabularx}{\textwidth}{|X|X|X|X|X|X|}
		\hline \centering\textbf{קטעי מקור} & \centering\textbf{השפעה על גורל היהודים} & \centering\textbf{השפעה על המלחמה} & \centering\textbf{מטרת הקרב} & \centering\textbf{הצדדים הלוחמים} & \textit{\textbf{שם + שנה}} \\
		\hline חיילים גרמנים מתקדמים לברה''מ (210, חזותי), מבט אווירי על ערכים שנהרסו (211, חזותי), הודעת הפלישה לרדיו (212), יומני הלדר (213), על קשיי הלחימה בברה''מ (213), הוראת היחס לשבויים רוסים (215), פקודת הקומיסרים (216)& 
		הצורך בפתרון חדש לבעית היהודים עקב מספרם הרב ברה''מ, ותחילת ביצוע הפתרון הסופי.&
		 פתיחת חזית חדשה והצטרפות ברה''מ למלחמה, מעבר להשקעת רוב חלק ניכר מהכוחות בברה''מ, לבסוף סיום המלחמה. & 
		כסלון הניסיון הגרמני להכרעת בריטניה, ותקווה שנפילת ברה''מ תוביל לסיום המלחמה, מימוש עקרונות האנטי־קומוניזם ומרחב המחייה, הערכה שגויה של כוחות הצבא האדום. &
		ברה''מ נגד גרמניה &
		\textit{מבצע ברברוסה} (יוני 1941) \\
		\hline נאום רוזוולט בפני בית הנבחרים (219)& 
		אין השפעה ישירה. &
		הצטרפות ארה''ב למלחמה ושינוי מאזן הכוחות נגד גרמניה. &
		הפסקת האמברגו הכלכלי על יפן, והוצאת ארה''ב מיכולת לחימה בזירה האנטלנטית. &
		יפן וארה''ב&
		\textit{המתקפה על פרל הארבור} (דצמבר 1941) \\
		\hline על חשיבות הקרב (326) &
		מניעת כניסת הנאצים לא''י והשמדת היישוב היהודי בא''י. החלת מדיניות אנטישמית בתוניס. &
		התחלת המפנה בחזית צפון אפריקה לטובת בריטניה וארה''ב. &
		ניסיון של היטלר לסייע לצבא האיטלקי בכיבושיו בצפון אפריקה, וניצול שדות הנפט של עיראק. &
		בריטניה, אוסטרליה וצרפת נגד גרמניה ואיטליה &
		\textit{אל־עלמיין} (יולי 1942) \\
		\hline על יחס הצבא האדום לחיליו (328)&
		אין השפעות ישירות.  &
		התשה משמעותית של הכוחות הגרמניים, שלרשותם עמדו פחות משאבים לעומת הסובייטים, ותחילת המפנה במלחמה בחזית המזרחית. &
		הגחמות של היטלר ששלט באופן ישיר הוארמכט. רוב המפקדים התנגדו לקרב ואף לא בצעו פקודות, נגד הוראות היטלר. &
		הצבא האדום, נגד מדינות הציר (גרמניה, איטליה, וגרורות רומניות, הונגריות וקוראטיות)& \textit{קרב סטלינגרד} (יולי 1942 עד פברואר 1943) \\
		\hline אין&
		אין השפעות ישירות. &
		פתיחת חזית נוספת והעברת כוחות גרמנים לאיטליה. נצחון בריטי בצפון אפריקה. &
		פתיחת חזית נוספת, בין היתר לבקשותיו של סטלין. סילוק כוחות הציר מצפון אפריקה. &
		בריטניה וארה''ב נגד איטליה עם תמיכה גרמנית. &\textit{מבצע לפיד} (יולי 1943) \\
		\hline 
	
	\end{tabularx}
	\npage
	\begin{tabularx}{\textwidth}{|X|X|X|X|X|X|}
		\hline \centering\textbf{קטעי מקור} & \centering\textbf{השפעה על גורל היהודים} & \centering\textbf{השפעה על המלחמה} & \centering\textbf{מטרת הקרב} & \centering\textbf{הצדדים הלוחמים} & \hfil\textit{\textbf{שם + שנה}} \\
		\hline נחיתת חיילי צבא ארה''ב (330, חזותי)&
		אין השפעות ישירות. &
		פתיחת חזית נוספת, ראשית הקץ של המלחמה. &
		פתיחת חזית מערבית ושחרור אירופה מהנאצים וסיום המלחמה. &
		כל מדינות הברית פרט לברה''מ. &\textit{הפלישה לנורמנדי} \\
		\hline חורבות אנדרטת השלום (346, חזותי), נאום הירוהיטו בטלווזיה על הודעת הכניעה (347)&
		אין השפעה. &
		כניעת יפן וסיום מלחמת העולם השנייה. &
		סיום המלחמה. ההנחה הייתה שבאמצעים קונבנציונליים יפן לא תכנע לעולם על לכיבושה המוחלט בידי מדינות הברית. &
		ארה''ב נגד יפן. ברה''מ פעלה בחזית אחרת נגד היפנים באותה התקופה. &\textit{הירושימה ונגסאקי} \\
		\hline
	\end{tabularx}
	
	\section*{תוניס בזמן הכיבוש הנאצי}
	\begin{enumerate}
		\item \textbf{שאלה: }\\
		הציגו את נסיבות הכיבוש הנאצי בתוניסיה על פי הפרק ועל פי השיעור שבו צפיתם. מה היו מטרותיו של כיבוש זה? 
		
		\textbf{תשובה: }\\
		המשטר הנאצי הוא משטר שהתפתח בשנת 1933 עד 1938, מנהיגו הוא אדולף היטלר, והוא דוגל באידיאולוגיה של היטלר המתוארת בספר ''מיין־קמפף`` (בתרגום חופשי, ''המסע שלי``). כדי לממש את האידיאולוגיה, היטלר פתח במסע כיבושים באירופה, וצידו נלחמה איטליה הפאשיסטית שנשלטה ע''י בניטו מוסליני. האידיאולוגיה הנאצית תומכת, בין היתר, באנטישמיות קיצונית. גרמניה הנאצית ואיטליה הפאשיסטית, יחדיו עם מדינות נוספות שהצטרפו ללחימה לצד היטלר, נקראות ''מדינות הציר``, והן נלחמו ב''מדינות הברית`` – בעיקר אנגליה, יחדיו עם מחתרות צרפתיות, וברה''מ וארה''ב שהצטרפו בשלבים מאוחרים יותר. 
		
		תוניסיה, מדינה בצפון אפריקה, הייתה נתונה משנת 1881 בידי הצרפתים, כחלק מכיבושיהם מחוץ לאירופה. בתוניסיה הוקם שלטון צרפתי, והיא הפכה למדינת חסות צרפתית. 
		
		שלטון וישי הוא שלטון צרפתי בשליטה נאצית שהוקם לאחר שגרמניה הנאצית פלשה לצרפת בשנת 1940. עם נפילת צרפת, תוניסיה עברה לידי שלטון וישי שנשלט ע''י מדינות הציר. בנובמבר 1942 נחתו כוחות של מדינות הציר במערב צפון אפריקה. בריטניה שלטה בחלקה המזרחי של צפון אפריקה, ורצתה להתחבר לכוחות שם, במטרה לקבל שליטה בצפון אפריקה. ממשלת וישי הורתה לצבאה להלחם בכוחות בעלות הברית. עם זאת, הפיקוד הצרפתי סירב לבצע פקודה זו. לאחר נחיתתם של בעלות הברית במערב צפון אפריקה היה צפי שבעלות הברית יפלשו לתוניסה ויכבשו אותה מידי מדינות הציר. 
		
		כהכנה לפלישה הבריטית, ומתוך הבנה ששלטון וישי לא ילחם נגד מדינות הברית, כוחות הציר פלשו לתנוסיה במטרה למנוע את נפילתה לידי בעלות הברית. מדינות הציר, בראשן היטלר (אך גם בעזרת מוסליני) הצליחו במטרה זו, עד מאי 1943 בו נכבשה תוניסיה ע''י הבריטים מידי כוחות הציר. 
		
		\item \textbf{שאלה: }\\
		הציגו את מדיניות הנאצים כלפי יהודי תוניס בין נובמבר 1942 למאי 1943. 
		
		\textbf{תשובה: }\\
		המשטר הנאצי הוא משטר שהתפתח בשנת 1933 עד 1938, מנהיגו הוא אדולף היטלר, והוא דוגל באידיאולוגיה של היטלר המתוארת בספר ''מיין־קמפף`` (בתרגום חופשי, ''המסע שלי``). כדי לממש את האידיאולוגיה, היטלר פתח במסע כיבושים באירופה. האידיאולוגיה הנאצית תומכת, בין היתר, באנטישמיות קיצונית. גרמניה הנאצית החילה מדיניות המממשת את האידיאולוגיה שלה במדינות אותן כבשה. 
		
		תוניסיה היא מדינה במרכז צפון אפריקה שהייתה נתונה לשליטה צרפתית, עד שצרפת נכבשה ע''י הנאצים בשנת 1940. בשלב זה צרפת עברה להשלט תחת ''שלטון וישי``, שלמעשה קיבל פקודות מהנאצים. 
		
		בנובמבר 1942 כבשה בריטניה, אחת מהמדינות שלחמו בגרמניה הנאצית, את צפון מערב אפריקה, במטרה להתחבר לכוחותיה ששהו ממזרח. גרמניה הנאצית נכנסה לתוניסיה באותו החודש מתוך הבנה שהכוחות הצרפתיים של שלטון וישי לא עומדים למנוע את כניסת הבריטים למדינה. עם כניסת הנאצים למדינה, הנאצים החלו להחיל את המדיניות האנטישמית שהפעילו במדינות נוספות באירופה: שעבוד את ראשי הקהילה היהודית על־מנת שישלטו מטעמם ביהודים, ניצול כלכלי, וגיוס כחמשת אלפים (מתוך 85 אלף) מיהודי תוניסיה לעבודות כפייה תחת תנאי חיים קיצוניים (רעב קיצוני, עבודות מסוכנות וכו'), שהרגו כשלושה רבעים מהם. זאת לצד פעולות שלא נוהלו באופן מאורגן אך הותרו ועודדו ע''י המשטר, כמו רצח, התעללות, והשפלה של יהודים. 
		
		
		\item \textbf{שאלה: }\\
		הביעו את עמדכתם – האם קורות יהודי תוניס הם חלק מהשואה? בתשובתכם התייחסו לשתי עובדות היסטוריות. 
		
		\textbf{תשובה: }\\
				המשטר הנאצי הוא משטר שהתפתח בשנת 1933 עד 1938, מנהיגו הוא אדולף היטלר, והוא דוגל באידיאולוגיה של היטלר המתוארת בספר ''מיין־קמפף`` (בתרגום חופשי, ''המסע שלי``). כדי לממש את האידיאולוגיה, היטלר פתח במסע כיבושים באירופה. האידיאולוגיה הנאצית תומכת, בין היתר, באנטישמיות קיצונית. גרמניה הנאצית החילה מדיניות המממשת את האידיאולוגיה שלה במדינות אותן כבשה. בשנת 1939 כבשה גרמניה את פולין. תוניסיה היא מדינה במרכז צפון אפריקה שהייתה נתונה לשליטה צרפתית, עד שצרפת נכבשה ע''י הנאצים בשנת 1940. בשלב זה צרפת עברה להשלט תחת ''שלטון וישי``, שלמעשה קיבל פקודות מהנאצים. 
				
				כאשר גרמניה כבשה את פולין, במשך תקופה הנאצים עשו ביהודים כל אשר רצו, דהיינו אנסו, רצחו ברחוב, התעללו והשפילו את היהודים. לאחר מכן היהודים נוצלו כלכלים ונשללו מהם נכסיהם, והנאצים את האוכלוסיה היהודית במקומות שנקראו ''גטאות`` – שכונות יהודיות. בגטאות, הנאצים לקחו את מנהיגי הקהילה ואנשים מרכזיים בחברה היהודית, ודרשו באיומי מוות מהם לבצע את פקודותיהם, תוך הסמכתם לניהול הקהילה. אנשים אלו נקראו יודנראטים. בכך, הנאצים לא היו צריכים לנהל את הקהילה היהודית באופן ישיר, ויצרו אנטגוניזם כלפי היודנראטים היהודיים, במקום כלפי הנאצים. היהודים בגטאות במקרים רבים גויסו לעבודות כפייה בתנאים קיצוניים. היהודים בגטאות חיו לרוב בתנאי חיים ירודים באופן קיצוני, תברואה ירודה, ומחסור חמור במזון. 
				
				אירועים אלו שהתרחשו בפולין, יחדיו עם אירועים דומים שהתרחשו בכל רחבי אירופה, קרויים ה''השואה``. 
				
				כאשר גרמניה הנאצית כבשה את תוניסיה, הם אסרו ארבעה מראשי הקהילה היהודית במטרה להפוך אותם למעין יודנראטים (תפקידם היה דומה אך שמם שונה). הנאצים נקטו בפרעות דומות לאלו שנקטו בפולין (רצח אקראי, השפלה פומבית, אונס ועוד). היהודים גם נוצלו כלכלית, וכספם נלקח מהם במגוון דרכים. 
				
				לדעתי, קורות יהודי תוניס הם חלק מהשואה. נתייחס לעובדה ההיסטורית הראשונה – כמו בשואה באירופה, יהודי תוניס הועבדו בפרך עד מוות, שנוהל על ידי ראשי הקהילה היהודית שהנאצים עליהם איימו (באירופה, היודנראטים). נתייחס לעובדה היסטורית שנייה: כמו באירופה בפולין, תרם הקמת המוסדות המסודרים, הנאצים ביצעו פרעות ורצח כאוות נפשם ביהודים. עקב הדמיון הרב בין קורות יהודי תוניס לבין שואת יהודי אירופה, דעתי היא שקורותיהם של יהודי תוניסיה צריכים להכלל כחלק מהשואה. 
		
		\item \textbf{שאלה: }\\בחרו מקור מספר הלימוד שמחזק את עמדתכם והסבירו במה הוא מחזק עמדה זו. 
		
		\textbf{תשובה: }(תשובה זו ממשיכה את השאלה הקודמת) \\
		נתבונן בקטע המקור המופיע בעמוד 291 בספר ''נאציזם, מלחמה ושואה`` (הוצאת היי־סקול, 2014) מאת הסופר אלבר ממי, יהודי בן התקופה שחי בתוניסיה בימי הכיבוש הנאצי. ממי מתאר את ההתעללות של הנאצים, ואת עבודות הכפייה במחנות העבודה.
		
		לפי הקטע ''ידענו קורבנות, אנשים שהוצאו להורג לשם עונש, בטעות או בדרך הלצה, נאשים שנאנסו, בתים שנבזזו, הגרמנים היו יורים לתוך החלונות [...] אך רק לתוך החלונות של היהודים``. ממי מספר על ההתעללויות של הנאצים ביהודי תוניסיה, המהווים שואה של ממש. 
		
		נוסיף ציטוט נוסף – ''הידיעות ממחנות העבודה היו רעות ביותר [...] אחי לדת [היהודים] קפחו כל צלם אדם מקץ ימים אחרים במחנה.`` הציטוט מתאר את מחנות העובדה בתוניסיה, הדומה למחנות העבודה באירופה, שבהם היהודים עבדו בתנאים קשים עד לכדי איבוד כל צלם אנוש. 
		
		לפי אירועים אילו, להם יש סימוכין בקטעי המקור, דעתי היא שקורות יהודי תוניסיה צריכים להכלל כחלק מהשואה. 
		
	\end{enumerate}
	
	\section*{מהלכי המלחמה וגורל היהודים}
	\begin{enumerate}[A.]
		\item \textbf{שאלה: }\\
		הסבר אחד מן הגורמים לפלישת גרמניה לברה''מ, שבא לידי ביטוי בקטע המקור. בחר מקור מספר הלימוד, והסבר כיצד בא לידי ביטוי במקור שבחרת גורם נוסף לפלישת גרמניה לברה''מ. 
		
		\textbf{תשובה: }\\
		בשנת 1933 עלה לשלטון בגרמניה המשטר הנאצי, שנשלט ע''י אדולף היטלר והאידיאולוגיה שלו הנתונה בספר ''מיין־קמפף``. במהרה הפך לשלטון היחיד בגרמניה. ב־1939 פתח היטלר במלחמה שלימים נקראה ''מלחמת העולם השנייה`` בה כבש חלקים נרחבים מאירופה עוד בשנותיה הראשונות. שיטת הכיבוש הנאצית נקראה ''בליצקריג``, בה הנאצים נכנסו למדינות עם כוחות אווירים וקרקעיים (לעיתים גם ימיים) נרחבים, במטרה למוטט את המדינה הנכבשת ולכבושה אותה תוך זמן קצר. מתוקף האידיאולוגיה של היטלר, הוא רוצה לכבוש שטחים נרחבים על מנת לאפשר מרחב מחייה לגזע הנאצית. 
		
		עד 1941 היטלר הספיק לכבוש את כל מזרח אירופה עד לבריטניה, חלקים ממערב אירופה. זאת עשה לרוב באמצעות שיטת ה''בליצקריג``. לאחר נפילת פולין ב־1939 לידי הנאצים, הנאצים חתמו עם סטלין, שליט ברה''מ, על הסכמים הידועים בשם ''הסכמי ריבנטרופ־מולוטוב`` – הסכמי אי־תקיפה בין ברה''מ לגרמניה, שגם הסדירו את נוכחותה של ברה''מ בחלקים נרחבים במערב אירופה. מערב אירופה וברה''מ המערבית (כמו חצי האי קרים) הכילו בעלי מזון ונפט, נוסף על אוכלוסיה לא מובטלת. כל אותה העת, ניטשה המלחמה בין גרמניה ובריטניה בחזית המערבית, לאורך תקופה ארוכה. 
		
		בשנת 1941 היטלר פתח ב''מבצע ברברוסה``, מבצע שמטרתו היה כיבוש ברה''מ. היו מספר גורמים לפלישה לברה''מ. אחד מן הגורמים הללו, הוא ניצול המשאבים שעל ברה''מ. גרמניה הייתה זקוקה לנפט, מזון, ואנשים, כדי לתמוך בלחימה באירופה, ובפרט כדי לתמוך בחזית המערבית. 
		
		באמצעות שיטת הבליצקריג, היטלר ציפה לכבוש את ברה''מ בקלות. שיטת הבליצקריג נחלה הצלחה בכל רחבי אירופה, פרט לבריטניה, והיטלר העריך שיכולתו הצבאית של הצבא האדום נחותה בהרבה מזו של הצבא הבריטי, ועל אחת כמה וכמה נחותה מצבאו שלו. הערכה זו נבעה מתפישת הרוסים כגזע נחות, ומחוסר הבנה של גודלו את הצבא הרוסי (על כל עשרה דיווזיות שהגרמנים חיסלו, לרוסים היו את האנשים להקים עוד עשרה). ההערכה הצבאית הלוקה של היטלר את יכולתו של הצבא האדום, וצפייה ששיטת הבליצקריג תכבוש בקלות לתוך אירופה. 
		
		היטלר קיווה לכבוש את בריטניה במהרה, אך נוכח שהדבר לא יתאפשר להתבצע בקלות. למרות אבדות שהסב לבריטניה בתחילת המלחמה בעת כיבוש צרפת, שם לכאורה היו בשפל המדרגה כאשר צבאם נכלא בצרפת הכבושה, הבריטים הצליחו לחלץ את כוחותיהם ולעמוד במתקפות הבליצקרינג שגרמניה הנחיתה עליהם. מלחמה שהייתה אמורה להיות קצרה ומהירה התארכה הרבה מעבר למה שהיטלר ציפה, כאשר הבריטים מתאימים את כלכלתם ומחזקים את צבאם תוך כדי הלחימה. 
		
		המטרה לסיים את הלחימה בחזית המערבית, היטלר ניסה לפתוח חזית במזרח. המטרה – להכניע את ברה''מ במהירות לה היטלר ציפה שגויות, ולאחר נפילת ברה''מ, היטלר ציפה שהבריטים יכנעו מיידית לגרמנים לנוכח העובדה שהאחרונים הפכו להיות השליטים על כל אירופה ואף מעבר לה, וכי תקוותה האחרונה של בריטניה תגוז. 
		
		לפי קטע המקור הנתון לנו על דף השאלות, הם קטעים מ''יומני האלדאר`` (מפקד כוחות היבשה בזמן מבצע ברברוסה, עד שהיטלר פיטר אותו מתפקידו שכן ביצע פעולות נגד הוראותיו חסרות כל קול הגיון, או, טקטיקה צבאית). לפי היומנים, ''משהו מוזר קרה בבריטניה! הבריטים כבר היו בשפל המדרגה, עכשיו הם שוב עומדים על רגליהם.``. האלדאר פורש אמרה של היטלר, בה היטלר מביע את מורת רוחו כנגד המשכות המערכה נגד בריטניה. ''רוסיה [ברה''מ] צריכה רק לרמוז לאנגליה שאין היא רוצה לראות בגרמניה חזרה מדי והאנגלים, כאדם טובע, יימלאו תקווה חדשה [...]`` – כאן היטלר לכאורה טוען שברה''מ יכולה להוות תקווה גדולה בעבור האנגלים. ''אולם אם רוסיה [ברה''מ] תימחץ, תעורער תקוותה האחרונה של בריטניה [...] החלטה: לאור שיקולים אלו יש לחסל את ברה''מ באביב 1941``. לנוכח השיקולים לעיל, היטלר מחליט לכבוש את ברה''מ, בציפייה שאותה תקווה תעורער, ובריטניה תכנע עקב נפילת ברה''מ (אותה ציפה לכבוש מהר וללא בעיות). נסכם – הציפייה לכניעת בריטניה עקב כיבוש ברה''מ, הוא גורם מרכזי בהחלטה לפתיחת הפלישה ברה''מ. 
		
		נתבונן בקטע המקור הוא ההוראות שהעביר מנכ''ל משרד החקלאות הנאצי, ה' בקה, על מבצע ברברוסה ועל ניצול המשאבים בחזית הזו (מתוך ''נאציזם, מלחמה ושואה``, עמוד 215, הוצאת היי־סקול 2014). בקה כותב ''המטרה הכללית של המבצע היא להשיג כמות גדולה ככל האפשר של מצרכי מזון ונפט`` – בכך, בקה טוען שחלק ממטרות המבצע הוא ניצול של משאבי הטבע באיזור. בקה מתאר גם את ניצול תושבי מערב ברה''מ – ''הכוונה אינה להפוך את הרוסים לנאצים, אלא לכלי שרת לגרמניה``. תושבי האיזור ישמשו את גרמניה הנאצית למטרותיה. נסכם, שלפי קטע המקור, בקה מציין מפורשות שמטרות המבצע הן בין היתר ניצול משאבי הטבע ומשאבי האנוש, ומכאן שהרצון לנצל את משאבי טבע אלא הוא גורם חשוב גם כן. 
		
		
		
		
		\item \textbf{שאלה: }\\
		הסבר כיצד השפיע אחד מן המהלכים או הקרבות במלחמת העולם השנייה על גורל היהודים בארצות צפון אפריקה. בתשובתך הצג את המלך או את הקרב, והסבר את השפעתו על גורל היהודים. 
		
		\textbf{תשובה: }\\
		בשנת 1933 עלתה לשלטון בגרמניה המפלגה הנאציונאל סוציאליסטית, או בקיצור המפלגה הנאצית. המפלגה הונהגה ע''י אדולף היטלר ונקטה באידיאולוגיתו שנפרשה בספר ''מיין־קמאמפ`` שהיטלר כתב בעת שהותו בכלא בגין ניסיון הפיכה. ב־1939 פתח היטלר במלחמה הקרויה ''מלחמת העולם השנייה``, בה גרמניה נלחמה לצד איטליה, יפן ומדינות נוספות נגד בריטניה, צרפת, ומאוחר יותר כל שאר העולם. באותה התקופה, צפון אפריקה נשלטה בקולוניות צרפתיות ובריטיות, ואיזור המזרח התיכון ממזרח לאפריקה נשלט ע''י הבריטים. תוך כדי המלחמה, היטלר כבש את צרפת, הטיל בה שלטון בובות נאצי (''משטר וישי``) ובכך למעשה הפך למדינות חסות באופן חלקי גם את הקולניות שלה. הנאצים והאיטלקים הפעילו כוחות בצפון אפריקה במטרה להקיף את כל הים התיכון ולהתחבר לכיבושים ממזרח. בפרט, בדרך עמדה ארץ ישראל. הכיבושים הללו התרחשו בשנים 1942-1943. הגרמנים הצליחו לחדור כל הדרך עד לתוך שטחי מצריים. 
		
		באותה התקופה, בארץ שבשליטת הבריטים חיי יישוב יהודי משגשג. האידיאולוגיה הנאצית שללה את היות היהודים גזע, וקבעה שהם משמידים ופוגעים באופן פעיל בחברה, ושהם טפיל שיש להפתר ממנו. סמוך לשנים אלו, הנאצים החלו בפעולות השמדה הומוניות של יהודי מזרח אירופה וברה''מ. ארץ ישראל שבשליטת הבריטים גבלה במצריים (גם היא בשליטת הבריטים) שהובקעה ביולי 1942 ע''י כוחות נאצים. המשך פעולות הנאצים וכיבושיהם היה מביא להשמדת היישוב היהודי בארץ. 
		
		בין יולי לנובמבר 1942, התרחשו מספר קרבות הידועים בשמות ''קרב אל־עלמיין הראשון`` ו''קרב אל־עלמיין השני``. על הראשון מבינהם פיקד הגנרל הבריטי קלוד אוקילנק, ושם הוא הצליח לבלום את הכוחות הגרמניים. הגנרל ברנרד מונטגומרי נטל ממנו את הפיקוד מאוחר יותר, והוא בלם מתקפה גרמנים נוספת באותו החודש (קרב אל עלמיין השני). בעקבות קרבות אלו, באוקטובר התרחש מפנה במלחמה, ומונטגומרי החל לתקוף את גרמניה ולהחזיר שטחים לשליטת בריטניה. האיטלקים והגרמנים נסוגו 1300 קילומטר מערבה עד נובמבר. 
		
		קרבות אל־עלמיין הוציאו מכלל סיכון את כיבוש א''י בידי הגרמנים והשמדתם את היישוב היהודי בארץ. הנצחון הבריטי הציל את היישוב היהודי, וייתכן שאף הציל את שיאופותיהם של היהודים להקמת מדינה על הארץ, וככזה, הייתה לו השפעה מכרעת על גורל היהודים. 
		
	\end{enumerate}
	
	\ndoc
\end{document}