%! ~~~ Packages Setup ~~~ 
\documentclass[]{article}
\usepackage{lipsum}
\usepackage{rotating}


% Math packages
\usepackage[usenames]{color}
\usepackage{forest}
\usepackage{ifxetex,ifluatex,amssymb,amsmath,mathrsfs,amsthm,witharrows,mathtools,mathdots}
\usepackage{amsmath}
\WithArrowsOptions{displaystyle}
\renewcommand{\qedsymbol}{$\blacksquare$} % end proofs with \blacksquare. Overwrites the defualts. 
\usepackage{cancel,bm}
\usepackage[thinc]{esdiff}


% tikz
\usepackage{tikz}
\usetikzlibrary{graphs}
\newcommand\sqw{1}
\newcommand\squ[4][1]{\fill[#4] (#2*\sqw,#3*\sqw) rectangle +(#1*\sqw,#1*\sqw);}


% code 
\usepackage{algorithm2e}
\usepackage{listings}
\usepackage{xcolor}

\definecolor{codegreen}{rgb}{0,0.35,0}
\definecolor{codegray}{rgb}{0.5,0.5,0.5}
\definecolor{codenumber}{rgb}{0.1,0.3,0.5}
\definecolor{codeblue}{rgb}{0,0,0.5}
\definecolor{codered}{rgb}{0.5,0.03,0.02}
\definecolor{codegray}{rgb}{0.96,0.96,0.96}

\lstdefinestyle{pythonstylesheet}{
    language=Java,
    emphstyle=\color{deepred},
    backgroundcolor=\color{codegray},
    keywordstyle=\color{deepblue}\bfseries\itshape,
    numberstyle=\scriptsize\color{codenumber},
    basicstyle=\ttfamily\footnotesize,
    commentstyle=\color{codegreen}\itshape,
    breakatwhitespace=false, 
    breaklines=true, 
    captionpos=b, 
    keepspaces=true, 
    numbers=left, 
    numbersep=5pt, 
    showspaces=false,                
    showstringspaces=false,
    showtabs=false, 
    tabsize=4, 
    morekeywords={as,assert,nonlocal,with,yield,self,True,False,None,AssertionError,ValueError,in,else},              % Add keywords here
    keywordstyle=\color{codeblue},
    emph={var, List, Iterable, Iterator},          % Custom highlighting
    emphstyle=\color{codered},
    stringstyle=\color{codegreen},
    showstringspaces=false,
    abovecaptionskip=0pt,belowcaptionskip =0pt,
    framextopmargin=-\topsep, 
}
\newcommand\pythonstyle{\lstset{pythonstylesheet}}
\newcommand\pyl[1]     {{\lstinline!#1!}}
\lstset{style=pythonstylesheet}

\usepackage[style=1,skipbelow=\topskip,skipabove=\topskip,framemethod=TikZ]{mdframed}
\definecolor{bggray}{rgb}{0.85, 0.85, 0.85}
\mdfsetup{leftmargin=0pt,rightmargin=0pt,innerleftmargin=15pt,backgroundcolor=codegray,middlelinewidth=0.5pt,skipabove=5pt,skipbelow=0pt,middlelinecolor=black,roundcorner=5}
\BeforeBeginEnvironment{lstlisting}{\begin{mdframed}\vspace{-0.4em}}
    \AfterEndEnvironment{lstlisting}{\vspace{-0.8em}\end{mdframed}}


% Deisgn
\usepackage[labelfont=bf]{caption}
\usepackage[margin=0.6in]{geometry}
\usepackage{multicol}
\usepackage[skip=4pt, indent=0pt]{parskip}
\usepackage[normalem]{ulem}
\forestset{default}
\renewcommand\labelitemi{$\bullet$}
\usepackage{titlesec}
\titleformat{\section}[block]
{\fontsize{15}{15}}
{\sen \dotfill (\thesection)\she}
{0em}
{\MakeUppercase}
\usepackage{graphicx}
\graphicspath{ {./} }

\usepackage[colorlinks]{hyperref}
\definecolor{mgreen}{RGB}{25, 160, 50}
\definecolor{mblue}{RGB}{30, 60, 200}
\usepackage{hyperref}
\hypersetup{
    colorlinks=true,
    citecolor=mgreen,
    linkcolor=black,
    urlcolor=mblue,
    pdftitle={Document by Shahar Perets},
    %	pdfpagemode=FullScreen,
}


% Hebrew initialzing
\usepackage[bidi=basic]{babel}
\PassOptionsToPackage{no-math}{fontspec}
\babelprovide[main, import, Alph=letters]{hebrew}
\babelprovide[import]{english}
\babelfont[hebrew]{rm}{David CLM}
\babelfont[hebrew]{sf}{David CLM}
%\babelfont[english]{tt}{Monaspace Xenon}
\usepackage[shortlabels]{enumitem}
\newlist{hebenum}{enumerate}{1}

% Language Shortcuts
\newcommand\en[1] {\begin{otherlanguage}{english}#1\end{otherlanguage}}
\newcommand\he[1] {\she#1\sen}
\newcommand\sen   {\begin{otherlanguage}{english}}
    \newcommand\she   {\end{otherlanguage}}
\newcommand\del   {$ \!\! $}

\newcommand\npage {\vfil {\hfil \textbf{\textit{המשך בעמוד הבא}}} \hfil \vfil \pagebreak}
\newcommand\ndoc  {\dotfill \\ \vfil {\begin{center}
            {\textbf{\textit{שחר פרץ, 2025}} \\
                \scriptsize \textit{קומפל ב־}\en{\LaTeX}\,\textit{ ונוצר באמצעות תוכנה חופשית בלבד}}
    \end{center}} \vfil	}

\newcommand{\rn}[1]{
    \textup{\uppercase\expandafter{\romannumeral#1}}
}

\makeatletter
\newcommand{\skipitems}[1]{
    \addtocounter{\@enumctr}{#1}
}
\makeatother


%! ~~~ Document ~~~

\author{שחר פרץ}
\title{\textit{היסטוריה 28} $\sim$ \en{WWII}}
\begin{document}
    \maketitle
    \section{\en{Intro. to WWII during 1939 to June 1941}}
    למדנו על גרמניה ב־1933-1938. מלהע''ר התחילה ב־ לספטמבר 1939 (מה שהיה ב־39 עד אז לא בחומר הלימוד) ואת הקטע הזה, עד 41 ולפני הפתרון הסופי – נלמד עתה. את  הקטע שכולל את הפתרון הסופי כנראה לא נספיק ללמוד השנה. ביוני 41 גרמניה פלשה לבריה''מ ואירועי המלחמה השתנו באופן משמעותי. 
    
    \subsection{הסכם רינטרופב־מולוטוב}
   ב־23.8.1939 חתמו שני צרכי הרחוץ, ריבונרופ הגרמני ומולוטוב הרוסי על הסכם בעל שני צדדים: 
   \begin{itemize}
       \item \textbf{החלק הגלוי} – הסכם אי התקפה, שיתופי פעולה מודיעינים ולכלכליים. 
       \item \textbf{חלק סמוי} – כלל את חלוקת פולין כאשר גרמניה תפלוש לפולין. לכל אחד מהצדדים היו אינטרסים בפולין, וההסכם חילק באופן כללי את מרכז/מערב פולין לגרמניה, וחלקה המזרחי ומדינות נוספות – לרוסים. מי שפלש לפולין במציאות  – זה גרמניה, אך ברית המועצות לקחה חלק בלחימה כחלק מההסכם הסודי. 
   \end{itemize}
   \textbf{הבעיות בהסכם: }לא יכול להיות שבתפישה הנאצית, האנטי־קומוניסטית, זה נוגד את (1) האנטי־קומוניזם באידיאולוגיה הנאצית ו־(2) את מרחב המחייה של הגזע הארי בברית המועצות. 
   
   כל הסכם שהיטלר חתם – נועד לאינטרסים של גרמניה, בידיעה ברורה של היטלר שההסכם יופר. במקרה הזה היה ברור לשני הצדדים שההסכם יורפ, כי אידיאולוגית גם הקומוניזם מנוגד לנאציזם הפשיסטי. ``פועלי כל העולם, התאחדו'' – ללא תלות בלאום, אנטי־פשיזם. היטלר לא מעוניין להלחם בשתי חזיתות, הצבא הרוסי חזק ולחימה מול הצבא האדום לא התאימה להיטלר. 
   
   \textbf{אינטרסים של היטלר: }
   \begin{itemize}
       \item לא רצה להלחם בשתי חזיתות
       \item הפרדה בין בריב''מ למערב – באותה התקופה הקומוניזם נתפש כאיו םיותר גדול מאשר הנציזם
       \item באמצעות ההסכם הוא יכול לממש את עקרון ``מרחב מהחייה''. איך? הוא יפר אותו. בהתחלה הוא יפלוש לפולין ובהמשך הוא יפר אותו. 
   \end{itemize}
   
   \textbf{האינטרסים של סטלין: }
   \begin{itemize}
       \item באותה התקופה טיהורים בראש הצבא, כלומר מחסל את מתנגדיו. בכך סטלין החליש את הצבא שלו, וזה לא היה טיימינג טוב מבחינתו להלחחם מול גרמניה.
       \item סטלין יודע שהיטלר יצא מחבר הלאומים ויכול להגדיל את צבאו כרצונו. 
       \item סטלין לא מחבב את מדיניות הפיוס של מדינות המערב – מדינות המערב חשבו שבמאצעות הסכים וויתורים, הם יוכלו למנוע פריצת מלחמה. 
       \item ברית המועצות מרוויחה חלק מפולין
   \end{itemize}
   
   ראה הסכם מינכן (שלא בחומר), הסכם בין מדינות המערב (בריטניה, צרפת ואיטליה) מול גרמניה שעל פיו מדינות המערב נותנות להיטלר את הסודתים – איזור בצ'כוסלובקיה שאוכלוסיתו גרמנית ברובה. הבטחתו של היטלר שזו תהיה בקשתו האחרונה, ורא שממשלת בריטניה עמד על הבמה ואומר ``זה שלום בדורנו, השגנו את השלום המיוחל''. אחרי כמה חודשים היטלר כבש את כל צ'כוסלובקיה, המדינה האחרונה שנשארה דמוקרטית מהסכמי וורסאי. זו היתה מדינה חזקה עם צבא חזק שהיה יכול להתמודד מול גרמניה, אבל מדינות המערב מכרו אותה להיטלר. סטלין היה מאוכז מאוד מההתנהגות הזו. 
   
   23 באוגוסט – שבוא אחרי, ה־1 בספטמבר וגרמניה פולשת לפולין. 
   
   \subsection{הפלישה לפולין}
   אז איך גרמניה פולשת לפולין? הרי היא חתומה איתה על הסכם אי־התקפה. איך עשו זאת? ביימו תקרית על הגבול שבעקבותיה נטען שפולין תקפה את גרמניה, ועל כן גרמניה יכולה לפלוש לפולין כדי להגן על עצמה. בכך גרמניה פולשת לפולין. הצבא הפולני היה צבא שחלקיו היו רובים על חניתות (לא כולם, אך חלק ניכר). זה היה צבא חלש ולא מפתוח שבטח שלא יכל להתמודד עם המטוסים הגרמניים. 
   
   עתה נתבונן בסרטון, ונבחין: מי הצדדים הלוחמים, מה זה בליצקריג, ואיך זה בא לידי ביטוי בהמלע''ר. 
   \begin{itemize}
       \item הגרמנים חילקו את פולין לשניים – מערב פולין, מזרח פולין ומרכז פולין. את מרכז פולין ניהל ממשל אזרחי גרמני, הגנרל גובנמן. מערב המדינה סופח לרייך. חלקה המזרחי סופח לברה''מ. 
       \item פולין מפסיקה להתקיים שבועיים אחרי פרוץ המלחמה. בניגוד למדינות אחרות שממשיכות להתקיים, בין אם כמדינות ואפילו חלקן נשארות בשלטון עם מדיניות פנים. \textbf{פולין לא קיימת יותר}. ממשלתה גולה ונמצאת בבריטניה. 
       \item על אוכלוסיית ףולין הוטל משטר קשה, ועבור היהודים אלימות חזקה יותר. 
       \item עם פלישת גרמניה לפולין הוכרז מלחמה ע''י צרפת ובריטניה, אך בהתחלה הן לא פעלו ומלחמו כמעט. 
       \item תוך חודש נכבשה פולין במה שכונה ``מלחמת הבזק'' – הבליצקריג, לפיה התנהלו השנתיים הראשונות של הלחימה. צבא פולין עמד לבדו מול הגרמנים. מה שאפשר להם להתחיל את המלחמה הוא הסכם רינטרוב־מולוטוב. 
   \end{itemize}
   המדינות שלחמו בשנים הללו: \textbf{מדינות הציר: }גרמניה, איטליה ויפן. מדינות כמו הונגריה, רומניה וכו' שיתפו פעולה אך לא באמת היוו חלק ממדינות הציר. \textbf{בנות הברית: }ברה''מ, ארה''ב ובריטניה. ארה''ב נכנסה בפרהל הרבור, צרפת לא חלק מבנות הברית כי היא נכבשה במהרה. 
   
   \textbf{המשלך רוב המלחמה, עד 1941, היחידה שנלחמה בגרמניה היא בריטניה. }היא המדינה היחידה שנלמחמה מתחילתה ועד סופה היא בריטניה. חלקן הצטרפו יותר מאוחר, חלק יצאו וחלק נכנסו, בריטניה היחידה שעברה את כל המלחמה. יש לכך משמעויות כלכליות על גריטניה במהשך. 
   
   חלוקת פולין תלווה אותנו עד סיום מלחמה. יש לחלוקה הזו חשיבות. דנציג – אזור המריבה/ויכוח של גרמניה, יש בו מוצא לים והרבה דברים שנאלצו לוותר עליהם בהסכמי ורסאי. 
   
   אין שום גנרל שקורסים לא גובנמאן. קוראים לזה הגנראל־גוברנאמן, מהמילה general, איזור הממשל הכללי. כל מה שנדבר על פולין ־ מחנות השמדה, כטאות וכו' – בגנראלגוברנאמן. אין מחנות השמדה בגרמניה, יש מחנות ריכוז ועבודה אך לא השמדה. המחנות בגנראלגובנאמן  –טרנוב, קרקוב, ובלין, ורשה ועוד. יש כטאות גם בחלק של ברית המועצמות, אך אלו גטאות מסוג אחר. יש גטו אחד חריג, גטו לודז', הקשה ביותר עם התנאים הקשים ביותר, שלא נמצא ובגנרלגובננאמן. 
   
   היטלר היה בטוח שהצרפתי םוהאנגלים לא ינקטו כל פעולה נגד פולין, אך הן הציבו לא אולטימטום. לפי היטלר, אוביו תולעים קטנות, ומי ירצה להסתבך מלחמת עולם? במהרה היתה הצהרת מלחמה. לפי עדים, היטלר נראה כאילו התאבד. הוא חשש מאוד ממלחמה בשתי חזיתות, אך הפור נפל והכוחות המזויינים של גרמניה הנאצית נשלחו לתוך פולין. 
   
   
   
   \subsection{כיבושי גרמניה הנוספים}
   
   גרמניה כבשה של המדינות הבלטיות ופחות או יותר כל מדינה באירופה, מה שאפשר לה בסוף לכבוש את צרפת וזהו זה. 
   
   תחילת הכיבוש, מאביב 1940: המדינות האריות. בפרט, דנמרק, נרווגיה, בלגיה והולנד. \textit{היחס לא כמו בפולין}. ביוני 1940 נכבשה צרפת. באוגוסט 1940 החל \textbf{הקרב על בריטניה} – המדינה הראשונה שגרמניה לא מצליחה להביס. הכשון הצבאי הראשון של גרמניה במלחמה. 
   
   בחורף 1940-1941 – היטרל אילץ את רומניה, הונגריה, בולגריה וסלובקיה להפוך למדינות חסות. המשמעות: המדינה נשארת קיימת, אך השלטון הוא שלטון נאצי פשיסטי. 
   
   יש הבדל בין מה שקרה בין בולגריה לשאר המדינות ליהודים. זו המדינה היחידה מבין המדינות לעיל שהיהודים בה לא נרצחו. הממשל בה \textit{הגן} על היהודים. זה לא משהו ידוע, מדברים בעיקר על דנמרק כעל מדינה שהצילו את היהודים שלה. לעומת הולנד, שיש לה תדמית של חסידי אומות עולם וכו' – רובם נרצחו בשואה. ``זוכרים את אנה פרנק בעליית הגג אבל שוכחים שיש מי שהלשין עליהם''. 
   
   בפריז, החתנתה הרכבת, אוגוסט 1914 – צעדו אל המלחמה בפרחים, אך לא כן במלחמת העולם השנייה. אף אחד לא רוצה להלחם. 4 מיליון גברים, ספרט כוכבי קולנוע, מתכייסים לצבא, מרביתם חלקאים. היה מחסור גדול בציוד, רובה אחד לכל שני אנשים. ``הייתה קופסא אחת עם 10 כדורים והיה אסור לנו לפתוחאותה''. בצבא הצרפתי התבססס על סוסים. כינויי גנאי רבים היו לגרמניים. 
   
   \subsubsection{סיכום שלי מהסרטים}
   
   חצי מיליון צרפתים מול 200 אלף גרמנים. 4 ימים אחרי הצהרת המלחמה של צרפת. ההתקפה שהיתה אמורה להראות שפולין לא הופקרה התקדמה 9 ק''מ. הצרפתים הציגו בגאווה את השלל המלחמתי, אופניים (סרטון של חיילים על אופניים בסרטון תעמולה צרפתי). ``החייל הצרפתי הראשון המעלה'' הפך למשתף פעולה הדוק על הנאצים והוצא הלורג בסו ףהמלחמה. הצרפתים חשבו שבגלל יתרונם המספרי ינצחו את הגרמנים בקלות. אף אחד לא רצה להלחם את המלחמה של 1914, וההנחה של צרפת הייתה כי יש להתבצר מאחורי קו מאג'ינו, באורך מליון וחצי מטרים מועקבים של בטון ו־16 שנים של בנייה, עם מבוך של מנהרות, ו־200 אלף איש איכלסו אותו. ההנחה – טנקים גרמנים לא יצליחו לעבור באיזור הצפוני לים, כי בלגיה התנגדה לכך (טרם נכבשה). קו ציקפריד – שרשרת הביצורים שהיטרל הקים מול מאג'ינו. הגרמנים לא תוקפים ומנסים להמנע מחזית שנייה. הצרפתים פינו את אלזס לורן כאמצעי זהירות. 
   
   [הבהרה לא קשורה ביחס לתגובה שקראתה בכיתה: סרט דוקומנטרי $\neq$ אמת. הוא ערוך, ואף אם הוא מורכז מסרטים ועדויות מאותה התקופה, יש לו עריכה וקריינות, ויש בחירה מה נכנס אליו ומה לא. יותר חשוב מזה, מישהו החליט מה לא להכניס לסרט, כי יש לו אג'נדה. תיאורטית, אם בסרט היה נערך ע''י האנצים, זה היה מוצג אחרת, כמו סרטי תדמית לגטאות]
   
   צנחים גרמים צנחו על אדמת הולנד וצעדו לתוך בלגיה. תרחיש ההטעיה של היטלר: כאילו הם עומדים לכבוש את צרפת דרך בלגיה, כמו ב־1914. גמלה המפקד העביר את עילית הצבא הצרפתי לתוך בלגיה. הם בלמו את התקדמות הגרמנים בבלגיה, אך היטלר שמח על כך – הוא הצליח להטעות את צרפת. ``יכולתי לבכות מרוב שמחה''. הוא נתן הוראה לא תקוף את הכוחות בבלגיה מהאוויר, אלא לתקוף מאוחר. היטלר תקף ממקומות עליהם לפי צבא צרפת ``אף טנק לא יוכל לעבור את ההרים הללו היוערים בצפיפות'', אך הדבר היחיד שעצר אותם הם פקקי התנועה שהם עצמם יצרו. הטנקים הגרמנים עברו מהר את העריות של צפון צרפת. הגרמנים מחצו כל התנגדות. מפקדי הצבא הצרפתי לא ידעו איפה האויב נמצא. תוך 8 ימים המפקדים כבר היו מוטשים כליל מבחינה פסיכולוגית. הגמנים הגיעו לתעלת למנש. גם חיילות אנגלים כותרו באיזור. ראש ממשלת צרפת מפטר את גמלה. הוא ממנה שני גיבורים ממלחמת העולם הראשונה.
   
   היטרל הניח לצבא צרפת לסגת בשקט, בבשיל לא לעצבן את בריטניה שכוחותיה נמצאו שם. צ'רצ'יל מצווה לקחת כל שלי שייט שמסוגל לשוט לחלץ את הלוחמים הנצורים, בינהם הגנרלים הבריטיים. מטוסי שטוקה הפילו ספינות רבות. אף על פי כן מאות אלפי חיילים חלצו, והצבא הבריטי ניצל. הבריטים נשלחים למרכזי הצטיידות מחדש, ובריטניה משבחת את המבצע. צ'רצ'יל טען ש''אין מנצחים מלחמות באמצעות נסיגה''. המבצע לעיל התרחש בדנקרק. ב־4 ביוני צ'רצ'יל נאם נאום לפיו לעולם לא יכנעו. אכזריותם של הנאצים הייתה כבר ברורה לבריטניה. הלוחמים הבריטים התירו מאחוריהם בדנקרק את כל ציודם, שגרמניה באותו היום (4 ביוני) לקחו. 
    
    \subsubsection{סיכום של שרית}
    המלחמה המדומה – בריטניה וצרפת מכריזות מלחמה בגלל פולין, אך בגלל תחושת העליונות והמחשבה שקו מאג'ינו (כמו הגדר של ביבי סביב עזה, או כמו ב־73' ביום כיפור עם קו מוצבים הקרוי קו בר־לב שחשבנו שהטנקים המצרים לא יצליחו לעבור אותו). טעות. 
    
    שיטת הלחימה: בליצדריג. הכנסת כל הכוחות, אוויר ים ויבשה, באש מאסיבית וכיבוש מהיר של אזורים. ``ברגע שזה מתחיל זה מאוד מאוד מהיר. לכבוש את פולין בשבועיים זה כלום''. אופן הלחימה הזה חדש, והתאפשר באמצעות התחזקותה לפני מלחמת העולם השנייה. גריטניה כורתת בריטות בין גרמניה לבין איטליה (ברית רומא־ברלין) ואח''כ רומא־ברלי־טקויו, שסייעו לגרמניה לנצח במלחמה בשנתיים הראושנות. 
    
    ישן 4 סיבות מרכזיות לנצחון גרמניה: 
    \begin{itemize}
        \item יש צלצול. 
    \end{itemize}
    
    ``יום שני ב־7:45, טלפונים מקדימה, כלי כתיבה, ספרים דפים לא אני פשוט מחכה שאני אוכל לדבר אז מבחינתי שזה יקרה עוד שעה תביאו מרקרים אתם מסמנים אתם זה''. זה שעתיים לפי השעון. 
    
    
    \ndoc
\end{document}