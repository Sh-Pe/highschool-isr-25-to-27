%! ~~~ Packages Setup ~~~ 
\documentclass[]{article}
\usepackage{lipsum}
\usepackage{rotating}


% Math packages
\usepackage[usenames]{color}
\usepackage{forest}
\usepackage{ifxetex,ifluatex,amssymb,amsmath,mathrsfs,amsthm,witharrows,mathtools,mathdots,centernot}
\usepackage{amsmath}
\WithArrowsOptions{displaystyle}
\renewcommand{\qedsymbol}{$\blacksquare$} % end proofs with \blacksquare. Overwrites the defualts. 
\usepackage{cancel,bm}
\usepackage[thinc]{esdiff}


% tikz
\usepackage{tikz}
\usetikzlibrary{graphs}
\newcommand\sqw{1}
\newcommand\squ[4][1]{\fill[#4] (#2*\sqw,#3*\sqw) rectangle +(#1*\sqw,#1*\sqw);}


% code 
\usepackage{algorithm2e}
\usepackage{listings}
\usepackage{xcolor}

\definecolor{codegreen}{rgb}{0,0.35,0}
\definecolor{codegray}{rgb}{0.5,0.5,0.5}
\definecolor{codenumber}{rgb}{0.1,0.3,0.5}
\definecolor{codeblue}{rgb}{0,0,0.5}
\definecolor{codered}{rgb}{0.5,0.03,0.02}
\definecolor{codegray}{rgb}{0.96,0.96,0.96}

\lstdefinestyle{pythonstylesheet}{
	language=Java,
	emphstyle=\color{deepred},
	backgroundcolor=\color{codegray},
	keywordstyle=\color{deepblue}\bfseries\itshape,
	numberstyle=\scriptsize\color{codenumber},
	basicstyle=\ttfamily\footnotesize,
	commentstyle=\color{codegreen}\itshape,
	breakatwhitespace=false, 
	breaklines=true, 
	captionpos=b, 
	keepspaces=true, 
	numbers=left, 
	numbersep=5pt, 
	showspaces=false,                
	showstringspaces=false,
	showtabs=false, 
	tabsize=4, 
	morekeywords={as,assert,nonlocal,with,yield,self,True,False,None,AssertionError,ValueError,in,else},              % Add keywords here
	keywordstyle=\color{codeblue},
	emph={var, List, Iterable, Iterator},          % Custom highlighting
	emphstyle=\color{codered},
	stringstyle=\color{codegreen},
	showstringspaces=false,
	abovecaptionskip=0pt,belowcaptionskip =0pt,
	framextopmargin=-\topsep, 
}
\newcommand\pythonstyle{\lstset{pythonstylesheet}}
\newcommand\pyl[1]     {{\lstinline!#1!}}
\lstset{style=pythonstylesheet}

\usepackage[style=1,skipbelow=\topskip,skipabove=\topskip,framemethod=TikZ]{mdframed}
\definecolor{bggray}{rgb}{0.85, 0.85, 0.85}
\mdfsetup{leftmargin=0pt,rightmargin=0pt,innerleftmargin=15pt,backgroundcolor=codegray,middlelinewidth=0.5pt,skipabove=5pt,skipbelow=0pt,middlelinecolor=black,roundcorner=5}
\BeforeBeginEnvironment{lstlisting}{\begin{mdframed}\vspace{-0.4em}}
	\AfterEndEnvironment{lstlisting}{\vspace{-0.8em}\end{mdframed}}


% Design
\usepackage[labelfont=bf]{caption}
\usepackage[margin=0.6in]{geometry}
\usepackage{multicol}
\usepackage[skip=4pt, indent=0pt]{parskip}
\usepackage[normalem]{ulem}
\forestset{default}
\renewcommand\labelitemi{$\bullet$}
\usepackage{graphicx}
\graphicspath{ {./} }

\usepackage[colorlinks]{hyperref}
\definecolor{mgreen}{RGB}{25, 160, 50}
\definecolor{mblue}{RGB}{30, 60, 200}
\usepackage{hyperref}
\hypersetup{
	colorlinks=true,
	citecolor=mgreen,
	linkcolor=black,
	urlcolor=mblue,
	pdftitle={Document by Shahar Perets},
	%	pdfpagemode=FullScreen,
}
\usepackage{yfonts}
\def\gothstart#1{\noindent\smash{\lower3ex\hbox{\llap{\Huge\gothfamily#1}}}
	\parshape=3 3.1em \dimexpr\hsize-3.4em 3.4em \dimexpr\hsize-3.4em 0pt \hsize}
\def\frakstart#1{\noindent\smash{\lower3ex\hbox{\llap{\Huge\frakfamily#1}}}
	\parshape=3 1.5em \dimexpr\hsize-1.5em 2em \dimexpr\hsize-2em 0pt \hsize}



% Hebrew initialzing
\usepackage[bidi=basic]{babel}
\PassOptionsToPackage{no-math}{fontspec}
\babelprovide[main, import, Alph=letters]{hebrew}
\babelprovide[import]{english}
\babelfont[hebrew]{rm}{David CLM}
\babelfont[hebrew]{sf}{David CLM}
%\babelfont[english]{tt}{Monaspace Xenon}
\usepackage[shortlabels]{enumitem}
\newlist{hebenum}{enumerate}{1}

% Language Shortcuts
\newcommand\en[1] {\begin{otherlanguage}{english}#1\end{otherlanguage}}
\newcommand\he[1] {\she#1\sen}
\newcommand\sen   {\begin{otherlanguage}{english}}
	\newcommand\she   {\end{otherlanguage}}
\newcommand\del   {$ \!\! $}

\newcommand\npage {\vfil {\hfil \textbf{\textit{המשך בעמוד הבא}}} \hfil \vfil \pagebreak}
\newcommand\ndoc  {\dotfill \\ \vfil {\begin{center}
			{\textbf{\textit{שחר פרץ, 2025}} \\
				\scriptsize \textit{קומפל ב־}\en{\LaTeX}\,\textit{ ונוצר באמצעות תוכנה חופשית בלבד}}
	\end{center}} \vfil	}

\newcommand{\rn}[1]{
	\textup{\uppercase\expandafter{\romannumeral#1}}
}

\makeatletter
\newcommand{\skipitems}[1]{
	\addtocounter{\@enumctr}{#1}
}
\makeatother

%! ~~~ Document ~~~

\author{שחר פרץ}
\title{\textit{היסטוריה 37}}
\begin{document}
	\maketitle
	
	\section{מבנה הבחינה}
	\begin{itemize}
		\item \textbf{פרק ראשון – }טוטאליטריות ושואה. 
		\item \textbf{פרק שני – }לאומיות וציונות. 
	\end{itemize}
	בכל פרק יש 3 שאלות, ובוחרים ממנו שאלה 1. בכל שאלה 2 סעיפים. משך הבחינה שעתיים (שעון, כלומר 120 דק') + חצי שעה תוספת זמן. 
	
	\section{התנגדות יהודית}
	ההתנגדות היהודית מחולקת לארבע שיטות שונות: 
	\begin{enumerate}
		\item מרד בגטאות
		\item יהודים חיילים בצבאות בעלות הברית
		\item פרטיזנים
		\item מרד במחנות השמדה
	\end{enumerate}
	אנחנו נלמד רק את השניים הראשונים, מתוכם את (1) למדו בשיעור הקודם. שרית תבהיר לקראת המבחן האם (2) נכלל במבחן הקרוב או לא. 
	
	\subsection{חיילים יהודים בצבעות בעלות הברית}
	למעוניינים – מוזיאון בלטרון. שמציג את הלחימה הזו בדיוק. הוא אינטראקטיבי עם סרטונים ומסכי מגע. צריך תיאום מראש. 
	
	רובם – פשוט אזרחים (המקרה של הבריגדה בארץ והצנחנים בה מקרה מיוחד). מה יש לציינם בצורה מיוחדת? הרי מרבית אזרחי המדינות התגייסו. בפועל יש דברים מיוחדים שכדאי לעבור עליהם, בפרט הסיבות להצטרפות שלהם: 
	\begin{itemize}
		\item פטריוטיות למדינה שלהם
		\item חובת גיוס (תלוי בצבא)
		\item התנדברות מתוך רצון לנקום בנאצים שהשמידו את בני עמם
		\item יהודי א''י שהתנדבו לצבע הבריחי במסגרת ``הבריגדה היהודית'', שקיוו שהבריגדה תהפוך לגרעין של הצהבע העברי כאשר תוקם מדינה יהודית בא''י. בריטניה לא התלהבה מלהקים את הבריגדה, היא הוקמה הנהגת הישוב היהודי. נוסף על כך היו הצנחנים. 
		
		
	\end{itemize}
	
	בפועל, באמת ראינו גיוס גבוהה יותר (ביחס לאוכלוסיה) של יהודים למלחמה. 
	
	כמליון וחצי יהודים נלחמו בכל החזיתותת, כרבע מיליון לוחמים יהודית נהרגו. נפרט קצת כל צבא: 
	\begin{itemize}
		\item בצבא האדום: לחמו כחצי מיליון יהודים. מתוכם 150 זכו בעיטור הגובהה הגבוהה ביותר שניתן שם, ``גיבור בריה''מ''. חלקם נמנו על משחררי מחנות ההשמדה ומהלוחמים שכבשו את ברלין. היהודים היו 10\% מהלוחמים בעור שיעורם באוכלוסיה היה 2\%. מתוכם 200 אלף (כ־40\%) נפגעו בקרב. אחת הלוחמות המפורסמות היא סרן פולימה גלמן, נווטת, שביצעה 869 משימות ב־1300 שעות טיסה והטילה 113 טון פצצות על ריכוז אוייב. זכתה לתואר ``גיבורת ברית המועצות''. בצבא האדום היו לא מעט חיילות ומפקדות באופן כללי. 
		
		לצבא האדום הייתה חובת גיוס יחסית משמעותית, אך גם יהודים מחוץ לטווח הגילאים התגייסו. 
		
		\item בצבא ארה''ב היו כחצי מליון לוחמים יהודים, כ־40 מתוכם הגיעו לפקידים בכירים. אחד מהם (קולנל מרקוס) אחרי המלחמה הגיעה לארץ כדי לסייע במלחמת העצמאות, ונהרג בטעות לאחר שלא ידע את סיסמת הכניסה לבסיס (לא דבר עברית). בצבא ארה''ב לא ממש הייתה חובת גיוס. 
		\item בצבא הבריטי, כ־30 מיהודי א''י התנגבו כדי להלחם בנאצים. הבריגדה היהודית מנתה כ־5000 מבני הישוב בארץ ישראלי שהוקמה בסוף המלחמה ולחמה בחזית הצרפתית. נוסף על כך הייתה את קבוצת המתנדבים הצנחנים, כולם נכנסו בצורה לא חוקית למדינות באמצעות תעודות מזוייפות. 
	\end{itemize}
	
	תצטרכו להכיר סיפור של חייל יהודי בצבאות בעלות הברית. יש דוגמה בספר. 
	
	\section{יחס האוכלוסייה ליהודים בתקופת הפתרון הסופי}
	צפינו בשני סרטונים על ניסיון סטנפורד ומילגרד. ננסה ממנו להבין איך קרה שאנשים רגילים השתתפו ברצח של היהודים בשואה (הייתה להם אפשרות לא להשתתף ברצח). איך קרה, שאנשים רגילים לכאורה, שאתמול לא היו מעלים על דעתם מעשים כאלו, היו רוצחים? בנסיון מילרד היו אנשים ש(חשבו) שהם הרגו אדם אחר, נטו מתוקף סמכות. 
	
	ישנם שלושה דפוסי פעילות של יחס לאוכלוסיה הכללית ביחס ליהודים: 
	\begin{itemize}
		\item משתפי פעולה: מסייעים לנאצים בהשמדה, והסגרת יהודים לנאצים. ברוב המקרים הנאצים פשוט פקרחו על הדברים. 
		\item הרוב הדומם: הגיבו באדישות לנעשה. לא סייעו לנעצים, אך לא עזרו ליהודים. אם ידפוק בדלתו יהודי בלילה, הוא לא יסגיר אותו, אך לא יעזור לו. 
		\item מצילים: אנשים בהצילו יהודים בתקופת השואה, והם מתפצלים לשניים: 
		\begin{itemize}
			\item יהודים שהצילו יהודים. יש כאן קושי רב, והסיכון כאן כפול. רובם היו במחנות פרטיזנים. 
			\item לא יהודים שהצילו יהודים בתקופת השואה, תוך סיכון חיהם ומשפחתם. הכרה בהם כחסידי אומות עולם אם יש תיעוד. היום יש בערך 25 אלף אנשים שהוכרו כחסידי אומות עולם, מתוך כנראה מאות אלפים שסייעו. 
		\end{itemize}
	\end{itemize}
	
	יש דיון על מי מוכר כחסיד אומות עולם. מי שקיבל תשלום ממשי עבור תעודות מזויפות, בבירור לא חסיד אומות עולם, כי למרות הסיכון זה נעשה למטרות רווח. אך אנשים שהצילו יהודים, וניתן להם כסף לדוגמה בשביל לקנות אוכל למשפחה, לרוב לא מוכרים כחסידי אומות עולם. לפי שרית, היא מכירה שני אנשים שאמרו לה מפורשות שהם לא סיפרו על כך שנתנו תמורה (או סתם כסף כדי שההצלה תתאפשר) למי שהציל אותם כדי שייקבלו את התואר. 
	
	אנשים שהוכרו כחסידי אומות עולם זוכים להכרכה ביד ושם ועזרה במידה והם (שבשלב הזה כבר מתו) או בני משפחותיהם (לדוגמה במלחמה באוקראינה) צריכים עזרה. 
	
	להלן המניעים של דפוסי הפעולה: 
	\begin{itemize}
		\item משת''פים: מסורת אנטישמית, הזדהות על המדיניות הנאצים, תגמול כספי, וזיהוי עם הקומוניזם. \textbf{פחד מהנאצים הוא לא גורם לשיתוף פעולה}. כולם פחדו מהנאצים, ומי שפחד באמת בד''כ לא עשה כלום. 
		\item הרוב הדומם היה מונע מאנטישמיות, פחד מהשלטון, פחד משכנים אנטישמים/הלשנה, מצוקה בשל המלחמה, ורצון לזכות ברכוש היהודי. בגלל המלחמה, רבים היו עסוקים בהשרדות של עצמם, ובמקרים רבים אין להם את האמצעיים לעזור לעוד מישהו. ``הפולנים היו רק קצת פחות מסכנים מהיהודים''. מיליוני פולנים נרצחו וחלקם הגדול עבדו בעבודות כפייה. 
		\item מניעי חסידי אומות עולם לרוב היו מונעים מרשות, רצון לסייע למכרים, מניעים דעתיים (לדוגמה אנשי דת וגמרים, ולעיתים גם סתם כתוליים), דרך לבטא התנגבות לנאציזם (מחתרות הוקמו בנידון), ועוד. 
	\end{itemize}
	
	למרות השנים הרבים שעברו, עדיין מתרחשת הכרה בחסידי אומות עולם, כי מתגלים סיפורים חדשים. לרוב ילדהם מקבלים את התעודה ומשתתפים בטקס. 
	
	
	
	
	
	
	
	
	
	
	\ndoc
\end{document}