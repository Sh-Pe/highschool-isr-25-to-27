%! ~~~ Packages Setup ~~~ 
\documentclass[]{article}
\usepackage{lipsum}
\usepackage{rotating}


% Math packages
\usepackage[usenames]{color}
\usepackage{forest}
\usepackage{ifxetex,ifluatex,amssymb,amsmath,mathrsfs,amsthm,witharrows,mathtools,mathdots,centernot}
\usepackage{amsmath}
\WithArrowsOptions{displaystyle}
\renewcommand{\qedsymbol}{$\blacksquare$} % end proofs with \blacksquare. Overwrites the defualts. 
\usepackage{cancel,bm}
\usepackage[thinc]{esdiff}


% tikz
\usepackage{tikz}
\usetikzlibrary{graphs}
\newcommand\sqw{1}
\newcommand\squ[4][1]{\fill[#4] (#2*\sqw,#3*\sqw) rectangle +(#1*\sqw,#1*\sqw);}


% code 
\usepackage{algorithm2e}
\usepackage{listings}
\usepackage{xcolor}

\definecolor{codegreen}{rgb}{0,0.35,0}
\definecolor{codegray}{rgb}{0.5,0.5,0.5}
\definecolor{codenumber}{rgb}{0.1,0.3,0.5}
\definecolor{codeblue}{rgb}{0,0,0.5}
\definecolor{codered}{rgb}{0.5,0.03,0.02}
\definecolor{codegray}{rgb}{0.96,0.96,0.96}

\lstdefinestyle{pythonstylesheet}{
	language=Java,
	emphstyle=\color{deepred},
	backgroundcolor=\color{codegray},
	keywordstyle=\color{deepblue}\bfseries\itshape,
	numberstyle=\scriptsize\color{codenumber},
	basicstyle=\ttfamily\footnotesize,
	commentstyle=\color{codegreen}\itshape,
	breakatwhitespace=false, 
	breaklines=true, 
	captionpos=b, 
	keepspaces=true, 
	numbers=left, 
	numbersep=5pt, 
	showspaces=false,                
	showstringspaces=false,
	showtabs=false, 
	tabsize=4, 
	morekeywords={as,assert,nonlocal,with,yield,self,True,False,None,AssertionError,ValueError,in,else},              % Add keywords here
	keywordstyle=\color{codeblue},
	emph={var, List, Iterable, Iterator},          % Custom highlighting
	emphstyle=\color{codered},
	stringstyle=\color{codegreen},
	showstringspaces=false,
	abovecaptionskip=0pt,belowcaptionskip =0pt,
	framextopmargin=-\topsep, 
}
\newcommand\pythonstyle{\lstset{pythonstylesheet}}
\newcommand\pyl[1]     {{\lstinline!#1!}}
\lstset{style=pythonstylesheet}

\usepackage[style=1,skipbelow=\topskip,skipabove=\topskip,framemethod=TikZ]{mdframed}
\definecolor{bggray}{rgb}{0.85, 0.85, 0.85}
\mdfsetup{leftmargin=0pt,rightmargin=0pt,innerleftmargin=15pt,backgroundcolor=codegray,middlelinewidth=0.5pt,skipabove=5pt,skipbelow=0pt,middlelinecolor=black,roundcorner=5}
\BeforeBeginEnvironment{lstlisting}{\begin{mdframed}\vspace{-0.4em}}
	\AfterEndEnvironment{lstlisting}{\vspace{-0.8em}\end{mdframed}}


% Design
\usepackage[labelfont=bf]{caption}
\usepackage[margin=0.6in]{geometry}
\usepackage{multicol}
\usepackage[skip=4pt, indent=0pt]{parskip}
\usepackage[normalem]{ulem}
\forestset{default}
\renewcommand\labelitemi{$\bullet$}
\usepackage{graphicx}
\graphicspath{ {./} }

\usepackage[colorlinks]{hyperref}
\definecolor{mgreen}{RGB}{25, 160, 50}
\definecolor{mblue}{RGB}{30, 60, 200}
\usepackage{hyperref}
\hypersetup{
	colorlinks=true,
	citecolor=mgreen,
	linkcolor=black,
	urlcolor=mblue,
	pdftitle={Document by Shahar Perets},
	%	pdfpagemode=FullScreen,
}
\usepackage{yfonts}
\def\gothstart#1{\noindent\smash{\lower3ex\hbox{\llap{\Huge\gothfamily#1}}}
	\parshape=3 3.1em \dimexpr\hsize-3.4em 3.4em \dimexpr\hsize-3.4em 0pt \hsize}
\def\frakstart#1{\noindent\smash{\lower3ex\hbox{\llap{\Huge\frakfamily#1}}}
	\parshape=3 1.5em \dimexpr\hsize-1.5em 2em \dimexpr\hsize-2em 0pt \hsize}



% Hebrew initialzing
\usepackage[bidi=basic]{babel}
\PassOptionsToPackage{no-math}{fontspec}
\babelprovide[main, import, Alph=letters]{hebrew}
\babelprovide[import]{english}
\babelfont[hebrew]{rm}{David CLM}
\babelfont[hebrew]{sf}{David CLM}
%\babelfont[english]{tt}{Monaspace Xenon}
\usepackage[shortlabels]{enumitem}
\newlist{hebenum}{enumerate}{1}

% Language Shortcuts
\newcommand\en[1] {\begin{otherlanguage}{english}#1\end{otherlanguage}}
\newcommand\he[1] {\she#1\sen}
\newcommand\sen   {\begin{otherlanguage}{english}}
	\newcommand\she   {\end{otherlanguage}}
\newcommand\del   {$ \!\! $}

\newcommand\npage {\vfil {\hfil \textbf{\textit{המשך בעמוד הבא}}} \hfil \vfil \pagebreak}
\newcommand\ndoc  {\dotfill \\ \vfil {\begin{center}
			{\textbf{\textit{שחר פרץ, 2025}} \\
				\scriptsize \textit{קומפל ב־}\en{\LaTeX}\,\textit{ ונוצר באמצעות תוכנה חופשית בלבד}}
	\end{center}} \vfil	}

\newcommand{\rn}[1]{
	\textup{\uppercase\expandafter{\romannumeral#1}}
}

\makeatletter
\newcommand{\skipitems}[1]{
	\addtocounter{\@enumctr}{#1}
}
\makeatother

%! ~~~ Document ~~~

\author{שחר פרץ}
\title{\textit{היסטוריה 36}}
\begin{document}
	\maketitle
	
	\section{המרידות בגטו}
	
	\subsection*{קשיי המרד}
	המרידות לא החלו ברגע שהתחיל הפתרון הסופי. הפתרון הסופי התחיל בברה''מ ב־1941 והמרידות ב־1943. יש לזה כמה סיבות. 
	\begin{itemize}
		\item הטעייה והסוואה מצד הנאצים. 
		\item זה לא התחיל בגטאות. זה התחיל במשאיות. 
		\item גם כאשר כבר היו מחנות, הם התנהגו כאילו הם הולכים ל''התיישבות מחדש במזרח'', הם הביאו מזוודות, וכאשר הגיעו למחנה הכניסו אותם למקלחות עם סבון וזרקו אז. 
		\item ה''יהודי הישן'' באג'נדה ובתרבות שלה לא חברה מורדת, אלא חברה שמורידה את הראש עד שיעבור זעם. זו הייתה ההתנהגות ביחס לפוגרומים. 
	\end{itemize}
	הצעירים שהתחילו לקבל מידע מגטאות אחרים באמצעות קשריות (ומאוחר יותר, אפילו ממחנאות ההשמדה עצמה, כמו טרבלינקה וסובידור, מאושוויץ ברחו ממש מעטים). אך גם אז אנשים לא האמינו להם. פרוטוקולים של אוושויץ הגיעו אף לשליטי מדינות אחרות, ולא האמינו להם. זה לא נתפס באותה התקופה, ולא היה תקדים למחנות שנועדו לרצח עם. גם אחרי השואה, אנשים בארץ התקשו להאמין שהשואה אכן התרחשה בממדים הללו. התחילו להקשיב להם במשפט אייכמן. 
	
	גם כאשר הידיעות התחילו, היו עוד כמה בעיות. 
	\begin{itemize}
		\item \textbf{מחסור בנשק} – הגטו היה סגור ולכן היה קשה להשיג נשק. היה ייצור מקומי של נשק ובקבוקי תבערה (בקבוקי מולוטוב), והפולנים הבריחו קצת. 
		\item \textbf{חוסר ידע קרבי} – באופן מסורתי היהודים לא התגייסו לצבא. רק מעטים שירתו בצבא הפולני, והיה קשה להתאמן בגטו. 
		\item \textbf{התנגדות פנימית} – הרבה מהאנשים לא היו מעוניינים במרד, ואף יהודים הסגירו את חברי המחתרת. הסיבה? המשמעות של מרד היא מוות לכל הגטו. היה חשש שבמידה והנאצים יגלו את אשר מתרחש הם יסגירו וישמידו את כל הגטו. 
	\end{itemize}
	
	\subsubsection*{מיקום המרד}
	\begin{itemize}
		\item עם במרד היה מתרחש בגטו – אין סיכוי להצלחה, והגטו על תושביו יחוסל. עם זזאת, נישאר עם המשפחה (עד שישמידו אותה בענישה קולקטיבית). 
		\item לצאת להלחם עם הפרטיזנים ביערות (עם כי היו גטאות בלי קרבה ליער שם זה לא התאפשר). כאן יש סיכוי שהגטו לא יחוסל, ואף סיכוי להישאר בחיים. אבל לא כולם יכולים לצאת והמשפחה נשארת מאחורה. 
		
		הלחימה עם הפרטיזנים לא פשוטה. חלק מהפרטיזנים היו אנטישמיים והרגו את היהודים שבאו להלחם. אין מה לנסות להגיע לפרטיזנים אם אין לך נשק. אבל האופציה השנייה לא ממש עדיפה. 
	\end{itemize}
	
	\subsubsection*{עיתוי המרידות}
	משום שהמרד הוא מוות וודאי, לרוב עיתוי המרד היה כאשר הנאצים החליטו לחסל את הגטו (ואי לך תלוי בנאצים ולא בנו) דהיינו ''לפנות אותו``. 
	
	עם זאת, יש יוצא דופן, הוא מרד גטו ורשה. 
	
	\subsubsection*{מרד גטו ורשה}
	גטו ורשה הוקם על 3\% משטח ורשה, שהיו 30\% מתושבי העיר. רובם הושמדו בטרבלינקה, והחלו להגיע ידיעות על הרצח בטרבלינקה. זה לקח זמן כי היהודים נרצחו ברגע שהגיעו לטרבלינקה. הייחוד של מרד גט וורשה: 
	\begin{itemize}
		\item במרד לקחו חלק כל תושבי הגטו, לא רק 700 לוחמים, אלא מרי עממי של 60K יהודים. הוא היה המרד העירוני הראשון באירופה הכבושה ע''י הנאצים. זה לא אומר שכל הגטו נלחם, אלא שיתף פעולה. הם עזרו לבנות את התשתית והסתתרו מתחת לאדמה בבונקרים, והיו כוח עורף שסייע בדברים אחרים. 
		\item הדרך בה הנאצים הצליחו להכניע את הבונקרים, היה באמצעות להביורים להוכנסו לבונקרים. לקח מה־19 באפריל ועד ה־16 במאי, כחודש ימים (לפורפורציות פולין נפלה תוך שלושה שבועות). 
		\item הלוחמים בגטו לא הכינו דרך נסיגה. יש ארגון מטורף של המרד עם תכנונים ארוכים מראש, והיה ברור שנסיגה היא איננה מטרה, אנשים הצליחו בסוף לברוח משם, בסיום הלחימה, אבל זו לא הייתה מטרה של המרד ולרוב הם ברחו דרך תעלות ביוב. 
		\item באופן יחסי הרבה גרמנים נהרגו. הם אפילו הצליחו להשמיד טנק שנכנס. למעלה ממאה גרמנים מתו במרד. 
	\end{itemize}
	הנאצים לא ציפו לזה, והייתה הפתעה מצדם. 
	
	מרד גטו ורשה הוא לא אירוע שנשאר בין הגטו לבין הנאצים, אלא הגיע לעיתונים ועודד מרידות נוספות בגטאות נוספים. 
	
	מרדכי אנילביץ' פיקד על הגטו, ונהרג בו. במכתבו האחרון הוא כותב שהוא ששאיפותיו התמלאו ושהצליח להיות מ''ראשוני הלוחמים היהודים בגטו''. 
	
	הגטו הורכב מכל מני תנועות נוער, לרוב ציוניות, אך גם הבון נלחמו בגטו ורשה. גם בגטו ורשה הנרחב היו פערים אידיאולוגיים, והוא היה מורכב בהרבה מאיך שהוא מוצג בד''כ. ההחלטות שבוצעו במרד שיקפו במקרים רבים את זהות היהודי החדש, בהתאם לאידיאולוגיה של התנועות הציוניות. 
	
	בגטו ורשה לחמו תנועות (גם ציוניות) נוספות פרט ל''אייל'' הציונית. אבל הם די נמחקו מההיסטוריה. גם בין התנועות הציוניות היו פערים אידיאולוגים, נאמר בין ההגנה, הפלמ''ח האצ''ל והלח''י. במרד גטו ורשה גם לחם ארגון ובשם האצ''י (הארגון הצבאי היהודי). אפילו בסיטואציה הקיצונית הזו האצ''י ואייל לא התאחדו ונלחמו לבד. כאשר הסתיים המרד, מי שסיפר את הסיפור במדיתנו הקדושה – היו שורדי אייל, ההנהגה (מפא''י) בארץ, והאצ''י נמחקו מההיסטוריה. בשנות ה־90 בערך התחילו לדייק את הסיפור, ויצא הסרט ``הסיפור של סופר'' (לפני כמה שנים). האצ''י היה לארגון החמוש ביותר בגטו בזכות המנהרות שבנה. לפי הסרט, בן גוריון לא אישר להכניס ספרי לימוד עם המילים אצ''ל, בית''ר וכיו''ב. 
	
	שרית מדגישה כאן דבר – כל מורה אשר נכנס לכיתה מגיע עם מטען ערכי/אידיאולוגי, לכל ספר לימוד יש אג'נדה, לכל תוכנית לימודים יש אג'נדה. לא להאמין למה שמספרים לכם כסיפור הנכון ולא כסיפור המלא. ''הסיפור הוא תמיד מורכב``. 
	
	\section{דילמת היודנראטים בזמן ביצוע הפתרון הסופי}
	\subsubsection{תזכורות}
	נכיר שלושה ראשי יודנראטים – חיים רומקובסקי מלודג', אדם צ'רניקוב מורשה ויעקב גנץ מגטו וילנה. נחזור בקצרה על הנושא: 
	\begin{itemize}
		\item מציאת איזון בין חובת הציות לגרמנים לבין טובת הגטו. הקושי העיקרי לא היה האם למלא את הפקודות, אלא כיצד למלא את הפקודות. לא הייתה אופציה לא לציית (כי ירצחו אותם ומשפחתם ואז יחליפו אותם) אך בתוך הסמכויות שלהם ניסו לסייע ככל יכולתם (לדוגמה מטבחים ציבוריים, כיתות לימוד, הקמת מרפאות ובתי חולים, ולקבל בשתיקה דברים שלא הקימו). 
		\item כיצד להתמודד עם יחס הציבור היהודי ששנא ובז להם – היה להם אורח חיים ראוותני, זכויות יתר, יותר אוכל ושיתוף פעולה עם הנאצים. 
	\end{itemize}
	תזכורת – כשקמה המדינה נתפסו היודנראטים כמשתפי פעולה, ואותם הניצולים שהיגעו ושרדו נתפסו כך. החוק לעשיית דין בנאצים ובעוזריהם כתוצאה מכך. אנשים הגיעו למשטרה ודיווחו על כך שראו משת''פים יודנראטים, ולא יכלה לעשות עם זה כלום משום שלא היה חוק לטיפול בנושא. עשרות יהודים שורדי שואה שהיו חברי יודנראט נשפטו במדינת ישראל בשנות ה־50, וחלק ניכר מהם נידונו למאסר. אחד מהם נידון למוות (לבסוף הוא לא הוצא להורג כי היה חולה וקיבל חנינה). 
	
	\subsubsection*{לא תזכורות}
	בתקופת הפתרון הסופי (1941-1945) נדרשו ליודנראטים לספק עשרות אלפי אנשים לגרמנים ל''התיישבות מחדש''. האם להמשיך לבצע את ההוראות? האם לנסות לקיים מדיניות שבה מצילים לפחות חלק מתושבי הגטו? את מי שולחים להשמדה? באופן כללי הגירושים החלו בינואר 1942, בהתחלה בעיקר לחלמנו. 
	
	גירוש גטו ורשה החל ב־22 ביולי 1942. בהתחלה נאלצו לספק 6K ביום ולאחר כן 10K. 
	יעקב גנץ החליט לענות על השאלה באמצעות התאבדות. לאחר התאבדותו תוך 58 יום נשלחו לטרבלינקה כ־300K יהודים. 
	
	ליודנראטים במקרים רבים הייתה את הגישה של ``עבודה כהצלה'' שבה ניסו להועיל לנאצים בבניית מפעלים בתקווה שזה יחזיק מספיר זמן (בתקופה הזו בעלות הברית החלו לנצח במזרח ותהקדם מערבה). 
	
	אז מה קרה בגטאות ששרדו הרבה זמן? נדבר על גטו לודג', שאותו ניהל חיים רומוקובסקי. האדם שנוי במחלוקת, כנראה היה כוחני. שורדי לודג' מספרים התנהגות ``מאוד לא סימפטית'' להתנהגות של אותו האיש ביחס לנשים יהודיות, בעיקר צעירות (ביחד עם סיפורים לא טובים אחרים). עם זאת, בעקבות האידיאולוגיה של ``עבודה כהצלה'' לודג' שרד הכי מאוחר וניצלו ממנו הכי הרבה אנשים. יש שתי סיבות. 
	\begin{itemize}
		\item מרבית יהודי לודג' הושמדו בחלמנו, וכאשר הוא נסגר הם הועברו לאשוויץ. 
		\item באושוויץ הייתה סלקציה, וככל שהגעת יותר מאוחר (ומלודג' הגיעו מאוחר) היה להם יותר סיכוי לשרוד את התנאים. 
	\end{itemize}
	רומוקובסקי נאם נאום הידוע בתור ``נאום הילדים''. בנאום רומוקובסקי בא ואומר – אני צריך לשלוח 20 אלף יהודים להשמדה. הוא מבהיר שאלו ימותו. הוא ביקש מהאימהות להביא את ילדיהן למות, וכן את הזקנים והחולים, ולהשאיר בגטו את מי שאפשר להציל. ``אני מוכרח לבצע את הניתוח הקשה השותת דם, אני מוכרח לקטוע איברים, בכדי להציל את הגוף! אני מוכרח ליטול ילדים ואם לא, עלולים להילקח, חס ולשום, גם אחרים''. ``אך כיוון שלא היינו מודרגים על ידי המחשב ``כמה יאבדו'' אלא ``כמה ניתן להציל?'' הגענו אנחנו [...] למסקנה שיהיה הדבר קשה ככל שיהיה, אנו מוכרחים לקבל את ביצוע הגזירה לידינו.''. 
	
	לא כל היודנראטים אמרו וסיפרו ליהודים שהם הולכים להשמדה, בגלל הסתכנות במרד ובלגנים. עם זאת זה אפשר ליהודים לברוח. 
	
	
	
	
	
	
	\ndoc
\end{document}