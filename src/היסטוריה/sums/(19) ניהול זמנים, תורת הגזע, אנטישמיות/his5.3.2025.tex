%! ~~~ Packages Setup ~~~ 
\documentclass[]{article}
\usepackage{lipsum}
\usepackage{rotating}


% Math packages
\usepackage[usenames]{color}
\usepackage{forest}
\usepackage{ifxetex,ifluatex,amsmath,amssymb,mathrsfs,amsthm,witharrows,mathtools,mathdots}
\WithArrowsOptions{displaystyle}
\renewcommand{\qedsymbol}{$\blacksquare$} % end proofs with \blacksquare. Overwrites the defualts. 
\usepackage{cancel,bm}
\usepackage[thinc]{esdiff}


% tikz
\usepackage{tikz}
\usetikzlibrary{graphs}
\newcommand\sqw{1}
\newcommand\squ[4][1]{\fill[#4] (#2*\sqw,#3*\sqw) rectangle +(#1*\sqw,#1*\sqw);}


% code 
\usepackage{listings}
\usepackage{xcolor}

\definecolor{codegreen}{rgb}{0,0.35,0}
\definecolor{codegray}{rgb}{0.5,0.5,0.5}
\definecolor{codenumber}{rgb}{0.1,0.3,0.5}
\definecolor{codeblue}{rgb}{0,0,0.5}
\definecolor{codered}{rgb}{0.5,0.03,0.02}
\definecolor{codegray}{rgb}{0.96,0.96,0.96}

\lstdefinestyle{pythonstylesheet}{
	language=Java,
	emphstyle=\color{deepred},
	backgroundcolor=\color{codegray},
	keywordstyle=\color{deepblue}\bfseries\itshape,
	numberstyle=\scriptsize\color{codenumber},
	basicstyle=\ttfamily\footnotesize,
	commentstyle=\color{codegreen}\itshape,
	breakatwhitespace=false, 
	breaklines=true, 
	captionpos=b, 
	keepspaces=true, 
	numbers=left, 
	numbersep=5pt, 
	showspaces=false,                
	showstringspaces=false,
	showtabs=false, 
	tabsize=4, 
	morekeywords={as,assert,nonlocal,with,yield,self,True,False,None,AssertionError,ValueError,in,else},              % Add keywords here
	keywordstyle=\color{codeblue},
	emph={var, List, Iterable, Iterator},          % Custom highlighting
	emphstyle=\color{codered},
	stringstyle=\color{codegreen},
	showstringspaces=false,
	abovecaptionskip=0pt,belowcaptionskip =0pt,
	framextopmargin=-\topsep, 
}
\newcommand\pythonstyle{\lstset{pythonstylesheet}}
\newcommand\pyl[1]     {{\lstinline!#1!}}
\lstset{style=pythonstylesheet}

\usepackage[style=1,skipbelow=\topskip,skipabove=\topskip,framemethod=TikZ]{mdframed}
\definecolor{bggray}{rgb}{0.85, 0.85, 0.85}
\mdfsetup{leftmargin=0pt,rightmargin=0pt,innerleftmargin=15pt,backgroundcolor=codegray,middlelinewidth=0.5pt,skipabove=5pt,skipbelow=0pt,middlelinecolor=black,roundcorner=5}
\BeforeBeginEnvironment{lstlisting}{\begin{mdframed}\vspace{-0.4em}}
	\AfterEndEnvironment{lstlisting}{\vspace{-0.8em}\end{mdframed}}


% Deisgn
\usepackage[labelfont=bf]{caption}
\usepackage[margin=0.6in]{geometry}
\usepackage{multicol}
\usepackage[skip=4pt, indent=0pt]{parskip}
\usepackage[normalem]{ulem}
\forestset{default}
\renewcommand\labelitemi{$\bullet$}
\usepackage{titlesec}
\titleformat{\section}[block]
{\fontsize{15}{15}}
{\sen \dotfill (\thesection)\!\!\she}
{0em}
{\MakeUppercase}
\usepackage{graphicx}
\graphicspath{ {./} }


% Hebrew initialzing
\usepackage[bidi=basic]{babel}
\PassOptionsToPackage{no-math}{fontspec}
\babelprovide[main, import, Alph=letters]{hebrew}
\babelprovide[import]{english}
\babelfont[hebrew]{rm}{David CLM}
\babelfont[hebrew]{sf}{David CLM}
\babelfont[english]{tt}{Monaspace Xenon}
\usepackage[shortlabels]{enumitem}
\newlist{hebenum}{enumerate}{1}

% Language Shortcuts
\newcommand\en[1] {\begin{otherlanguage}{english}#1\end{otherlanguage}}
\newcommand\sen   {\begin{otherlanguage}{english}}
	\newcommand\she   {\end{otherlanguage}}
\newcommand\del   {$ \!\! $}

\newcommand\npage {\vfil {\hfil \textbf{\textit{המשך בעמוד הבא}}} \hfil \vfil \pagebreak}
\newcommand\ndoc  {\dotfill \\ \vfil {\begin{center} {\textbf{\textit{שחר פרץ, 2024}} \\ \scriptsize \textit{נוצר באמצעות תוכנה חופשית בלבד}} \end{center}} \vfil	}

\newcommand{\rn}[1]{
	\textup{\uppercase\expandafter{\romannumeral#1}}
}

\makeatletter
\newcommand{\skipitems}[1]{
	\addtocounter{\@enumctr}{#1}
}
\makeatother

%! ~~~ Math shortcuts ~~~

% Letters shortcuts
\newcommand\N     {\mathbb{N}}
\newcommand\Z     {\mathbb{Z}}
\newcommand\R     {\mathbb{R}}
\newcommand\Q     {\mathbb{Q}}
\newcommand\C     {\mathbb{C}}

\newcommand\ml    {\ell}
\newcommand\mj    {\jmath}
\newcommand\mi    {\imath}

\newcommand\powerset {\mathcal{P}}
\newcommand\ps    {\mathcal{P}}
\newcommand\pc    {\mathcal{P}}
\newcommand\ac    {\mathcal{A}}
\newcommand\bc    {\mathcal{B}}
\newcommand\cc    {\mathcal{C}}
\newcommand\dc    {\mathcal{D}}
\newcommand\ec    {\mathcal{E}}
\newcommand\fc    {\mathcal{F}}
\newcommand\nc    {\mathcal{N}}
\newcommand\vc    {\mathcal{V}} % Vance
\newcommand\sca   {\mathcal{S}} % \sc is already definded
\newcommand\rca   {\mathcal{R}} % \rc is already definded

\newcommand\prm   {\mathrm{p}}
\newcommand\arm   {\mathrm{a}} % x86
\newcommand\brm   {\mathrm{b}}
\newcommand\crm   {\mathrm{c}}
\newcommand\drm   {\mathrm{d}}
\newcommand\erm   {\mathrm{e}}
\newcommand\frm   {\mathrm{f}}
\newcommand\nrm   {\mathrm{n}}
\newcommand\vrm   {\mathrm{v}}
\newcommand\srm   {\mathrm{s}}
\newcommand\rrm   {\mathrm{r}}

\newcommand\Si    {\Sigma}

% Logic & sets shorcuts
\newcommand\siff  {\longleftrightarrow}
\newcommand\ssiff {\leftrightarrow}
\newcommand\so    {\longrightarrow}
\newcommand\sso   {\rightarrow}

\newcommand\epsi  {\epsilon}
\newcommand\vepsi {\varepsilon}
\newcommand\vphi  {\varphi}
\newcommand\Neven {\N_{\mathrm{even}}}
\newcommand\Nodd  {\N_{\mathrm{odd }}}
\newcommand\Zeven {\Z_{\mathrm{even}}}
\newcommand\Zodd  {\Z_{\mathrm{odd }}}
\newcommand\Np    {\N_+}

% Text Shortcuts
\newcommand\open  {\big(}
\newcommand\qopen {\quad\big(}
\newcommand\close {\big)}
\newcommand\also  {\text{\en{, }}}
\newcommand\defi  {\text{\en{ definition}}}
\newcommand\defis {\text{\en{ definitions}}}
\newcommand\given {\text{\en{given }}}
\newcommand\case  {\text{\en{if }}}
\newcommand\syx   {\text{\en{ syntax}}}
\newcommand\rle   {\text{\en{ rule}}}
\newcommand\other {\text{\en{else}}}
\newcommand\set   {\ell et \text{ }}
\newcommand\ans   {\mathscr{A}\!\mathit{nswer}}

% Set theory shortcuts
\newcommand\ra    {\rangle}
\newcommand\la    {\langle}

\newcommand\oto   {\leftarrow}

\newcommand\QED   {\quad\quad\mathscr{Q.E.D.}\;\;\blacksquare}
\newcommand\QEF   {\quad\quad\mathscr{Q.E.F.}}
\newcommand\eQED  {\mathscr{Q.E.D.}\;\;\blacksquare}
\newcommand\eQEF  {\mathscr{Q.E.F.}}
\newcommand\jQED  {\mathscr{Q.E.D.}}

\DeclareMathOperator\dom   {dom}
\DeclareMathOperator\Img   {Im}
\DeclareMathOperator\range {range}

\newcommand\trio  {\triangle}

\newcommand\rc    {\right\rceil}
\newcommand\lc    {\left\lceil}
\newcommand\rf    {\right\rfloor}
\newcommand\lf    {\left\lfloor}

\newcommand\lex   {<_{lex}}

\newcommand\az    {\aleph_0}
\newcommand\uaz   {^{\aleph_0}}
\newcommand\al    {\aleph}
\newcommand\ual   {^\aleph}
\newcommand\taz   {2^{\aleph_0}}
\newcommand\utaz  { ^{\left (2^{\aleph_0} \right )}}
\newcommand\tal   {2^{\aleph}}
\newcommand\utal  { ^{\left (2^{\aleph} \right )}}
\newcommand\ttaz  {2^{\left (2^{\aleph_0}\right )}}

\newcommand\n     {$n$־יה\ }

% Math A&B shortcuts
\newcommand\logn  {\log n}
\newcommand\logx  {\log x}
\newcommand\lnx   {\ln x}
\newcommand\cosx  {\cos x}
\newcommand\cost  {\cos \theta}
\newcommand\sinx  {\sin x}
\newcommand\sint  {\sin \theta}
\newcommand\tanx  {\tan x}
\newcommand\tant  {\tan \theta}
\newcommand\sex   {\sec x}
\newcommand\sect  {\sec^2}
\newcommand\cotx  {\cot x}
\newcommand\cscx  {\csc x}
\newcommand\sinhx {\sinh x}
\newcommand\coshx {\cosh x}
\newcommand\tanhx {\tanh x}

\newcommand\seq   {\overset{!}{=}}
\newcommand\slh   {\overset{LH}{=}}
\newcommand\sle   {\overset{!}{\le}}
\newcommand\sge   {\overset{!}{\ge}}
\newcommand\sll   {\overset{!}{<}}
\newcommand\sgg   {\overset{!}{>}}

\newcommand\h     {\hat}
\newcommand\ve    {\vec}
\newcommand\lv    {\overrightarrow}
\newcommand\ol    {\overline}

\newcommand\mlcm  {\mathrm{lcm}}

\DeclareMathOperator{\sech}   {sech}
\DeclareMathOperator{\csch}   {csch}
\DeclareMathOperator{\arcsec} {arcsec}
\DeclareMathOperator{\arccot} {arcCot}
\DeclareMathOperator{\arccsc} {arcCsc}
\DeclareMathOperator{\arccosh}{arccosh}
\DeclareMathOperator{\arcsinh}{arcsinh}
\DeclareMathOperator{\arctanh}{arctanh}
\DeclareMathOperator{\arcsech}{arcsech}
\DeclareMathOperator{\arccsch}{arccsch}
\DeclareMathOperator{\arccoth}{arccoth}
\DeclareMathOperator{\atant}  {atan2} 

\newcommand\dx    {\,\mathrm{d}x}
\newcommand\dt    {\,\mathrm{d}t}
\newcommand\dtt   {\,\mathrm{d}\theta}
\newcommand\du    {\,\mathrm{d}u}
\newcommand\dv    {\,\mathrm{d}v}
\newcommand\df    {\mathrm{d}f}
\newcommand\dfdx  {\diff{f}{x}}
\newcommand\dit   {\limhz \frac{f(x + h) - f(x)}{h}}

\newcommand\nt[1] {\frac{#1}{#1}}

\newcommand\limz  {\lim_{x \to 0}}
\newcommand\limxz {\lim_{x \to x_0}}
\newcommand\limi  {\lim_{x \to \infty}}
\newcommand\limh  {\lim_{x \to 0}}
\newcommand\limni {\lim_{x \to - \infty}}
\newcommand\limpmi{\lim_{x \to \pm \infty}}

\newcommand\ta    {\theta}
\newcommand\ap    {\alpha}

\renewcommand\inf {\infty}
\newcommand  \ninf{-\inf}

% Combinatorics shortcuts
\newcommand\sumnk     {\sum_{k = 0}^{n}}
\newcommand\sumni     {\sum_{i = 0}^{n}}
\newcommand\sumnko    {\sum_{k = 1}^{n}}
\newcommand\sumnio    {\sum_{i = 1}^{n}}
\newcommand\sumai     {\sum_{i = 1}^{n} A_i}
\newcommand\nsum[2]   {\reflectbox{\displaystyle\sum_{\reflectbox{\scriptsize$#1$}}^{\reflectbox{\scriptsize$#2$}}}}

\newcommand\bink      {\binom{n}{k}}
\newcommand\setn      {\{a_i\}^{2n}_{i = 1}}
\newcommand\setc[1]   {\{a_i\}^{#1}_{i = 1}}

\newcommand\cupain    {\bigcup_{i = 1}^{n} A_i}
\newcommand\cupai[1]  {\bigcup_{i = 1}^{#1} A_i}
\newcommand\cupiiai   {\bigcup_{i \in I} A_i}
\newcommand\capain    {\bigcap_{i = 1}^{n} A_i}
\newcommand\capai[1]  {\bigcap_{i = 1}^{#1} A_i}
\newcommand\capiiai   {\bigcap_{i \in I} A_i}

\newcommand\xot       {x_{1, 2}}
\newcommand\ano       {a_{n - 1}}
\newcommand\ant       {a_{n - 2}}

% Linear Algebra
\DeclareMathOperator{\chr}    {char}

\newcommand\lra       {\leftrightarrow}
\newcommand\chrf      {\chr(\F)}
\newcommand\F         {\mathbb{F}}
\newcommand\co        {\colon}
\newcommand\tmat[2]   {\cl{\begin{matrix}
			#1
		\end{matrix}\, \middle\vert\, \begin{matrix}
			#2
\end{matrix}}}

\makeatletter
\newcommand\rrr[1]    {\xxrightarrow{1}{#1}}
\newcommand\rrt[2]    {\xxrightarrow{1}[#2]{#1}}
\newcommand\mat[2]    {M_{#1\times#2}}
\newcommand\tomat     {\, \dequad \longrightarrow}
\newcommand\pms[1]    {\begin{pmatrix}
		#1
\end{pmatrix}}

\DeclareMathOperator{\Sp}     {span} 
\DeclareMathOperator{\sgn}    {sgn} 
\DeclareMathOperator{\row}    {Row} 
\DeclareMathOperator{\adj}    {adj} 
\DeclareMathOperator{\rk}     {rank} 
\DeclareMathOperator{\col}    {Col} 
\DeclareMathOperator{\tr}     {tr}

% someone's code from the internet: https://tex.stackexchange.com/questions/27545/custom-length-arrows-text-over-and-under
\makeatletter
\newlength\min@xx
\newcommand*\xxrightarrow[1]{\begingroup
	\settowidth\min@xx{$\m@th\scriptstyle#1$}
	\@xxrightarrow}
\newcommand*\@xxrightarrow[2][]{
	\sbox8{$\m@th\scriptstyle#1$}  % subscript
	\ifdim\wd8>\min@xx \min@xx=\wd8 \fi
	\sbox8{$\m@th\scriptstyle#2$} % superscript
	\ifdim\wd8>\min@xx \min@xx=\wd8 \fi
	\xrightarrow[{\mathmakebox[\min@xx]{\scriptstyle#1}}]
	{\mathmakebox[\min@xx]{\scriptstyle#2}}
	\endgroup}
\makeatother


% Greek Letters
\newcommand\ag        {\alpha}
\newcommand\bg        {\beta}
\newcommand\cg        {\gamma}
\newcommand\dg        {\delta}
\newcommand\eg        {\epsi}
\newcommand\zg        {\zeta}
\newcommand\hg        {\eta}
\newcommand\tg        {\theta}
\newcommand\ig        {\iota}
\newcommand\kg        {\keppa}
\renewcommand\lg      {\lambda}
\newcommand\og        {\omicron}
\newcommand\rg        {\rho}
\newcommand\sg        {\sigma}
\newcommand\yg        {\usilon}
\newcommand\wg        {\omega}

\newcommand\Ag        {\Alpha}
\newcommand\Bg        {\Beta}
\newcommand\Cg        {\Gamma}
\newcommand\Dg        {\Delta}
\newcommand\Eg        {\Epsi}
\newcommand\Zg        {\Zeta}
\newcommand\Hg        {\Eta}
\newcommand\Tg        {\Theta}
\newcommand\Ig        {\Iota}
\newcommand\Kg        {\Keppa}
\newcommand\Lg        {\Lambda}
\newcommand\Og        {\Omicron}
\newcommand\Rg        {\Rho}
\newcommand\Sg        {\Sigma}
\newcommand\Yg        {\Usilon}
\newcommand\Wg        {\Omega}

% Other shortcuts
\newcommand\tl    {\tilde}
\newcommand\op    {^{-1}}

\newcommand\sof[1]    {\left | #1 \right |}
\newcommand\cl [1]    {\left ( #1 \right )}
\newcommand\csb[1]    {\left [ #1 \right ]}
\newcommand\ccb[1]    {\left \{ #1 \right \}}

\newcommand\bs        {\blacksquare}
\newcommand\dequad    {\!\!\!\!\!\!}
\newcommand\dequadd   {\dequad\duquad}
\newcommand\wmid      {\;\middle\vert\;}

\renewcommand\phi     {\varphi}
\newcommand\bcl[1]    {\big(#1\big)}

%! ~~~ Document ~~~

\author{שחר פרץ}
\title{\textit{היסטוריה 18}}
\begin{document}
	\maketitle
	\section{\en{Time Mangement}}
	דוגמה בעבור איתמר: 
	\begin{center}
		\begin{tabular}{|c|c|c|c|c|c|c|}
			\hline ראשון & שני & שלישי & רביעי & חמישי & שישי & שבת \\
			\hline לימודים עד 16:20 &14:00&14:10&16:20&16:20&פרויקט&\\
			\textbf{1} + 2 חור& \textbf{3} & \textbf{1} & \textbf{1} &\textbf{1}&\textbf{5}&\textbf{4}\\
			דר כושר 19:00-21:00 && צהריים 16:00-20:30. & חדר כושר 19:00-21:30& ישיבת צוות 17:00-21:30 & פעולה 15:30-17:30  & \\
			\hline
			הגשה + כושר & הגשה & מבחן בהיסטוריה &&&&\\
			\textbf{1} & \textbf{4} &&&&&\\\hline
		\end{tabular}
	\end{center}
	להמשיך חודשי. 
	\subsection{חשיבות שיבוץ דברים במערכת: }
	\begin{multicols}{2}
		\begin{enumerate}
			\item נשבץ "זמן מת" – זמן שלא תכננו בו שום דבר, ויורד מסך השעות. לא זמן אוכל/מקלחת; זמן מת לחלוטין לבהייה בקיר, או להשלמת בלת"מים. 
			\item ארוחת צוהריים
			\item ארוחת ערב
			\item למידה שוטפת (ש.ב.)
			\item למידה למבחנים
		\end{enumerate}
	\end{multicols}
	ראשית כל, נוריד מכל יום שעה. 
	נסמן בבולד שעות לימודים. 
	
	נסכום את הזמן עד המבחן בהיסטוריה: 
	שבוע ראשון: 16, שבוע שני: 5 [עד היסטוריה]. סה"כ 21 שעות עד היסטוריה. 
	\begin{multicols}{2}
		\begin{itemize}
			\item 1 סקירה
			\item 5 הנדסת תוכנה
			\item 7 פרויקט
			\item 8 פרויקט
			\item 5 מבחן בהיסטוריה (לפי שרית, לוקח פי 2 יותר)
		\end{itemize}
	\end{multicols}
	
	\subsection{עוד הערות}
	יומן שבועי: דבר מאוד נחמד, אך לא מאפשר לראות את המשך החודש כראוי, ולתכנן בהתאם. גם עם עכשיו ניהול הזמן הנוכחי עובד, בי"א זה לא יעבוד, כי יש יותר מבחנים ומטלות. 
	
	לא ללמוד להיסטוריה, ספרות, תנ"ך, אזרחות וכו' אחד אחרי השני – עדיף שיהיה לשון והיסטוריה, מתמטיקה ותנ"ך או משהו כזה, כי זה "יושב על אותו המקום במוח ועושה סמתוכה". אם אין ברירה, כדאי לעשות הפסקה. 
	
	\section{\en{Back to the Nazis}}
	\subsection{עקרונות האידיאולוגיה הנאצית}
	\begin{multicols}{2}
		\begin{itemize}
			\item תורת הגזע
			\item מרחב המחייה
			\item אנטישמיות מודרנית
			\item המנהיג "פיהרר פרנציפ"
			\item הלאום/האומה כערך עליון
			\item שלילת הדמוקרטיה והקומוניזם
		\end{itemize}
	\end{multicols}
	
	בניגוד למה שכתוב בספר הלימוד, הסדר החדש הוא לא עקרון – אלא רעיון – ומרחב המחייה נכלל בו. 
	
	\subsection{תורת הגזע}
	צפינו בדוגמה מהארי פוטר שעקרון "תורת הגזע" נוכח בו. הנאצים לא המציאו את תורת הגזע, אך הייתה להם גישה משלהם. הגזע התחלק לשלושה סוגים: 
	\begin{itemize}
		\item הגזע הארי – הגזע העליון, יוצרי התקבות. גם בו קיימת חלוקה פנימית, לגרמנים ואוסטרים, לעומת ארים ממדינות אחרות (נורדים). 
		\item נושאי התרבות, הסלאבים – גזע שמסוגל לקיים את התרבות אבל לא לייצר אותה בעצמו. 
		\item בתחתית הפרדמיה, הגזע הנחות שסופו להיכחד – צוענים, אנטי־סוציאליסטים, נכים וכו'. 
	\end{itemize}
	היהודים לא נמצאים הפירמידה הזו – הם אנטי־גזע, לא בני אדם. 
	
	מה שמשותף לסלאביים הוא השפה. השפות הסלאביות דומות; רוסית, פולנית, אוקראינית וכו'. רוב דוברי השפות יוכלו להבין אחד את השני. גם כן השפות השמיות (ערבית ועברית לדוגמה) דומות במיוחד. 
	
	ע"פ האידיאולוגיה הנאצית, כתוצאה מתורתו של דארווין, בני בני האדם מתקיימת מלחמת קיום כמו בטבע. בני האדם מחולקים לגזעים שונים אשר אינם שווים בכוחם וביניהם מתנהל מאבק שמטרתו לשרוד המלחמת הקיום. על כן, יש לשמור על הגזע הארי הטוב יותר ולשמור על טוהר גזעו. הגזע הארי בעל שלמות פיזית ורוחנית, הוא יצר את התרבות, וחובה עליו להישאר טהור ולשלוט. הסלאביים נועדו לשרת את האריים. התפקיד של הצוענים וכו' – לא להיות. גם יהודים, צריך להשמיד. 
	
	הנאצים הפכו את תורת הגזע למדע. עד היום יש אנשים שמאמינים בזבל הזה. לכן, הנאצים המציאו קריטריונים ומכשריים לקביעת הגזע – מכשירים למדידת רוחב גשר האף, מכשיר למדידת רוחב המצח, ועוד. למדו אותם מגיל צעיר. התפישה שמדברת על היהודים, טענה שהיהודים חדרו לחברה, מנסים להשתלב ולהשתלט עליה, ולטמא את הדם הארי. לכן, צריך אמצעיים מדעיים בשביל לזהות את היהודים שדומים לגזע הארי ונראים אותו הדבר. על כן מגיל צעיר במיוחד למדו לזהות את הגזעים השונים. 
	
	\subsection{אנטישמיות}
	האנטישמיות הנאצית היא אנטישמיות גזענית, אך מהווה עקרון בפני עצמו. תורת הגזע והנאטישמיות הינם שני דברים שונים – טוהר האדם לאו דווקא קשור ליהודים. ליהודים יש דיומיים של חיות – עבר, טפיל, עכברוש וכו'. האידיאולוגיה אנצית עושה להיהודים דה־הומניזציה. ילד בגיל 10 ב־1933 כשהנאצים עלו לשלטון, יהיה בן 16 עם פתיחת המלחמה ב־1939. לדידו, יהודים אינם בני אדם. לא כל הגרמנים האמינו בלב שלם באידיאולוגיה הנאצית; ילדים יותר, אך גם זה תלוי בבית. אף קיימים גרמנים מעטים שהיו לחסידי אומות עולם. היהודים המציאו את הליברליזם והקומוניזם לטענת הנאצים (מארקס היה יהודי לדוגמה) ומטרתם להחריב את האנושות. 
	
	להלן בעית מתמטיקה מספק ילדים משנת 1935: 
	"מטוס סטוקה שעומד להמריא, נושא 12 תריסרים של פצצות, שכל אחת שוקלת 10 ק"ג. המטוס ממריא לוורשה ,המרכז את היהדות הבין לאומית. הוא מפציץ את העיר. בשעת ההמראה, כשמיכל הדלק שלו מכיל 1000 ק"ג דלק, שוקל המטוס 8 טונות. עם שובו ממסע הצלב נותאו עדיין במיכלו 230 ק"ג דלק. מה משקלו של המטוס כאשר הוא ריק?". 
	
	בעיה שכזו היא תעמולה, שטיפת מוח, שמנרמלת את הרעיון של להפציץ את מרכז היהודות הבין לאומית – זה מסע צלב, אין בעיה בכך. זה היה שיעור מתמטיקה – לא שיעור על תורת הגזע – ובו גם החדירו את הרעיון של קנוניה יהודית בין לאומית. 
	
	מיין־קאמפף הופך להיות תורה בגרמניה הנאצים. דוגמה לאשר נכתב בו: "שעות ארוכות אורב היהודי בעל השיעור השחור, לנערה שאינה חודשת לכל רע, והוא מבצע בה מעשה מכונה, כשפניו קורנות תאווה שטנית והי, הנעשה, נשארת תמאה לכל ימי חייאה, ובכך הצליח היהודי לגזה מחיק עימה". הוא ממשיך ומתאר כיצד היהודי מנסה לנוון את היסודות הגזעיים של העמים הטאורים, ולהביא לעולם ייצואים שהם "תערובת בין אדם וקוף". 
	
	\subsection{דוגמאות למימוש תורת הגזע}
	
	ישנו מושג שנקרא T4 (הבניין שהתרחב בו) – מבצע לרצח אנשים "אסצויאליים" – אריים, גרמנים, עם מוגבלויות נפשיות ופיזיות. האנשים האלו הושמדו בתאי גזים. זו הייתה אחת הפעמים היחידות שבה מחאה מלמטה השיגה משהו. אלפי ילדים ומבוגרים אריים נשרפו ונהגרו בתאי גזים. אותם האנשים שתכננו את המבצע, לימים, ממשו את הפתרון הסופי. לרעיון הזה קוראים אוטנזיה. 
	
	מצע לבנסבורן היא תוכנית של הרבאה של הגזע הנאצי. בחורות צעירות בעלות מראה ארי (לעיתים לא באמת טהורות) הורבו עם חיילי SS וילדיהם נתנו למשפחות אריות טהורות. גם ילדים מחטפו לשם המטרה הזו. באופן דומה, היום מבוצעות הרבאות בין כלבים ליצירת גזע טהור של כלבים. 
	
	הארים לא רק בני אדם. הם צריכים להיות מושלמים. האסוציאלים פוגעים בטיהור. ראשית כל, טיהרו את הגזע הארי – האותונזיה קדמה להשמדת היהודים. 
	
	\subsection{דוגמאות למימוש האנטישמיות}
	נביא דוגמאות מספר בשם "הפטריה המורעלת" – סיפור ילדים שכתב יוליס שטרייכר, עורך העיתון האנטישמי "דר שטאימר". בתרבות הגרמנית יוצאים ליער וקוטפים פטריות, ולכן לומדים לזהות פטריה מורעלת כחלק מתרבותם. בספר "הפטריה המורעלת" מוצגות תמונות של היהודים כסוחר רמאי, בעל אף בולט, וכו'. הסיפור סופר "לפני השינה". נתבונן בתמונה  אחרת שמתאתר את משפחות אוטשילד, עם כתר זהב על הראש, חובקת את העולם בטפרים. תמונה אחרת של יהודי שיושב על כסף, עכביש עם מגן דוד שבכוריו נתלים גופות, ועוד. הקריקטורות האלו מציירות תחושה של גועל. שבוע הבא רואים "הדרדסים". 
	
	
\end{document}