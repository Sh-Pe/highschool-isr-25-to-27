%! ~~~ Packages Setup ~~~ 
\documentclass[]{article}
\usepackage{lipsum}
\usepackage{rotating}


% Math packages
\usepackage[usenames]{color}
\usepackage{forest}
\usepackage{ifxetex,ifluatex,amssymb,amsmath,mathrsfs,amsthm,witharrows,mathtools,mathdots,centernot}
\usepackage{amsmath}
\WithArrowsOptions{displaystyle}
\renewcommand{\qedsymbol}{$\blacksquare$} % end proofs with \blacksquare. Overwrites the defualts. 
\usepackage{cancel,bm}
\usepackage[thinc]{esdiff}

% Design
\usepackage[labelfont=bf]{caption}
\usepackage[margin=0.6in]{geometry}
\usepackage{multicol}
\usepackage[skip=4pt, indent=0pt]{parskip}
\usepackage[normalem]{ulem}
\forestset{default}
\renewcommand\labelitemi{$\bullet$}
\usepackage{graphicx}
\graphicspath{ {./} }


% Hebrew initialzing
\usepackage[bidi=basic]{babel}
\PassOptionsToPackage{no-math}{fontspec}
\babelprovide[main, import, Alph=letters]{hebrew}
\babelprovide[import]{english}
\babelfont[hebrew]{rm}{David CLM}
\babelfont[hebrew]{sf}{David CLM}
%\babelfont[english]{tt}{Monaspace Xenon}
\usepackage[shortlabels]{enumitem}
\newlist{hebenum}{enumerate}{1}

% Language Shortcuts
\newcommand\en[1] {\begin{otherlanguage}{english}#1\end{otherlanguage}}
\newcommand\he[1] {\she#1\sen}
\newcommand\sen   {\begin{otherlanguage}{english}}
	\newcommand\she   {\end{otherlanguage}}
\newcommand\del   {$ \!\! $}

\newcommand\npage {\vfil {\hfil \textbf{\textit{המשך בעמוד הבא}}} \hfil \vfil \pagebreak}
\newcommand\ndoc  {\dotfill \\ \vfil {\begin{center}
			{\textbf{\textit{שחר פרץ, 2025}} \\
				\scriptsize \textit{קומפל ב־}\en{\LaTeX}\,\textit{ ונוצר באמצעות תוכנה חופשית בלבד}}
	\end{center}} \vfil	}

\newcommand{\rn}[1]{
	\textup{\uppercase\expandafter{\romannumeral#1}}
}

\makeatletter
\newcommand{\skipitems}[1]{
	\addtocounter{\@enumctr}{#1}
}
\makeatother

%! ~~~ Document ~~~

\author{שחר פרץ}
\title{\textit{היסטוריה 38}}
\begin{document}
	\maketitle
	מה החומר? 
	\begin{itemize}
		\item {פרק א'} – טוטאליטריות ושואה
		\item {פרק ב'} – לאומיות וציונות
		\item {פרק ג'} – בונים מדינה
		\begin{itemize}
			\item מבוא
			\item המאבק על הקמת המדינה (המחתרות, כ''ט בנובמבר, הכל עד מלחמת העצמאות)
			\item מלחמת העצמאות (בפרט הכרזת המדינה, הבעיות הבלתי פתורות, הסכמי שביתת הנשק ועוד) (לא נלמד כנראה)
		\end{itemize}
		\item פרק ד' – סוגיות בתולדות מדינת ישראל והאיזור
		\begin{itemize}
			\item מלחמת ישראל: ששת הימים או כיפור (מלמדים מלחמה אחת)
			\item דה־קולוניזציה וגורל יהודי ארצות האיסלם
			\item עליה וקליטה (בטוח לא נלמד)
		\end{itemize}
	\end{itemize}
	
	בגלל מה שלא נלמד את כל החומר, בוחרים רק 3 שאלות מ־3 פרקים (כל אחד 2-3 שאלות, בהתאם לכמה למדנו). שאלת הטוטאליטריות ושואה שאלת חובה. לכן, \textbf{לא מהמרים}. מי שרוצה יוכל ללמוד באופן עצמאי מהסיכומים של שרית. 
	
	\section{מאבק היישוב היהודי בשלטונות המנדט הבריטי}
	\subsection{רקע}
	קצת שיעור אזרחות, כי כאלו לא היו לנו עד עכשיו. 
	\begin{itemize}
		\item \textbf{דמוקרטיה ישירה: }אזרחים פשוט באים ומצביעים על דברים. לדוגמה ביוון, הגברים החופשיים (לא עבדים) מעל גיל 18 (בערך 30\% מהפוליס) היו האזרחים. הם היו מגיעים לכיכר העיר ומצביעים. 
		\item \textbf{דמוקרטיה עקיפה: }דמוקרטיה באמצעות נציגים. אנחנו נותנים לנציגים ייפוי כוח לייצוג אותנו, על סמך המצע של המפלגה (אם יש). לייפוי כוח קוראים מנדט. 
	\end{itemize}
	
	פעם אחרונה שנטשנו את התנועה הלאומית, זה היה בהצהרת בלפור ויחסי האימפריה העות'מנית. דיברנו על הקבוצה, הקיבוץ, התפתחות השפה, קצת מפלגות, וקצת עולים. עכשיו אנחנו קופצים ל־45. בינתיים אימפריות קמו ונפלו, אז הנה סיכום קצר של השנים האלו: 
	\subsubsection{המנדט הבריטי}
	הבריטים מקבלים כתב מנדט מחבר הלאומים, שהוקם לפי תפישת 14 הנקודות של ווילסון בסוף מלחמת העולם הראשונה.  כמה שאתם חושבים שהאו''ם שמום חלש – חבר הלאומים חלש אף יותר. לא ארה''ב ולא רוסיה היו בו. ב־14 הנקודות דובר על ''זכות להגדרה עצמית של לאומית``. בעקבות זה קמות באירופה מדינות דמוקרטיות (שפרט לאחת מהן, הדמוקרטיה קרסה בכולן עד 1939). 
	
	באיזורנו, יש דבר אחר: שיטת המנדט – ייפוי כוח מחבר הלאומים (היו שלושה סוגי מנדטים, A, B, ו־C, לפי רמת המוכנות של העם) להיות בא''י, עד שהעם תיאורטית מוכן לעצמאות. עקרונית חבר הלאומים (פרקטית הבריטים) מחליט מתי נגמר המנדא, זכות שאחרי מלחמת העולם השנייה עברה לאו''ם שמום. 
	
	אז למה בריטניה קיבלה את הזכות על ישראל, ולא צרפת, או שילוב של שתיהן (כמו בסייקס פיקו)? על סמך הצהרת בלפור. הצהרת הכוונות שלה מתכתבת עם המטרה. 
	
	הבריטים לוקחים גם חלק מא''י ובונים את ממלכת ירדן, בגלל עוד כל מיני התחחיבויות ממלחמת העולם הראשונה. הם גם שלטו במצריים אך יצאו מהר. 
	
	בכתב המנדט, רוב המחוייבות (אך לא כולה) היא להתחייב ליהודים בארץ ישראל. יש שם כל מני דברים המפורטים כדי להגיע לעצמאות: תמיכה, אפשרות עלייה, קרקעות וכו'. יש כמובן גם התחייבות כלפי מי שאינם יהודים, חופש דת, לשמור על מקוומת קדושים וכו'. 
	
	יש בא''י קונספט בשם המשולש הארץ ישראלי. לכובע שלי שלוש פינות: הבריטים (שלטון) – ערבים (הרוב) – יהודים (מיעוט). המיעוט היהודי יותר מפותח מהרוב הערבי. החברה הערבית היא חברה אגררית (חלקאית) נכשלת עם ילודה גבוהה. היהודים חברה יותר מתפתחת כלכלית, פוליטית, ''בכל תחום שאתם רוצים``. ב־1948 כשקמה המדינה, יש פה אשכרה מדינה: בנק, מפלגות, מוסדות שלטון, מערכת בריאות, מערכת חינוך, צבא, כל מה שצריך בשביל מדינה. ההכרזה הייתה הכרזה רשמית וחשובה, אבל המערכות היו קיימות. כל המוסדות האלו קמו בתקופה של המנדט, למרות שאנחנו אוהבים לבקר אותו (''הם עזרו לנו לא מעט, למרות מה שאני אגיד עכשיו``). זאת בניגוד למדינה האחרת שקמה באיזור. 
	
	מעגל האירועים היה בערך ככה: 
	
	\begin{itemize}
		\item עלייה (שלישית, רביעית, חמישית)
		\item מאורעות/פרעות או מאבק לאומי/טרור ערבי (כל אחד מהשמות בהתאם לאינטרס של הדובר)
		\item ספר לבן בריטי חדש (ספר החוקים). השלישי והאחרון יצא ב־1939. 
		\item עוד עלייה, וחוזר חלילה
	\end{itemize}
	
	הנסיגה של הספר הלבן של 1939 מכתב המנדט היא ''מטורפת``. יש שם הגבלת עלייה מאסיבית. קצת לאחריו יצא חוק בשם ''חוק הקרקעות``. החוק קובע שהיהודים לא יכולים לקנות ממי שאינם יהודים (כלומר – אי אפשר להגדיל את היישוב היהודי). פרקטית אוסר התיישבות. 
	
	\subsubsection{המחתרות}
	ב־1945 היו כאן שלוש מחתרות. 
	\begin{itemize}
		\item \textbf{ההגנה: }הוקמה ב־1920. הוקמה ע''י יוצאי השומר. רוב האנשים ביישוב היהודי הם חלק מההגנה, גם מבחינת התפישה הפוליטית. הם היו משוייכים פוליטית למפלגת הפועלים. 
		\item \textbf{האצ''ל: }בשלב מסויים, היו אנשים הנקראו אחר־כך ''הפורשים`` שהקימו את האצ''ל, כי לא הסכימו מבחינה אידיאולוגית עם צורת המאבק בשלטון הבריטי. ההגנה לא פעלו נגד הבריטים אלא אם ידעו שעומדים לפעול כנגדנו. האצ''ל לא. להגנה היו נציגים מול הבריטים כחלק מהאחיזה הפוליטית שלהם. 
		\item \textbf{הלח''י: }ב־1940 כבר יש מלחמה באיזור. דה־פקטו, זו רק בריטניה נגד גרמניה. צרפת לא מאוד רלוונטית. בן־גוריון אמר שאנחנו לא נלחם נגד הבריטים, ולא נשחוק כאן את הכוחות הבריטיים. האצ''ל הסכים להתיישר עם התפיסה הזו. מי שלא הסכים הפך להיות חלק מהלח''י, שאף היו לו נסיונות ליצור קשרים עם הנאצים. זוהי קבוצה יותר קיצונית, ומיליטאנטית. 
		\item \textbf{הפלמ''ח: }הוקם ב־1941 מתוך ההגנה. הבריטים הקימו את הפלמ''ח (ביחד עם היהודים). היא התחילה כמחתרת לגאלית, כארגון צבאי שהוקם ע''י הבריטים להלחם בגרמנים. היה איום שהנאצים יכנסו מאיזור מצריים, ועד אל־עלמיין, היה איום רציני. היה פרויקט מגירה מתוכנן בשם ''מצדה על הכרמל`` – העברת היישוב היהודי לאיזור הכרמל, איזור שמבחינה אסטרטגית אפשר להגן עליו. הבריטים הקימו, חימשו, ואימנו את הפלמ''ח. 
		
		אחרי אל־עלמיין ומבצע לפיד, נעלם האיום. משם ואילך, מבחינת הבריטים הארגון הפך להיות לא לגאלי, והוא ירד למחתרת. גם במלחמת העצמאות, פרט לבריגה ולחיילים אחרים שהגיעו, הפלמ''ח היו בעלי הניסיון הצבאי הגדול ביותר. בניגוד להגנה, שאינם לוחמים, הפלמ''ח היו אנשים שהתאמנו על בסיס קבוע. ארגון יחסית קטן, בשיאו סדר גודל של ארבעה גדודים. 
	\end{itemize}
	מבחינה אידיאולוגית, האצ''ל משוייך לתנועה הרווזיוניסטית (שלאחר מכן הפך לתנועת חירות ואז הליכוד). הלח''י לא משוייך אידיאולוגית־פוליטית. הוא כן יותר קיצוני מהאצ''ל, אבל הוא לא משוייך מפלגתית. 
	
	\subsection{היהודים אחרי מלחמת העולם השנייה}
	בעלות הברית נצחו וכבשו. גרמניה חולקה וכו'. בסופו של יום הסתיימה המלחמה, וישנם מיליוני פליטים שמנסים לחזור לבית. ורשה נראית כמו עזה. ערים חרבו, אנשים ברחו מבתים, ואירופה מלאה בפליטים שחלקם מנסים לחזור חזרה. חלק (קטן מאוד) מהפליטים האלו, הם יהודים. חלקם רוצים לחזור לבתיהם וחלקם לא. אחרי מלחמת העולם השנייה, יש עדיין אירועי אנטישמיות ופוגרומים ביהודים. רוב היהודים שרוצים לחזור לבתיהם, מגורשים, שכן אנשים עברו לחיות שם. רוב היהודים לא רוצים להמשיך לחיות באירופה. 
	
	מי שיכל, עלה לארה''ב. אך שם היו חוקים בשם חוקי ג'ונסון, ולא כי מי שרצה יכל להגיע לשם. הם מגבילים את ההגירה האחוזים מכל סוג אוכלוסיה. היו עוד יהודים שהגרו לאוסטרליה וקנדה. בסופו של דבר באירופה הוקמו מחנות בשם ''מחנות עקורים`` – אנשים שנעקרו מהבתים שלהם. בהתחלה, היו שם אנשים שאינם יהודים. בשלב מסויים הם הופרדו, ולאחר מכן חזרו לבתיהם. האוכלוסיה הלא יהודית חזרה הרבה יותר מהר לבתים שלהם, ולכן בשלב מסויים נשארו שם רק יהודים, רובם רוצים לעלות לארץ. בשלב מסויים הגיעו למחנות האלו אנשים מהיישוב היהודי, שאמנו אותם צבאית (לא בנשק חם, אלא קרב מגע וכו'), לימדו אותם עברית, וכו'. מחנה העקורים האחרון נסגר בשנות ה־50. מי שהחזיק את המחנות האלו היו בעלות הברית, בעיקר ארה''ב, ולכן הם היו מעורבים בצורה מאוד משמעותית בעניין הזה. זה נקרא \textbf{בעיית העקורים}. 
	
	העולם עסוק בבעיית העקורים. גם בגלל שהעולם גילה מה קרה בשואה, וגם מתוך אינטרס כלכלי (מחנות העקורים אלו המון). ישנה בעיה נוספת שמעסיקה את העולם – \textbf{שאלת א''י}. מי ישלוט פה? מה יהיה פה? 
	
	מבחינת התנועה הציונית, הפתרון לבעיית העקורים מאוד פשוט – א''י. העקורים יהודים, תעבירו אותם לארץ. ''קשה? לא קשה`` – חיים כהן. מבחינת הבריטים זה לא הפתרון – יש פה רוב ערבי, ומדינות ערב מסביב. זוכרים את המעגל? תוסיפו על זה את נושא הנפט (שלא היה גרוע כמו שהוא היום), וחלק מהאינטרסים של בריטניה באיזור נקשרו למדינות ערב אלו. פתיחת שערי ההגירה ליהודים פעל נגד האינטרסים של בריטניה. יש כמה אירועים שחשובים להבנה של מה שקורה כאן. 
	
	\begin{itemize}
		\item \textbf{נאום בווין (בריטניה): }
		
		אחרי המלחמה הלייבור עלו לשלטון בבריטניה, שכל השנים תמכו בתנועה הציונית. ואז הם עשו סייק והתעלמו מהיהודים. ''ציפיות נועדו לכריות``, וכל הציפיות התנפצו. לכאורה מלחמת הלייבור הייתה פרו־ציונית, ואנחנו עזרנו לבריטניה במהלך המלחמה. שר החוץ הבריטי, בווין, נאם נאום בפרלמנט הבריטי. 
		
		ב־13 בנובמבר נשא בווין נאום בו פירט את מדיניות ממשלת הלייבור בשאלת א''י. 
		\begin{enumerate}
			\item ''ממשלת הוד רוממותו הקדישה תשומת לב רצינית וממושכת לכל בעיית העדה היהודית שהתעוררה עקב הרדיפות, שאדפוה הנאצים בגרמניה`` – לא \textit{העם היהודי}, אלא \textit{העדה היהודית}. 
			\item ''הבעיה היהודית היא בעיה אנושית גדולה. אין אנו יכולים להשלים את ההשקפה, שיש לגרש את היהודים מאירופה לבלי להניח להם לשוב ולחיות במדינות אלה ללא הפליה ולתרום מכוח יכולתם וכישרונם לקימום אירופה ושגשוגה``: נורא עצוב לנו מה שקרה ליהודים, אך הם צריכים לשקם את המדינות שלהם. 
			\item ''זה מקרוב נתבענו לאשר עליה בקנה מידה נרחב לפלשתינה. בעוד אשר פלשתינה עשויה לתרום את תרומתה לעניין, אין בה כשהיא לעצמה להקיף את כל הבעיה כולה`` – א''י היא פתרון קטן מהבעיה. היהודים שישארו באירופה. 
			\item ''בעיה פלשתינה היא בעיה קשה מאוד מבחינה בין־לאומית – על ממשלת הוד מלכותו הוטל חיוב כפול כלפי היהודים מזה וכלפי הערבים מזה. מאז ניתן המצע אי אפשר למצוא מצע משותף לערבים וליהודים [...]`` – חובתנו כלפי הערבים זהה. 
			\item ''המשך מדיניות הספר הלבן ועד קבלת פתרון מוסכם יש לאפשר כניסתם של 1500 פליטים יהודים בחודש``. 
			\item ''יש לערב את ארה''ב בשאלת א''י ולשתף עימה פעולה במציאת פתרון ראוי`` – תכניסו את ארה''ב לסיפור. 
			
			מלחמת העולם השנייה נגמרה והתחיל דיון על פתיחת חזית שנייה. יש לנו מאבק בין גושים – בריהמ''ש וארה''ב. אחרי המלחמה מצב בריטניה היה בכי רע (אולי קצת יותר טוב מצרפת), והיא הייתה חלק מהצד המערבי של ארה''ב. היא הייתה תלויה בכל ההבטים בארה''ב, ולא יכלה ללכת כנגדה. לכן הם הזמינו את ארה''ב להתעסק בזה. 
			
			אז מה עשו? הקימו ועדה כדי שתסיק מסקנות ואז נתעלם מהן. 
		\end{enumerate}
		
		מה שחשוב אצל בווין, הוא שבניגוד לעמדת התנועה הציונית, מבחינת בריטניה הפתרון לבעיית הפליטים הוא לא הגירה לא''י. 
		
		\item \textbf{הועדה האנגלו־אמריקאית: }בלחץ יהודי ארה''ב (ולא יהודים שהרגישו אשמים על מה שהם ראו), נשיא ארה''ב טרומן החליט לשלוח את ארל הריסון לחקור את בעיית העקורים (בעיקר היהודים, אך גם גרמנים ואוסטרים) באירופה. 
		
		האריסון מגיע לאירופה זמן קצר אחרי המלחמה. הוא נפגש עם הניצולים ועם הזוועות ממחנות ההשמדה, כותב דין וחשבון לנשיא ארה''ב עם המסקנות הבאות: 
		\begin{itemize}
			\item כדאי להפריד בין היהודים לאילו שאינם יהודים (בעיקר גרמנים ואוסטריים)
			\item יש לאשר עליייה לא''י של כמאה אלף עקורים (זה המספר שהיה ידוע באותו הזמן, בפועל היו מאתיים־שלוש מאות אלף)
		\end{itemize}
		
		בוועדה היו 6 בריטים ו־6 אמריקאים. הוועדה באופן מפתיע ומרגש קיבלה את דו''ח האריסון ואישרה אותו (חלק מהבריטים הצביעו בעד, שכן הם פרקטית מדינת חסות אמריקאית). המסקנות:
		\begin{itemize}
			\item יש לאשר עלייה של כמאה אלף יהודים לארץ
			\item ממשלת המנדט תקל על עליית היהודים, ותעשה הכל כדי שאף אחד מהצדדים (היהודים והערבים) לא ישלטו בארץ
		\end{itemize}
		
		הבריטים לא היו מרוצים אז הם הקימו עוד וועדה. 
		\item ואז הייתה ועדת מוריסון־גריידי, ואז הייתה עוד ועדה. 
	\end{itemize}
	
	\subsection{תנועת המרי העברי}
	שבוע הבא
	
	לבדוק: תוכניות הלימודים במדינות אחרות. 
	
	
	
	
	
	
	\ndoc
\end{document}