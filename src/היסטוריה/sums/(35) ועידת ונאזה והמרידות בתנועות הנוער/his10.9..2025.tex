%! ~~~ Packages Setup ~~~ 
\documentclass[]{article}
\usepackage{lipsum}
\usepackage{rotating}


% Math packages
\usepackage[usenames]{color}
\usepackage{forest}
\usepackage{ifxetex,ifluatex,amssymb,amsmath,mathrsfs,amsthm,witharrows,mathtools,mathdots}
\usepackage{amsmath}
\WithArrowsOptions{displaystyle}
\renewcommand{\qedsymbol}{$\blacksquare$} % end proofs with \blacksquare. Overwrites the defualts. 
\usepackage{cancel,bm}
\usepackage[thinc]{esdiff}


% tikz
\usepackage{tikz}
\usetikzlibrary{graphs}
\newcommand\sqw{1}
\newcommand\squ[4][1]{\fill[#4] (#2*\sqw,#3*\sqw) rectangle +(#1*\sqw,#1*\sqw);}


% code 
\usepackage{algorithm2e}
\usepackage{listings}
\usepackage{xcolor}

\definecolor{codegreen}{rgb}{0,0.35,0}
\definecolor{codegray}{rgb}{0.5,0.5,0.5}
\definecolor{codenumber}{rgb}{0.1,0.3,0.5}
\definecolor{codeblue}{rgb}{0,0,0.5}
\definecolor{codered}{rgb}{0.5,0.03,0.02}
\definecolor{codegray}{rgb}{0.96,0.96,0.96}

\lstdefinestyle{pythonstylesheet}{
	language=Java,
	emphstyle=\color{deepred},
	backgroundcolor=\color{codegray},
	keywordstyle=\color{deepblue}\bfseries\itshape,
	numberstyle=\scriptsize\color{codenumber},
	basicstyle=\ttfamily\footnotesize,
	commentstyle=\color{codegreen}\itshape,
	breakatwhitespace=false, 
	breaklines=true, 
	captionpos=b, 
	keepspaces=true, 
	numbers=left, 
	numbersep=5pt, 
	showspaces=false,                
	showstringspaces=false,
	showtabs=false, 
	tabsize=4, 
	morekeywords={as,assert,nonlocal,with,yield,self,True,False,None,AssertionError,ValueError,in,else},              % Add keywords here
	keywordstyle=\color{codeblue},
	emph={var, List, Iterable, Iterator},          % Custom highlighting
	emphstyle=\color{codered},
	stringstyle=\color{codegreen},
	showstringspaces=false,
	abovecaptionskip=0pt,belowcaptionskip =0pt,
	framextopmargin=-\topsep, 
}
\newcommand\pythonstyle{\lstset{pythonstylesheet}}
\newcommand\pyl[1]     {{\lstinline!#1!}}
\lstset{style=pythonstylesheet}

\usepackage[style=1,skipbelow=\topskip,skipabove=\topskip,framemethod=TikZ]{mdframed}
\definecolor{bggray}{rgb}{0.85, 0.85, 0.85}
\mdfsetup{leftmargin=0pt,rightmargin=0pt,innerleftmargin=15pt,backgroundcolor=codegray,middlelinewidth=0.5pt,skipabove=5pt,skipbelow=0pt,middlelinecolor=black,roundcorner=5}
\BeforeBeginEnvironment{lstlisting}{\begin{mdframed}\vspace{-0.4em}}
	\AfterEndEnvironment{lstlisting}{\vspace{-0.8em}\end{mdframed}}


% Design
\usepackage[labelfont=bf]{caption}
\usepackage[margin=0.6in]{geometry}
\usepackage{multicol}
\usepackage[skip=4pt, indent=0pt]{parskip}
\usepackage[normalem]{ulem}
\forestset{default}
\renewcommand\labelitemi{$\bullet$}

\usepackage{graphicx}
\graphicspath{ {./} }

\usepackage[colorlinks]{hyperref}
\definecolor{mgreen}{RGB}{25, 160, 50}
\definecolor{mblue}{RGB}{30, 60, 200}
\usepackage{hyperref}
\hypersetup{
	colorlinks=true,
	citecolor=mgreen,
	linkcolor=black,
	urlcolor=mblue,
	pdftitle={Document by Shahar Perets},
	%	pdfpagemode=FullScreen,
}
\usepackage{yfonts}
\def\gothstart#1{\noindent\smash{\lower3ex\hbox{\llap{\Huge\gothfamily#1}}}
	\parshape=3 3.1em \dimexpr\hsize-3.4em 3.4em \dimexpr\hsize-3.4em 0pt \hsize}
\def\frakstart#1{\noindent\smash{\lower3ex\hbox{\llap{\Huge\frakfamily#1}}}
	\parshape=3 1.5em \dimexpr\hsize-1.5em 2em \dimexpr\hsize-2em 0pt \hsize}



% Hebrew initialzing
\usepackage[bidi=basic]{babel}
\PassOptionsToPackage{no-math}{fontspec}
\babelprovide[main, import, Alph=letters]{hebrew}
\babelprovide[import]{english}
\babelfont[hebrew]{rm}{David CLM}
\babelfont[hebrew]{sf}{David CLM}
%\babelfont[english]{tt}{Monaspace Xenon}
\usepackage[shortlabels]{enumitem}
\newlist{hebenum}{enumerate}{1}

% Language Shortcuts
\newcommand\en[1] {\begin{otherlanguage}{english}#1\end{otherlanguage}}
\newcommand\he[1] {\she#1\sen}
\newcommand\sen   {\begin{otherlanguage}{english}}
	\newcommand\she   {\end{otherlanguage}}
\newcommand\del   {$ \!\! $}

\newcommand\npage {\vfil {\hfil \textbf{\textit{המשך בעמוד הבא}}} \hfil \vfil \pagebreak}
\newcommand\ndoc  {\dotfill \\ \vfil {\begin{center}
			{\textbf{\textit{שחר פרץ, 2025}} \\
				\scriptsize \textit{קומפל ב־}\en{\LaTeX}\,\textit{ ונוצר באמצעות תוכנה חופשית בלבד}}
	\end{center}} \vfil	}

\newcommand{\rn}[1]{
	\textup{\uppercase\expandafter{\romannumeral#1}}
}

\makeatletter
\newcommand{\skipitems}[1]{
	\addtocounter{\@enumctr}{#1}
}
\makeatother

%! ~~~ Math shortcuts ~~~

% Letters shortcuts
\newcommand\N     {\mathbb{N}}
\newcommand\Z     {\mathbb{Z}}
\newcommand\R     {\mathbb{R}}
\newcommand\Q     {\mathbb{Q}}
\newcommand\C     {\mathbb{C}}
\newcommand\One   {\mathit{1}}

\newcommand\ml    {\ell}
\newcommand\mj    {\jmath}
\newcommand\mi    {\imath}

\newcommand\powerset {\mathcal{P}}
\newcommand\ps    {\mathcal{P}}
\newcommand\pc    {\mathcal{P}}
\newcommand\ac    {\mathcal{A}}
\newcommand\bc    {\mathcal{B}}
\newcommand\cc    {\mathcal{C}}
\newcommand\dc    {\mathcal{D}}
\newcommand\ec    {\mathcal{E}}
\newcommand\fc    {\mathcal{F}}
\newcommand\nc    {\mathcal{N}}
\newcommand\vc    {\mathcal{V}} % Vance
\newcommand\sca   {\mathcal{S}} % \sc is already definded
\newcommand\rca   {\mathcal{R}} % \rc is already definded
\newcommand\zc    {\mathcal{Z}}

\newcommand\prm   {\mathrm{p}}
\newcommand\arm   {\mathrm{a}} % x86
\newcommand\brm   {\mathrm{b}}
\newcommand\crm   {\mathrm{c}}
\newcommand\drm   {\mathrm{d}}
\newcommand\erm   {\mathrm{e}}
\newcommand\frm   {\mathrm{f}}
\newcommand\nrm   {\mathrm{n}}
\newcommand\vrm   {\mathrm{v}}
\newcommand\srm   {\mathrm{s}}
\newcommand\rrm   {\mathrm{r}}

\newcommand\Si    {\Sigma}

% Logic & sets shorcuts
\newcommand\siff  {\longleftrightarrow}
\newcommand\ssiff {\leftrightarrow}
\newcommand\so    {\longrightarrow}
\newcommand\sso   {\rightarrow}

\newcommand\epsi  {\epsilon}
\newcommand\vepsi {\varepsilon}
\newcommand\vphi  {\varphi}
\newcommand\Neven {\N_{\mathrm{even}}}
\newcommand\Nodd  {\N_{\mathrm{odd }}}
\newcommand\Zeven {\Z_{\mathrm{even}}}
\newcommand\Zodd  {\Z_{\mathrm{odd }}}
\newcommand\Np    {\N_+}

% Text Shortcuts
\newcommand\open  {\big(}
\newcommand\qopen {\quad\big(}
\newcommand\close {\big)}
\newcommand\also  {\mathrm{, }}
\newcommand\defis {\mathrm{ definitions}}
\newcommand\given {\mathrm{given }}
\newcommand\case  {\mathrm{if }}
\newcommand\syx   {\mathrm{ syntax}}
\newcommand\rle   {\mathrm{ rule}}
\newcommand\other {\mathrm{else}}
\newcommand\set   {\ell et \text{ }}
\newcommand\ans   {\mathscr{A}\!\mathit{nswer}}

% Set theory shortcuts
\newcommand\ra    {\rangle}
\newcommand\la    {\langle}

\newcommand\oto   {\leftarrow}

\newcommand\QED   {\quad\quad\mathscr{Q.E.D.}\;\;\blacksquare}
\newcommand\QEF   {\quad\quad\mathscr{Q.E.F.}}
\newcommand\eQED  {\mathscr{Q.E.D.}\;\;\blacksquare}
\newcommand\eQEF  {\mathscr{Q.E.F.}}
\newcommand\jQED  {\mathscr{Q.E.D.}}

\DeclareMathOperator\dom   {dom}
\DeclareMathOperator\Img   {Im}
\DeclareMathOperator\range {range}

\newcommand\trio  {\triangle}

\newcommand\rc    {\right\rceil}
\newcommand\lc    {\left\lceil}
\newcommand\rf    {\right\rfloor}
\newcommand\lf    {\left\lfloor}
\newcommand\ceil  [1] {\lc #1 \rc}
\newcommand\floor [1] {\lf #1 \rf}

\newcommand\lex   {<_{lex}}

\newcommand\az    {\aleph_0}
\newcommand\uaz   {^{\aleph_0}}
\newcommand\al    {\aleph}
\newcommand\ual   {^\aleph}
\newcommand\taz   {2^{\aleph_0}}
\newcommand\utaz  { ^{\left (2^{\aleph_0} \right )}}
\newcommand\tal   {2^{\aleph}}
\newcommand\utal  { ^{\left (2^{\aleph} \right )}}
\newcommand\ttaz  {2^{\left (2^{\aleph_0}\right )}}

\newcommand\n     {$n$־יה\ }

% Math A&B shortcuts
\newcommand\logn  {\log n}
\newcommand\logx  {\log x}
\newcommand\lnx   {\ln x}
\newcommand\cosx  {\cos x}
\newcommand\sinx  {\sin x}
\newcommand\sint  {\sin \theta}
\newcommand\tanx  {\tan x}
\newcommand\tant  {\tan \theta}
\newcommand\sex   {\sec x}
\newcommand\sect  {\sec^2}
\newcommand\cotx  {\cot x}
\newcommand\cscx  {\csc x}
\newcommand\sinhx {\sinh x}
\newcommand\coshx {\cosh x}
\newcommand\tanhx {\tanh x}

\newcommand\seq   {\overset{!}{=}}
\newcommand\slh   {\overset{LH}{=}}
\newcommand\sle   {\overset{!}{\le}}
\newcommand\sge   {\overset{!}{\ge}}
\newcommand\sll   {\overset{!}{<}}
\newcommand\sgg   {\overset{!}{>}}

\newcommand\h     {\hat}
\newcommand\ve    {\vec}
\newcommand\lv    {\overrightarrow}
\newcommand\ol    {\overline}

\newcommand\mlcm  {\mathrm{lcm}}

\DeclareMathOperator{\sech}   {sech}
\DeclareMathOperator{\csch}   {csch}
\DeclareMathOperator{\arcsec} {arcsec}
\DeclareMathOperator{\arccot} {arcCot}
\DeclareMathOperator{\arccsc} {arcCsc}
\DeclareMathOperator{\arccosh}{arccosh}
\DeclareMathOperator{\arcsinh}{arcsinh}
\DeclareMathOperator{\arctanh}{arctanh}
\DeclareMathOperator{\arcsech}{arcsech}
\DeclareMathOperator{\arccsch}{arccsch}
\DeclareMathOperator{\arccoth}{arccoth}
\DeclareMathOperator{\atant}  {atan2} 
\DeclareMathOperator{\Sp}     {span} 
\DeclareMathOperator{\sgn}    {sgn} 
\DeclareMathOperator{\row}    {Row} 
\DeclareMathOperator{\adj}    {adj} 
\DeclareMathOperator{\rk}     {rank} 
\DeclareMathOperator{\col}    {Col} 
\DeclareMathOperator{\tr}     {tr}

\newcommand\dx    {\,\mathrm{d}x}
\newcommand\dt    {\,\mathrm{d}t}
\newcommand\dtt   {\,\mathrm{d}\theta}
\newcommand\du    {\,\mathrm{d}u}
\newcommand\dv    {\,\mathrm{d}v}
\newcommand\df    {\mathrm{d}f}
\newcommand\dfdx  {\diff{f}{x}}
\newcommand\dit   {\limhz \frac{f(x + h) - f(x)}{h}}

\newcommand\nt[1] {\frac{#1}{#1}}

\newcommand\limz  {\lim_{x \to 0}}
\newcommand\limxz {\lim_{x \to x_0}}
\newcommand\limi  {\lim_{x \to \infty}}
\newcommand\limh  {\lim_{x \to 0}}
\newcommand\limni {\lim_{x \to - \infty}}
\newcommand\limpmi{\lim_{x \to \pm \infty}}

\newcommand\ta    {\theta}
\newcommand\ap    {\alpha}

\renewcommand\inf {\infty}
\newcommand  \ninf{-\inf}

% Combinatorics shortcuts
\newcommand\sumnk     {\sum_{k = 0}^{n}}
\newcommand\sumni     {\sum_{i = 0}^{n}}
\newcommand\sumnko    {\sum_{k = 1}^{n}}
\newcommand\sumnio    {\sum_{i = 1}^{n}}
\newcommand\sumai     {\sum_{i = 1}^{n} A_i}
\newcommand\nsum[2]   {\reflectbox{\displaystyle\sum_{\reflectbox{\scriptsize$#1$}}^{\reflectbox{\scriptsize$#2$}}}}

\newcommand\bink      {\binom{n}{k}}
\newcommand\setn      {\{a_i\}^{2n}_{i = 1}}
\newcommand\setc[1]   {\{a_i\}^{#1}_{i = 1}}

\newcommand\cupain    {\bigcup_{i = 1}^{n} A_i}
\newcommand\cupai[1]  {\bigcup_{i = 1}^{#1} A_i}
\newcommand\cupiiai   {\bigcup_{i \in I} A_i}
\newcommand\capain    {\bigcap_{i = 1}^{n} A_i}
\newcommand\capai[1]  {\bigcap_{i = 1}^{#1} A_i}
\newcommand\capiiai   {\bigcap_{i \in I} A_i}

\newcommand\xot       {x_{1, 2}}
\newcommand\ano       {a_{n - 1}}
\newcommand\ant       {a_{n - 2}}

% Linear Algebra
\DeclareMathOperator{\chr}     {char}
\DeclareMathOperator{\diag}    {diag}
\DeclareMathOperator{\Hom}     {Hom}
\DeclareMathOperator{\Sym}     {Sym}
\DeclareMathOperator{\Asym}    {ASym}

\newcommand\lra       {\leftrightarrow}
\newcommand\chrf      {\chr(\F)}
\newcommand\F         {\mathbb{F}}
\newcommand\co        {\colon}
\newcommand\tmat[2]   {\cl{\begin{matrix}
			#1
		\end{matrix}\, \middle\vert\, \begin{matrix}
			#2
\end{matrix}}}

\makeatletter
\newcommand\rrr[1]    {\xxrightarrow{1}{#1}}
\newcommand\rrt[2]    {\xxrightarrow{1}[#2]{#1}}
\newcommand\mat[2]    {M_{#1\times#2}}
\newcommand\gmat      {\mat{m}{n}(\F)}
\newcommand\tomat     {\, \dequad \longrightarrow}
\newcommand\pms[1]    {\begin{pmatrix}
		#1
\end{pmatrix}}

\newcommand\norm[1]   {\left \vert \left \vert #1 \right \vert \right \vert}
\newcommand\snorm     {\left \vert \left \vert \cdot \right \vert \right \vert}
\newcommand\smut      {\left \la \cdot \mid \cdot \right \ra}
\newcommand\mut[2]    {\left \la #1 \,\middle\vert\, #2 \right \ra}

% someone's code from the internet: https://tex.stackexchange.com/questions/27545/custom-length-arrows-text-over-and-under
\makeatletter
\newlength\min@xx
\newcommand*\xxrightarrow[1]{\begingroup
	\settowidth\min@xx{$\m@th\scriptstyle#1$}
	\@xxrightarrow}
\newcommand*\@xxrightarrow[2][]{
	\sbox8{$\m@th\scriptstyle#1$}  % subscript
	\ifdim\wd8>\min@xx \min@xx=\wd8 \fi
	\sbox8{$\m@th\scriptstyle#2$} % superscript
	\ifdim\wd8>\min@xx \min@xx=\wd8 \fi
	\xrightarrow[{\mathmakebox[\min@xx]{\scriptstyle#1}}]
	{\mathmakebox[\min@xx]{\scriptstyle#2}}
	\endgroup}
\makeatother


% Greek Letters
\newcommand\ag        {\alpha}
\newcommand\bg        {\beta}
\newcommand\cg        {\gamma}
\newcommand\dg        {\delta}
\newcommand\eg        {\epsi}
\newcommand\zg        {\zeta}
\newcommand\hg        {\eta}
\newcommand\tg        {\theta}
\newcommand\ig        {\iota}
\newcommand\kg        {\keppa}
\renewcommand\lg      {\lambda}
\newcommand\og        {\omicron}
\newcommand\rg        {\rho}
\newcommand\sg        {\sigma}
\newcommand\yg        {\usilon}
\newcommand\wg        {\omega}

\newcommand\Ag        {\Alpha}
\newcommand\Bg        {\Beta}
\newcommand\Cg        {\Gamma}
\newcommand\Dg        {\Delta}
\newcommand\Eg        {\Epsi}
\newcommand\Zg        {\Zeta}
\newcommand\Hg        {\Eta}
\newcommand\Tg        {\Theta}
\newcommand\Ig        {\Iota}
\newcommand\Kg        {\Keppa}
\newcommand\Lg        {\Lambda}
\newcommand\Og        {\Omicron}
\newcommand\Rg        {\Rho}
\newcommand\Sg        {\Sigma}
\newcommand\Yg        {\Usilon}
\newcommand\Wg        {\Omega}

% Other shortcuts
\newcommand\tl    {\tilde}
\newcommand\op    {^{-1}}

\newcommand\sof[1]    {\left | #1 \right |}
\newcommand\cl [1]    {\left ( #1 \right )}
\newcommand\csb[1]    {\left [ #1 \right ]}
\newcommand\ccb[1]    {\left \{ #1 \right \}}

\newcommand\bs        {\blacksquare}
\newcommand\dequad    {\!\!\!\!\!\!}
\newcommand\dequadd   {\dequad\duquad}

\renewcommand\phi     {\varphi}

\newtheorem{Theorem}{משפט}
\theoremstyle{definition}
\newtheorem{definition}{הגדרה}
\newtheorem{Lemma}{למה}
\newtheorem{Remark}{הערה}
\newtheorem{Notion}{סימון}


\newcommand\theo  [1] {\begin{Theorem}#1\end{Theorem}}
\newcommand\defi  [1] {\begin{definition}#1\end{definition}}
\newcommand\rmark [1] {\begin{Remark}#1\end{Remark}}
\newcommand\lem   [1] {\begin{Lemma}#1\end{Lemma}}
\newcommand\noti  [1] {\begin{Notion}#1\end{Notion}}

% DS
\newcommand\limsi     {\limsup_{n \to \inf}}
\newcommand\limfi     {\liminf_{n \to \inf}}

%! ~~~ Document ~~~

\author{שחר פרץ}
\title{\textit{היסטוריה 35} – המשך שיחה על הפתרון הסופי}
\begin{document}
	\maketitle
	``הגענו למנוחה ולנחלה''
	
	היום נסיים את ועידת ואנזה ואולי נתחיל מרד בגטאות. בשבוע הבא, אין שיעורים בכלל – בשלישי נוסעים לאנשהו וברביעי יש הרצאה על השיעור. שבוע לאחר מכן מתחיל ראש השנה. לאחר מכן סוכות, ואז איפשהו אחרי סוכות יש שיעור אחד שלם. לאחר מכן אמורים להיות עוד שני שיעורים, ובסופם מבחן. כלומר יש שלושה שיעורים עד המבחן. 
	
	\section{הלמידה מהשלבים השונים בתהליך ההשמדה}
	צעדות המוות עקרונית נכללות בפתרון הסופי אבל הן די יוצאות דופן מסיבות שנציין בהמשך. 
	\begin{itemize}
		\item \textbf{תיעוש וייעול של הרצח}
		\item \textbf{ריחוק של רוצח מהנרצח: }התהליך הגיע לשיא באשוויץ (בירקנאו). הרוצחים לא רואים את הרוצח בלבד. יש לו 4 מתחמי קרמטוריום, ``יש שם תהליך שבו בן אדם נכנס חיי, ויוצא אפר''. לרוצח יש כיסא שעליו הוא יושב, פותח חלון, זורק פחית ומסיים. הריחוק הזה איפשר את הדה־הומניזציה של היהודים. ``מגייסים את כל הכוחות האינטלקטואלים כדי לגרום לזה לקרות'' – הכל היה תוחכם מבחינה טכנולוגית. 
		\item \textbf{הרחבה גיאוגרפית: }בהתחלה התחיל משטחי הכיבוש שנכבשו במבצע ברברוסה, באזורים המערביים, ואז התקדם מערבה (חלמו נמצא בלודז', במערב פולין). לאושוויץ מגיעים מכל מדינה באירופה, כולל יוון. 
	\end{itemize}
	
	ליהודי יוון (שידעו לדינו וקצת יוונית) התקשו מאוד להבין את הפקודות הגמרניות, ולכן הם סבלו קשות מהתעללות/רצח – הם לא הבינו את הפקודות הגרמניות. 
	
	\subsection{ועידת ואנזה}
	הועידה התקיימה ב־20 בינואר 1942 בברלין, כ־7 חודשים \textit{אחרי} שהחל הפתרון הסופי, וכמליון יהודים כבר נרצחו בשטחי ברימה''ש. הועידה \textit{לא} נועדה להחליט על הפתרון הסופי, אלא ליעל, לארגן, לתכנן ולהחיש את הפתרון הסופי. \textit{אין החלטה רשמית מתועדת על הפתרון הסופי}, וכנראה הדברים נאמרו באופן לא ישיר. היטלר לא נכח בוועידה כדי שלא יווצר קשר בינו לבין תהליך ההשמדה. 
	
	הזמינו אנשים לוילה בה אכלו ושתו. בוועידה נכחו ראש הגאסטפו, ראש שירות הבטחון, הימלר (ראש ה־SS) ואייכמן. אייכמן היה האדם היחיד שאיננו היה משכיל. גם ראשי המשטרה מחבלי הכביוש השונים במזרח, וראשי מנהל נוספים (\textit{מבצעים}). זוהי וועדת ביצוע, שרוצים להסיק ממנה דברים ברמה הביצועית. מטרותיה: 
	\begin{itemize}
		\item לתת אישור רשמי להשמדת היהודים, דהיינו לרתום את כל הרשויות והמערכת האזרחית לתיאום ולמימוש המטרה. 
		\item יידע את אנשי המפתח על המבצע המיוחד שבראשו יעמוד היידריך היינרג, ולהבהיר להם שהם נדרשים להרתם ולפעול ע''פ ההנחיות. נוסף על כך קבעו מי הפוסק במקרה של חילוקי דעות בין הרשויות. 
		\item להדגיש שהפתרון הסופי מתייחס לכל יהודי אירופה ללא הגבלות גיאוגרפיות (שטחים כבודים, מדינות גרורות, ניטרליות, בעלי ברית בעתיד ובהווה וכו'). 
	\end{itemize}
	
	\subsubsection{פרוטוקולי ועידת ואנזה}
	מה המשמעות והחשיבות של הנתונים בטבלה בעמודים 236-238? 
	
	תשובה שלי – מטרת הנתונים היא לארגן את השמדת היהודים, כך שכל מנהל איזורי יוכל לדעת כמה יהודים צריכים לצאת משטחו ולהגיע אל מחנות ההשמדה. חשיבותם היא שקיום הטבלה הופך את התהליך למסודר ורשמי, תוך הגדרת מטרות מספריות ברורות. הן מבטאות את הטוטאליות של הפתרון הסופי, בכך שהיעד להשמדת יהודים מכל מדינה, היא כמות היהודים באותה המדינה. יש תיעוד אף של המדינות עם מיעוט היהודים, כגון אלבניה וכו'. איפה שיש מידע מדויק (אוקראינה לדוגמה), סופרים במדויק כדי לדעת כמה להשמיד. 
	
	דוגמה לנתונים: הרייך הישן, 131K, ברה''מ 5000K, אנגליה 330K וכו', אגטוניה חופשית מיהודים, אלבניה 200, צרפת בלתי כבוש 700K, צרפת כבוש 165K, נרווגיה 1.3K, אוקראינה 2,994,685 וכו'. בלראוס 4K, שוויץ 8K וכו'. הסכום הכולל הוא 11M, כולם מיועדים להשמדה. כנראה המספר של חמשיה מליון בברית המועצות לא היה מדויק, כי הייתה תזוזה בין ברימה''ש לפולין וכנראה ספרו כפלים מאלו שברחו מזרחה. 
	
	חזרה למה שאומרים בכיתה. ממספרים כמו ``אסטוניה חופשית מיהודים, אלבניה 200'' – עד היהודי האחרון. כולם מיועדים להשמדה. התפישה היא שילכו עד ההר האקראי באלבניה ובלבד שישמידו אותם. באוקראינה נבחין שהמספרים מדוייקים להפליא – זה לא בערך, סופרים עד היהודי האחרון, וייוודאו שכולם ילכו להשמדה. היו מקומות שבהם היה אפשר לדעת יותר בקלות כמה יהודים יש, באמצעות פנקסי קהילות – בכל קהילה היה פנקס שתיאר אנשים שנולדו ומתו. הדבר המיוחד בשוודיה, שוויץ וכו' – אלו מדינות ניטרליות, והם סופרים יהודים גם שם. הם ספרו יהודים גם באנגליה, מקום שהם בכלל לא כבשו, וכנ''ל על השטחים הלא כבושים בצרפת (זה היה משטר וישי שהיה משטר בובות, הוא היה כבוש בפועל). 
	
	צרפת לאורך ההיסטוריה שלטה על מדינות כמו טוניס ואלז'יר, אתיופה ולוב ע''י איטליה, סוריה ולבנון גם צרפת, וכו'. הם ספרו גם את יהודי צפון אפריקה. מדובר על משהו טוטאלי. ביטויי הטוטאליות הם: 
	\begin{itemize}
		\item לצד 5 מיליון יהודים, יש 200 מאלבניה. רצו להשמיד את כולם עד האחרון
		\item נכללו יהודים ממדינות ניטרליות ומדינות שתרם נכבשו. 
		\item הטבלה כללה את יהודי צפון אפריקה בקולוניות. 
	\end{itemize}
	
	לא הצליחו להגיע להסכמות בדבר בני תערובת. ההחלטה הזו לא הוחלטה, ומה שקרה הוא שגורלם של בני התערובת נקבע ע''י המפקדים המקומיים באותו האיזור. 
	
	\subsubsection{סיכום}
	הועדה הייתה מנהלית ומתאמת, שמנסה להבין מי ייפסור במקרה של חילוקי דעות (לאייכמן היה תפקיד מרכזי בכך, למרות שטען שהיה רק פקיד שקיבל פקודות). היא נחשבת שלב בפתרון הסופי, והיא מעידה על כך ש־: 
	\begin{itemize}
		\item זוה רצח עולמי טוטאלי ולא מקומי
		\item כל המערכות רתומות לרצח
		\item הרצח יהיה מאורגן ע''י רשות מרכזית אחת
		\item ההשדמה תהיה שיטתית ותעשייתית. 
	\end{itemize}
	
	פרוטוקול הוועידה הוא המסמך הכי קרוב למסמך רשמי המאשר את השמדת העם היהודי. 
	
	\section{המרידות בתנועות הנוער}
	\subsection{אבא קובנר ודבריו}
	מהניג התנועה הציונית, אבא קובנר. וילנה, אמרה שמשונים אלף היהודים ביורשלים־דלתיא (וינה) שרדו רק עשרים אלף. ``איה מאות הגברים שגורשו?'' ``מי שהוצא משער הגטו – לא חזר עוד. כל דרכי הגאסטפו מובילות לפונאר, ופונאר זה מוות!''. פונאר אילו בורות ירי ענקיים ליד וינה, שנרצחו בהם עשרות אלפי מיהודי וינה. ``הפרו את האשליה, המיואשים: ילדכם, בעליכם, נשיכם – אינם עוד! אתכולם רצחו שם. היטלר חושב להשמיד את כל יהודי אירופה. יהודי ליטא נעמדו בתור הראשונים. אל נלך כצאן לטבח``. 
	
	באמרתו, ``אחים, טוב לפול כלוחמים בני־חורין מלחיות בחסד מרצחים'', מופיעה בחירה – בחירה איך למות. בציטוט הזה מובא כחצי שנה אחרי מבצע ברברוסה. 
	
	הם נלחמו במדירות, על החופש למות בכבוד. המרידות תוכננו בצורה כזו שאין כוונה להמשיך לחיות אח''כ. ברור היה לכולם שהם נלחמים בכוח גדול מהם וחזק מהם בסדר גודל. 
	
	\subsection{על המורדים}
	מאפייני מנהייג תנועות הנועד והמורדים: 
	\begin{itemize}
		\item צעירים בגיל 17-25 בתחילת המלחמה
		\item רווקים וללא ילדים
		\item בעלי השקפת עולם ערכית מוצקה
		\item הבינו את העולם של האידיאולוגיות הגדולות בו הם חיו
		\item חשו מחויבים באופן מוחלט לתנועה
	\end{itemize}
	
	לעומתם, היו היודנראטים, שהחזיקו בתקופה הישנה – ננסה לשרוד כמה שיותר, עד שבעלות הברית ישחררו. חברי תנועות הנוער (הציוניות) לא האמינו בכך, והיו בטוחים שהנאצים ירצחו אותם עוד לפני שבעלות הברית יגיעו אליהם. 
	\subsection{מטרות הלחימה בגטאות}
	\begin{itemize}
		\item הרצון לנקום בגרמנים. 
		\item למות בכבוד ולא ללכת ``כצאן לטבח''. 
		\item ``למען שלוש שורות בהיסטוריה'' – תפישה שהייתה לכל חברי תנועות הנוער, כדי שהדורות הבאים יידעו שהייתה התנגדות יהודית לנאצים, והיהודים לחמו ומתו כבני חורין. 
		\item רצון לשמש דוגמה ומופת לנוער בא''י. 
	\end{itemize}
	
	אין פה נסיון להנצל. עם זאת, הם לא התאבדו, הם היו צריכים לבחור בין למות בדרך אחת לבין למות בדרך אחרת. בסופו של דבר בסיום המרידות, ניצלו מעטים דרך תעלות הביוב וכו'. אבל זו לא מטרה מוגדרת. 
	
	נפגש בפעם הבאה יום לפני יום כיפור. שנה טובה. 
	
	
	
	
	\ndoc
\end{document}