%! ~~~ Packages Setup ~~~ 
\documentclass[]{article}
\usepackage{lipsum}
\usepackage{rotating}


% Math packages
\usepackage[usenames]{color}
\usepackage{forest}
\usepackage{ifxetex,ifluatex,amssymb,amsmath,mathrsfs,amsthm,witharrows,mathtools,mathdots}
\usepackage{amsmath}
\WithArrowsOptions{displaystyle}
\renewcommand{\qedsymbol}{$\blacksquare$} % end proofs with \blacksquare. Overwrites the defualts. 
\usepackage{cancel,bm}
\usepackage[thinc]{esdiff}


% tikz
\usepackage{tikz}
\usetikzlibrary{graphs}
\newcommand\sqw{1}
\newcommand\squ[4][1]{\fill[#4] (#2*\sqw,#3*\sqw) rectangle +(#1*\sqw,#1*\sqw);}

% Deisgn
\usepackage[labelfont=bf]{caption}
\usepackage[margin=0.6in]{geometry}
\usepackage{multicol}
\usepackage[skip=4pt, indent=0pt]{parskip}
\usepackage[normalem]{ulem}
\forestset{default}
\renewcommand\labelitemi{$\bullet$}
\usepackage{graphicx}
\graphicspath{ {./} }

\usepackage[colorlinks]{hyperref}
\definecolor{mgreen}{RGB}{25, 160, 50}
\definecolor{mblue}{RGB}{30, 60, 200}
\usepackage{hyperref}
\hypersetup{
    colorlinks=true,
    citecolor=mgreen,
    linkcolor=black,
    urlcolor=mblue,
    pdftitle={Document by Shahar Perets},
    %	pdfpagemode=FullScreen,
}


% Hebrew initialzing
\usepackage[bidi=basic]{babel}
\PassOptionsToPackage{no-math}{fontspec}
\babelprovide[main, import, Alph=letters]{hebrew}
\babelprovide[import]{english}
\babelfont[hebrew]{rm}{David CLM}
\babelfont[hebrew]{sf}{David CLM}
%\babelfont[english]{tt}{Monaspace Xenon}
\usepackage[shortlabels]{enumitem}
\newlist{hebenum}{enumerate}{1}

% Language Shortcuts
\newcommand\en[1] {\begin{otherlanguage}{english}#1\end{otherlanguage}}
\newcommand\he[1] {\she#1\sen}
\newcommand\sen   {\begin{otherlanguage}{english}}
    \newcommand\she   {\end{otherlanguage}}
\newcommand\del   {$ \!\! $}

\newcommand\npage {\vfil {\hfil \textbf{\textit{המשך בעמוד הבא}}} \hfil \vfil \pagebreak}
\newcommand\ndoc  {\dotfill \\ \vfil {\begin{center}
            {\textbf{\textit{שחר פרץ, 2025}} \\
                \scriptsize \textit{קומפל ב־}\en{\LaTeX}\,\textit{ ונוצר באמצעות תוכנה חופשית בלבד}}
    \end{center}} \vfil	}

\newcommand{\rn}[1]{
    \textup{\uppercase\expandafter{\romannumeral#1}}
}

\makeatletter
\newcommand{\skipitems}[1]{
    \addtocounter{\@enumctr}{#1}
}
\makeatother

%! ~~~ Math shortcuts ~~~

%! ~~~ Document ~~~

\author{שחר פרץ}
\title{\textit{היסטוריה 30}}
\begin{document}
    \maketitle
    \section{מנהלות}
    שרית שמחה לצאת לחופש. תזכורת להגיש עד ה־15 ביוני את החלק של ה־30\%. היא ננעלת באותו היום ולא תפתח עוד פעם – לא יתקבלו איחורים. תקבלו עבודה לקיץ, ויש להגישה במבחן הראשון (ולא בשיעור הראשון) בניילונית. היא מקנה בונוס לציון המבחן הראשון. היא איננה חובה להגשה. 
    
    \section{צעדים שנעשו נגד היהודים לפני הכניסה לגטאות}
    \subsection{חזרה על שיעור קודם}
    כל הדברים שנעשו בגרמניה עד 39 הגיעו בבום אחד לפולין, ואף מעבר לכך (כלומר, היחס ליהודי פולין היה אף יותר גרוע מזה של יהודי גרמניה, או דנמרק לצורך הנקודה). 
    \begin{enumerate}
        \item \textbf{פיגעה במוסדות דת ותרבות יהודיים: }זקני היהודים נגחו בהתרסה ברחובות, ספרי קודש חוללו, בתי כנ סובתי מדרש נשרפו, לעיתים עם המתפללים בפנים. כמו כן נאסרה שחיטה כשרה, לימודי תורה, תפילה בציבור, וחובת עבודה בימי שבת וחג. תשומת הלב העיקרית הפונתה להשפלתם את היהודים
        \item \textbf{צעדים להפרדה: }טלאי צהוב, איסור כניסה וכוק. 
        \item \textbf{הגבלת חופש התנועה}
        \item \textbf{הרס קיו םכלכלי: }אריזציה של כל בית מלאכה גדול (עד לכדי חנויות מכולת קטנות), הוחרמו דירות יהוהידם אמידים, נאסר להחזיק כסף בסכומים גדולים וכו'. 
        \item \textbf{חטיפות יהודים לעבודות כפייה: }לא קרה בגרמניה. חדש לפולין. חטפו מהרוב לעבודות כפייה (לפני הקמת הגטאות). דברים כמו סחיבת משאיות, עבודות שירות במחנות צבאיים, גרייה, בניית תשתית, ועוד. חלק מהעבודות מועילות, וחלקן חסרי פואנטה ונוצרו אך ורק בשביל להשפיל ולנצל את היהודים – לדוגמה ללכת ולהחזיר אבנים ממקום למקום. מיום הקמת הגנרל גוברנמן חויבו יהודים בגיל עבודה לצאת ולעבוד. 
        \item ישנה תרומה מפורסמת של גרמנים הבוזזים יהודים זקנים פולניים. במקרים אחרים הכריחו אותם לרקוד ברחוב, ועמדו וצחקו. המון דברים שמטרתם הייתה השפלה בלבד. 
    \end{enumerate}
    
    ההפרדה הוא לא בידוד בין היהודים לגרמנים (הגרמנים עוד לא גרים בפולין), אלא גם מול האוכלוסיה המקומית. 
    
    \subsection{גטאות}
    זוהי לא מילה שאנשים נבהלו ממנה לפני השואה. זו לא הייתה מילה עם קונוטציה של צפיפות, רעב, מחלות ומוות. לפני כן, זו פשוט הייתה שכונה יהודית סגורה שבה חיו יהודים. היהדוים באירופה חיו באיזור סגורים. אפילו במדינות ערביות – שם קראו לזה ``מלאח'' (דומה מאוד לגטו), כי ע''פ הדת המוסלמי יש מעמד מיוחד ליהודים שנועד להשפלי אותם. 
    
    בערים באירופה, היהודים לא יכלו לגור מחוץ לגטו, אבל זה היה גם אינטרס של היהודים – זה שמר עליהם. זו הייתה שכונה סגורה, עם שער, שבה הם יכלו לשמור על המנהגים שלהם, ולא להתבולל. בחלק גטאות באירופה, בחגים וימי ארשון היו עיתות שבהם לא הייתה אופציה ליציאה וכניסה מהגטו. לא היה שם רעב, זה לא גרם למגיפות, ולא הייתה לכך גונוטציה שלילית. כמו הרובע היהודי והמוסלמי בירושלים. הגטו הראשון היה הגטו נובו, בונציה. התחיל בימי הביניים. 
    
    \subsubsection{הגטאות הנאציים}
    בשונה מבעבר, בגטו הנאצים שיסודותיו הוקמו בעקבות אגרת הבזק של היידריך, התפיסה השתנתה – רוצים לבודד אותם, בשביל ``משהו'' (לא ידעו עוד מה). 
    
    \begin{enumerate}
        \item הגטו הראשון – פיוטקרוב, הוקם ב־39. 
        \item הגטאות ברוסיה הוקמו בידיעה שהולכים להשמדה, בניגוד לגטאות ברוסיה. 
        \item גטו לודג' – השני מבינהם, מאוד מיוחד ועליו נדבר בהמשך
        \item גטו ורשה – הגדול והצפוץ מבינהם. בסוף היו שם 60K היהודים והוא התחיל עם 400K שנדחפו לשם בהדרגה (כשליש מאוכלוסיית ורשה). הקמתו נמשכה כשנה. תמדי היה צפוף כי גם כאשר לא היו שם הרבה אנשים צמצמו את השטח. 
    \end{enumerate}
    
    גטאות הוקמו בעיקר המרכזית/הגדולה ביותר באיזור, ועוד בהתחלה הגיעו לשם יהודים מכל האיזור. לגטאות מסויימות כמו גטו ורשה הגיעו בשלבים מסויימים אף ממידנות אחרות כמו גרמניה. 
    
    המעבר לגטאות היה מהיר וחד. מותר להם לקחת כמות מסויימת של דברים, ורוב רכושם נשאר בבתים. הם לא ידעו לאן הם הוכלים והגטאות האירופיים היו הדבר שחשבו עליהם. נאלצו למסור לגרמנים תגישטים, כסף וזהב. גם בתוך הגטו וגם מחוצה לו נגרשו לשאת את הטלאי הסימן (כי בפנים התובבו אנשים שלא היו יהודים). הרכבת החשמלית לדוגמה, עוברת בתוך הגטו (על אף שאין תחנה שם). בפולין לא היה טלאי צהוב כמעט, אלא סרט לבן עם מגן דוד כחול. 
    
    \subsubsection{הקמת הגטאות}
    לא קורה ביום אחד. לודג' היה בין הראשונים, ובגלל מיקומו (צמוד לגנרל גוברנמן, אבל בתוך גרמניה והאוכלוסיה סביב גרמנית). לכן היה בין הסגורים והקשים ביותר. הגטאות הוקמו בשכונות הכי עניות בעיר – ליד איפה שעוברים פסי רכבת. מקומות רועשים ובהם התנאים פחות טובים, האוויר פחות נקי וכו'. כמובן שלנושא פסי הרכבת היו מטרות נוספות. 
    
    ישנה תרומה מפורסמת של גרמנים הבוזזים יהודים זקנים פולניים. במקרים אחרים הכריחו אותם לרקוד ברחוב, ועמדו וצחקו. המון דברים שמטרתם הייתה השפלה בלבד. 
    
    \textit{הבהרה: }כל הגטאות שונים. ננסה לדבר על הדברים שדומים בינהם. אך כמות המגיפות, אופן הניהול, כמות הצפיפות, רמת הסגירות וכו' – מאוד שונים בין גטאות. 
    
    השטח היה מאוד מצומצם. בגטו וילנא היו בין 5-8 נפשות בחדר אחד. בגטו ורשה הייתה הצפיפות הגדולה ביותר, בין 15-30 אנשים בחדר. לכל דירה (קבוצה של חדרים) היה מקלחת אחת – בערך 30-40 אנשים על מקלחת
    
    בגטו אין שוויון כלכלי־חברתי. לא כולם עניים. לא כולם רעבים. יש שם אנשים מאוד עניים ויש בה אנשים עשירים. גטו זו שכונה. יש שם נשים, וגברים, וילדים, ופושעים, ומנהלים והכל. וחיים שם כמשפחות. בניגוד למחנות (לא בהכרח מחנות השמדה), גטו היה שכונה לכל דבר. במחנות כמעט ואין ילדים, יש הפרדה מגדרית, לא חיים כמשפחה, ומטרתם אחרת. 
    
    אומנם רוב האוכלוסייה ענייה ורעבה, אך לא כולם. עדה מורשה במסע הקודם לפולין, ספירה שבגלל שחיו עוד קודם לכן בורשה והנאצים היו צריכים את המפעל של המשפחה העמידה שלהם, היא לא כל־כך סבלה מרעב. 
    
    \subsection{מטרות הקמת הגטאות}
    כאשר נדבר על הסיבות באופן כללי – נדבר על הסיבות האמיתיות. לא על הסיבות שהנאצים פרסמו תאוכלוסייה וליהודים. 
    
    הסיבות המוצהרות: 
    \begin{itemize}
        \item נועד לשמור על היהודים בגלל האנטישמיות והכל
        \item למנוע שמועות שליליות פוליטיות, חתרניות ותבוסתניות מצד היהודים
        \item למנוע התפשטות של מגיפות מדבקות שמקורן ביהודים, ובכך לשמור על מצב סניטרי תקין. 
        \item למנוע שוק שחור (שהיהודים הואשמו בניהולו)
    \end{itemize}
    
    המטרות המוצהרות נועדו גם להרחיק את הפולנים לגטאות. רצו שהפולנים לא ירצו להיות קרובים ולעשות עסקים עם היהודים (אם כי התקיימו כאלו במקומות מסויימים). להסברים האלו לא היה בסיס. להלן הסיבות האמיתיות: 
    \begin{itemize}
        \item \textbf{ריכוז ובידוד: }הפחדת האוכלוסיה הפולנית במאצעות הפצת שמועות בדבר מחלות וכו'. 
        \item \textbf{ניצול כלכלי:} כאשר נכנסים לגטו הנאצים לוקחים את הרכוש שלהם. חלק מהרכוש נשלח לגמרניה, בחלק השתמשו. הניצול הכלכלי בא גם בנושא שם עובודת כפייה, כי הפכו לכוח עבודה זול/חינם. בתמורה היהודים קיבלו עוד מנה של אוכל. עבודות כפייה היו משהו שבשלב מסויים יהודים ``רצו'' לצאת אליו, כי מקבלים אוכל (שלא היה בגטו). חלקם הצליחו להבריח את האוכל לתוך הגטו. 
        \item \textbf{אמצעי הכחדה עקיף: }תמותה כתוצאה מרעב, צפיפות, מחסור ומחלות. הנאצים דאגו לפיפות בטו, הן לשם נוחות לתפוס את היהודים אבל גם לשם תמותה. הנאצים קראו לזה מוות ``טבעי''. אומנם הרעבה זה מוות מאוד יחסית ישיר ו''עקיף'' זה קצת ``מכבשת מילים'', אך אין זה מחנה השמדה שיטתי = מטרת ההדעה היא לא לרצוח אותם. בין 41-42 מתו 112K יהודים בלודג' ובורשה. ביולי אוגוסט חום אימים, הדירות צפופות ולא מאווררות. אין הגיינה ותרופות (רופאים לא חסרים). 
        
        עד שלב מסויים אפשרו לארגונים להיכנס לגטו ולסייע, מה שנגמר אחרי שארה''ב נכנסה למלחהמה. 
        \item \textbf{לשבור את רוחם של היהודים: }כשנדבר על התנגדות, זה חלק מהעניין. 
    \end{itemize}
    
    
    בשלב הזה באירופה כבר לא היו גטאות. 
    
    \subsubsection{ניהול הגטו}
    הגטאות שונים אחד מהשני. אבל באופן כללי: 
    \begin{itemize}
        \item הקפת הגטו (חומה/טיל). יש מקרי קצה זניחים של גטאות ``פתוחים'', כלומר אנשים יכלו לצאת בחוץ ולחזור (שהנאצים אישרו). 
        \item חובת נשיאת סימן זיהוי ותעודות מזהות. 
        \item משטרה גרמנית השומרת מחוץ לגטו (כדי שהאינטראקציה עם היהודים תהיה מינימלית). השומרים פולניים. 
        \item יודנראטים – אמצעי פיקוח מאוד משמעותי שאף הקים את המשטרה היהודית (תזכורת: זו שכונה לכל דבר וצריך למנוע שם פשע). תפקיד היוגנראטים היה לשרת את הנאצים ולסייע להם בפיקוח הגטו. נדבר עליהם בנפרד ובגדול הם קיבלו יחס יותר טוב (מה שלא מנע לרצוח אותם אחכ). 
        \item גאסטפו (המשטרה החשאית הגרמנית). הפעילה הגטאות רשתות של סוכנים ומרגלים (שלא היו גרמנים). 
    \end{itemize}
    
    שוב, יש הבדלים בין הגטאות. נבתונן בחומות של הגטאות – החומה הגבוהה של קרקוב זה לא החומות עץ של הגטאות האחרים. לדוגמה, קרקוב היה מוקף חומת אבן אולם והיה פחות סגור ומנותק מגטו ורשה, והציאה ממנו לצורכי עבודה הייתה קלה יותר. גטו לודג' היה שונה משאר הגטאות והוא היה מבודד לחלוטין, וסגור יותר. הוא נשמר מבחוץ ע''י שוטרים גרמניים ובתוכו המשטרה היהודית פיקחה על הסדר. בגטאות מסויימים היו מצליחים להבריח אוכל לתוך הגטו. לא רק פוחי אדמה, אלא כמויות של אוכל. בגטאות כמו לודג' לא היה אפשר להכניס אוכל בכלל ולכן היה הרעב ביותר. 
    
    היו גם גטאות קטנים יותר, עם מידת ניתוק קטנה יותר. 
    
    באגרת הבזק של היידריך, רא משטרת ה־SS, נאמר להקים מועצה שתוקם מהזקנים והרבנים. הם מנהלים את הגטו והם אנשי הקשר עם הנמאצים. הם גם אחראים על מה שקורה בתוך הגטו. 
    למה צריך אותם? 
    \begin{itemize}
        \item כדי שיספגו את האשמה מצד היהודים. לפהנות את הזעם והמרירות כנגדם. 
        \item חסיכת כוח אםד גרמני. 
        \item השוואה והטעיה, כאילו חייהם של היהודים מתנהלים ע''י היודנראטים
    \end{itemize}
    
    היודנראט כיחיד היה מוכר ומוסמך, בעל אחריות אישית לביצוע המדיניות הנאצים, כלומר אם היא לא בוצעה הוא לא ימשיך בתפקידו וכנראה ישלח למחנה. הם שימשו כמתווכים ומשתפי פעולה עם הנאצים. 
    
    כאשר הצלב האדום הגיע לבקר, הנאצים הכינו גטו לדוגמה. לשם כך שלחו המון אנשים לגטאות אחרים מטרזין, צבעו את טרזין, והאיכלו אותם. היהודים בגטו היו צריכים לשחק את המשחק. ``היה נראה בונבון''. הצלב האדום האמין להם (``לא ארגון וואו, ארגון בלי שיניים שיכולתו לסייע היא מול מדינות ריבוניות החתומות על אמנות (לא רק חמאס)''. הוא מצליח רק בזכות אמנות בין לאומיות [מכאן ואילך רנאט מוצדק על אונר''א והאו''ם שמום]). 
    
    תפקדי היודנראט מול הנאצים: 
    \begin{enumerate}
        \item ריכוז והעברת היהודים ומציאת מדומות מגורים ליהודי הכפרים והייערות בגטו. לא פשוט – בורשה בשלב מסויים אין מקום בבתים, ואנשים חיים בחדרי מדרגות. 
        \item ארגון מפקדים לפי מין וקבוצת גיל – כדי לדעת כמה אוכל להכניס, כמה אפשר לקחת לעבוודת כפייה וכו'. 
        \item סיוע בהחרמת רכוש, תשלומי ביצוע קנסות ויישום גיזרות שונות. 
        \item דאגה על הגיינה בגטו, כדי שמגיפות לא יצאו מחוץ לגטו. 
        \item ארגון ואיסוף היהודים לגירושים עם תחילת ``הפתרון הסופי''. היודנראטים קלטו לאן הם אוספים את היהודים מאוד מהר. 
    \end{enumerate}
    בכל פעולה הם תיווכו בין הנאצים לבין היהודים. 
    
    לאחר השואה, מעטי היודנראים ששרדו הואשמו ע''י ניצולי שואה כמשתפי פעולה עם הנאצים. חוק עשיית עוזרי הדין בנאצים ובעוזריהם נולד מהצורך, בין היתר, לפעול כנגדם. בדיעבד, היום מבינים שזה תפקיד יותר מורכב ממה שנתפש בעבר. ישנה הבנה כי למרות שחלקם רצו להיות יודנראטים כדי לקבל תנאים יותר טובים, זהו תפקיד נוראי בעיקר בזמן הפתרון הסופי. 
    
    היו מעט דברים שהיודראטים יכלו לעשות (ועשו) כדי לעזור. לדוגמה, בגטו לודג' גידלו בחצרות הבתים אוכל (חצרות קטנות, משהו בגודל של הכיתה פעמים לבניין עם 700-800 איש). הם גם הקימו מטבחים ציבוריים – מקומות שבהם בישלו אוכל והעניין יותר היגעו לשם פעם ביום כדי לקבל ארוחה אחת. זאת עשו באמצעות ההקצבה שהייתה להם. הם גם אומנם לא יכלו לעזור לאנשים בהברחות (יהרגו אותם) אך במקרים רבים העלימו עין מהברחות לתוך הגטו. אומנם היו אסורים בתי ספר, הקמת מניין, בתי חולים וכו' – אך הם אפשרו ואף הקימו מוסדות כאלו. תפקידים אזרחיים נפלו לידם, כמו דת, קבורה, חינוך ועוד – והם טיפלו בהם. 
    
    \subsubsection{מפעלי תעשייה}
    הייתה תפיסה שהיהודים לא פרודקטיבים ויצרנים. לכן יודנראטים רבים עסקו בהקמת מפעלי תעשייה, הן בשביל לספק תעסוקה והן בשביל להועיל לנאצים ולקוות שלא יהרגו אותם. ``עבודה כהצלה''. בשלב מסויים הם יבינו שבכל מקרה ירצחו את כולם, וניסו לדחות כמו שאפשר את הגירוש בתקווה שבעלות הברית ישחררו את האיזור לפני. 
    
    
    \ndoc
\end{document}