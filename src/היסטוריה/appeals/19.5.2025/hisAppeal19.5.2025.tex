%! ~~~ Packages Setup ~~~ 
\documentclass[]{article}
\usepackage{lipsum}
\usepackage{rotating}


% Math packages
\usepackage[usenames]{color}
\usepackage{forest}
\usepackage{ifxetex,ifluatex,amssymb,amsmath,mathrsfs,amsthm,witharrows,mathtools,mathdots}
\usepackage{amsmath}
\WithArrowsOptions{displaystyle}
\renewcommand{\qedsymbol}{$\blacksquare$} % end proofs with \blacksquare. Overwrites the defualts. 
\usepackage{cancel,bm}
\usepackage[thinc]{esdiff}


% Design
\usepackage[labelfont=bf]{caption}
\usepackage[margin=0.6in]{geometry}
\usepackage{multicol}
\usepackage[skip=4pt, indent=0pt]{parskip}
\usepackage[normalem]{ulem}
\forestset{default}
\renewcommand\labelitemi{$\bullet$}
\usepackage{titlesec}
\titleformat{\section}[block]
{\fontsize{15}{15}}
{\sen \dotfill (\thesection)\dotfill\she}
{0em}
{\MakeUppercase}
\usepackage{graphicx}
\graphicspath{ {./} }

\usepackage[colorlinks]{hyperref}
\definecolor{mgreen}{RGB}{25, 160, 50}
\definecolor{mblue}{RGB}{30, 60, 200}
\usepackage{hyperref}
\hypersetup{
    colorlinks=true,
    citecolor=mgreen,
    linkcolor=black,
    urlcolor=mblue,
    pdftitle={Document by Shahar Perets},
    %	pdfpagemode=FullScreen,
}
\usepackage{yfonts}
\def\gothstart#1{\noindent\smash{\lower3ex\hbox{\llap{\Huge\gothfamily#1}}}
    \parshape=3 3.1em \dimexpr\hsize-3.4em 3.4em \dimexpr\hsize-3.4em 0pt \hsize}
\def\frakstart#1{\noindent\smash{\lower3ex\hbox{\llap{\Huge\frakfamily#1}}}
    \parshape=3 1.5em \dimexpr\hsize-1.5em 2em \dimexpr\hsize-2em 0pt \hsize}



% Hebrew initialzing
\usepackage[bidi=basic]{babel}
\PassOptionsToPackage{no-math}{fontspec}
\babelprovide[main, import, Alph=letters]{hebrew}
\babelprovide[import]{english}
\babelfont[hebrew]{rm}{David CLM}
\babelfont[hebrew]{sf}{David CLM}
%\babelfont[english]{tt}{Monaspace Xenon}
\usepackage[shortlabels]{enumitem}
\newlist{hebenum}{enumerate}{1}

% Language Shortcuts
\newcommand\en[1] {\begin{otherlanguage}{english}#1\end{otherlanguage}}
\newcommand\he[1] {\she#1\sen}
\newcommand\sen   {\begin{otherlanguage}{english}}
    \newcommand\she   {\end{otherlanguage}}
\newcommand\del   {$ \!\! $}

\newcommand\npage {\vfil {\hfil \textbf{\textit{המשך בעמוד הבא}}} \hfil \vfil \pagebreak}
\newcommand\ndoc  {\dotfill \\ \vfil {\begin{center}
            {\textbf{\textit{שחר פרץ, 2025}} \\
                \scriptsize \textit{קומפל ב־}\en{\LaTeX}\,\textit{ ונוצר באמצעות תוכנה חופשית בלבד}}
    \end{center}} \vfil	}

\newcommand{\rn}[1]{
    \textup{\uppercase\expandafter{\romannumeral#1}}
}

\makeatletter
\newcommand{\skipitems}[1]{
    \addtocounter{\@enumctr}{#1}
}
\makeatother

\author{שחר פרץ}
\title{\textit{דברים שלא הבנתי מבדיקה שלך את המבחן $\sim$ מבחן שכבתי מרץ $\sim$ תשפ''ה $\sim$ לאומיות וציונות}}
\date{9 ליוני 2025, על מבחן מה־19 במאי}
\begin{document}
    \maketitle
    
    שלום שרית, 
    
    היו כמה דברים במבחן האחרון שכתבת על הבחינה ואני לא בטוח למה התכוונת בהם. אשמח אם תמצאי את הזמן להרחיב קצת יותר. 
    
    העמודים (שמספרתי עוד במהלך הבחינה) ממספרים מ־A ל־F ואתייחס למספור זה. 
    
    \begin{enumerate}
        \item בעמוד A: לא ברור לי מה לא מלא בהגדרה (מה ספציפית הייתי צריך להרחיב עליו). 
        \item בעמוד C, כתבת ``אל תבלבל'' כאשר התחלתי להסביר על היסטוריה משותפת ואיך היא הפכה לחשובה (תנועת הרומנטיקה), כתבתי לי ``אל תבלבל'' על הרומנטיקה וכן שההגדרה שלי לא הייתה מלאה. לא בדיוק הבנתי מה חסר בתיאור שלי. 
        \item בעמוד E, לא הבנתי למה הקפת את ``תורת הגזע'' (לא הסברת בצד)
        \item בעמוד G, כאשר כתבתי ``מאאסר יותר עדיף ובעילות מומצאת'' והשוותי אותו ל''מאסר ישיר והמוני של מאות אלפי יהודים'' – ניסיתי להתייחס בעקיפין להיות המאסר מאסר כולל ומאורגן של יהודים על בסיס גזע, ולא מאסר של יהודים באקראי על בסיס חוקים עקיפים וסיבות שאינן מוגדרות היטב. אני לא יודע אם את מקבלת את זה, כי לא הבהרתי את עצמי טוב במבחן, אבל שאלת ``מה זה מאסר עקיף'' על הטופס  – אני מקווה שהסברתי את עצמי. 
        \item בעמוד F, גם כאן לא ברורה לי הטענה שההגדרה שלי של תורת הגזע והאנטישמיות לא מלאה – הסברתי איך שני המושגים הללו משתקפים באידיאולוגיה הנאצית, מה החסרתי? 
        
        פרט לכך, מבחינת סדר הקריאה עמוד F בא לפני עמוד G למרות שבדפים עצמם הם מסודרים הפוך, אני מקווה שזה היה ברור ולא קראת את השאלה הפוך. 
    \end{enumerate}
    
    אני לא מאוד מרוצה מביצועים שלי במבחנים השנה וארצה להשתפר לקראת שנה הבאה, ולכן אעריך כל תשובה שתעזור לי להבין יותר טוב איפה טעיתי (גם אם הציון ישאר זהה או ירד).
    
    \hfill תודה מראש – שחר פרץ
    
    \dotfill
\end{document}