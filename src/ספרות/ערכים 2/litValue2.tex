%! ~~~ Packages Setup ~~~ 
\documentclass[]{article}
\usepackage{lipsum}
\usepackage{rotating}


% Math packages
\usepackage[usenames]{color}
\usepackage{forest}
\usepackage{ifxetex,ifluatex,amssymb,amsmath,mathrsfs,amsthm,witharrows,mathtools,mathdots}
\usepackage{amsmath}
\WithArrowsOptions{displaystyle}
\renewcommand{\qedsymbol}{$\blacksquare$} % end proofs with \blacksquare. Overwrites the defualts. 
\usepackage{cancel,bm}
\usepackage[thinc]{esdiff}


% tikz
\usepackage{tikz}
\usetikzlibrary{graphs}
\newcommand\sqw{1}
\newcommand\squ[4][1]{\fill[#4] (#2*\sqw,#3*\sqw) rectangle +(#1*\sqw,#1*\sqw);}


% code 
\usepackage{algorithm2e}
\usepackage{listings}
\usepackage{xcolor}

\definecolor{codegreen}{rgb}{0,0.35,0}
\definecolor{codegray}{rgb}{0.5,0.5,0.5}
\definecolor{codenumber}{rgb}{0.1,0.3,0.5}
\definecolor{codeblue}{rgb}{0,0,0.5}
\definecolor{codered}{rgb}{0.5,0.03,0.02}
\definecolor{codegray}{rgb}{0.96,0.96,0.96}

\lstdefinestyle{pythonstylesheet}{
    language=Java,
    emphstyle=\color{deepred},
    backgroundcolor=\color{codegray},
    keywordstyle=\color{deepblue}\bfseries\itshape,
    numberstyle=\scriptsize\color{codenumber},
    basicstyle=\ttfamily\footnotesize,
    commentstyle=\color{codegreen}\itshape,
    breakatwhitespace=false, 
    breaklines=true, 
    captionpos=b, 
    keepspaces=true, 
    numbers=left, 
    numbersep=5pt, 
    showspaces=false,                
    showstringspaces=false,
    showtabs=false, 
    tabsize=4, 
    morekeywords={as,assert,nonlocal,with,yield,self,True,False,None,AssertionError,ValueError,in,else},              % Add keywords here
    keywordstyle=\color{codeblue},
    emph={var, List, Iterable, Iterator},          % Custom highlighting
    emphstyle=\color{codered},
    stringstyle=\color{codegreen},
    showstringspaces=false,
    abovecaptionskip=0pt,belowcaptionskip =0pt,
    framextopmargin=-\topsep, 
}
\newcommand\pythonstyle{\lstset{pythonstylesheet}}
\newcommand\pyl[1]     {{\lstinline!#1!}}
\lstset{style=pythonstylesheet}

\usepackage[style=1,skipbelow=\topskip,skipabove=\topskip,framemethod=TikZ]{mdframed}
\definecolor{bggray}{rgb}{0.85, 0.85, 0.85}
\mdfsetup{leftmargin=0pt,rightmargin=0pt,innerleftmargin=15pt,backgroundcolor=codegray,middlelinewidth=0.5pt,skipabove=5pt,skipbelow=0pt,middlelinecolor=black,roundcorner=5}
\BeforeBeginEnvironment{lstlisting}{\begin{mdframed}\vspace{-0.4em}}
    \AfterEndEnvironment{lstlisting}{\vspace{-0.8em}\end{mdframed}}


% Deisgn
\usepackage[labelfont=bf]{caption}
\usepackage[margin=0.5in]{geometry}
\usepackage{multicol}
\usepackage[skip=4pt, indent=0pt]{parskip}
\usepackage[normalem]{ulem}
\forestset{default}
\renewcommand\labelitemi{$\bullet$}
\usepackage{titlesec}
\titleformat{\section}[block]
{\fontsize{15}{15}}
{\sen \dotfill (\thesection)\dotfill\she}
{0em}
{\MakeUppercase}
\usepackage{graphicx}
\graphicspath{ {./} }

\usepackage[colorlinks]{hyperref}
\definecolor{mgreen}{RGB}{25, 160, 50}
\definecolor{mblue}{RGB}{30, 60, 200}
\usepackage{hyperref}
\hypersetup{
    colorlinks=true,
    citecolor=mgreen,
    linkcolor=black,
    urlcolor=mblue,
    pdftitle={Document by Shahar Perets},
    %	pdfpagemode=FullScreen,
}


% Hebrew initialzing
\usepackage[bidi=basic]{babel}
\PassOptionsToPackage{no-math}{fontspec}
\babelprovide[main, import, Alph=letters]{hebrew}
\babelprovide[import]{english}
\babelfont[hebrew]{rm}{David CLM}
\babelfont[hebrew]{sf}{David CLM}
%\babelfont[english]{tt}{Monaspace Xenon}
\usepackage[shortlabels]{enumitem}
\newlist{hebenum}{enumerate}{1}

% Language Shortcuts
\newcommand\en[1] {\begin{otherlanguage}{english}#1\end{otherlanguage}}
\newcommand\he[1] {\she#1\sen}
\newcommand\sen   {\begin{otherlanguage}{english}}
    \newcommand\she   {\end{otherlanguage}}
\newcommand\del   {$ \!\! $}

\newcommand\npage {\vfil {\hfil \textbf{\textit{המשך בעמוד הבא}}} \hfil \vfil \pagebreak}
\newcommand\ndoc  {\dotfill \\ \vfil {\begin{center}
            {\textbf{\textit{שחר פרץ, 2025}} \\
                \scriptsize \textit{קומפל ב־}\en{\LaTeX}\,\textit{ ונוצר באמצעות תוכנה חופשית בלבד}}
    \end{center}} \vfil	}

\newcommand{\rn}[1]{
    \textup{\uppercase\expandafter{\romannumeral#1}}
}

\makeatletter
\newcommand{\skipitems}[1]{
    \addtocounter{\@enumctr}{#1}
}
\makeatother


%! ~~~ Document ~~~

\author{\normalsize שחר פרץ}
\title{\Large \textit{ערכים $\sim$ כתיבה בעקבות אנטיגונה $\sim$ כתיבה בעקבות אנטיגונה}}
\date{\normalsize 15 במאי 2025}
\begin{document}
    \maketitle
    \small
    \textit{[הערה 1: אני עומד לכתוב כחצי עמוד על דברים שנראים לכאורה לא קשורים, אבל אני עומד לקשר בינהם אחר כך]}
    
    \textit{[הערה 2: אני מסתמך על העבודה הקודמת שעדיין לא הגשתי כי לא היה שום שיעור מאז שניתנה לי]}
    
    בעבודה הקודמת (שאגיש בראשון), נימקתי מדוע הערך היחיד שאני עוקב לפיו הוא יכולת ההשרדות של אנושות לטווח זמן ארוך. לפיכך, קונפליקט בין הערכים שלי יצטרך להיות קונפליקט בין אותו הערך לעצמו (כלומר, נסיון להבין באיזה מבין האפשרויות ``גרועה יותר''). לשם כך, אציג שאלה שלכאורה תשובתה פשוטה ולא רצויה, אך אותה התשובה שאיננה רצויה נובעת מבעיות באופן ההיסק. 
    
    ראשית ארצה להציג קונספט שבהתחלה לא יהיה קשור לכלום, אבל בעוד כמה פסקאות אתחיל לקשר אותו לדברים.  
    
    ישנה תורה במתמטיקה (שאני לא מבין מספיק לעומק, אבל אני מקווה שהבנתי השטוחה שלי תספיק לצורך דיון זה) העוסקת בכאוס – מה קורה אם בתוך מערכת כלשהי המתארת לי דבר מה, כתלות באיזשהו פרמטר (לצורך הדוגמה – זמן) התנהגותה משתנה מקצה אל הקצה בצורה ``בלתי ניתנת לחזייה'' – בינתן תנאי התחלה מעט שונים התוצאה תהיה אחרת לחלוטין, וקשה ליצור מודל מתמטי ``פשוט'' שקל לחשב. דוגמאות נפוצות למערכות כאלו הן מערכות של זוג מטוטלות, שתנהגותן משתנה במהירות ודי באקראי (ואף בתנאי םמושלמים, כלומר רביק וללא תאוצה). דוגמה אחרת היא ניתוח מזג אוויר. 
        
        אופן הניבוי של מערכת מעין אלו לרוב מסתמך על כלים אמפריים (מדידתיים), סטטיסטיים, והיורסטיים (נסיוניים). במילים אחרות, אם ארצה לתאר התנהגות של מערכת כזו – לא אוכל להשתמש בכלים כמו סימולציה של מחשב המתבססת על חוקים פיזיקלים ברמה האטומית לבדם – אצטרך לבסס חוקים היורסטים, שאני יודע סטטיסטית שעובדים די טוב, וסביבם לבנות מערכות ניבוי. זה נובע מאופיה הכיאוטי של המערכת, שלולא דיוק כמעט אינסופי בתנאי ההתחלה וכן משאבי מחשוב כמעט אינסופיים גם הם, היא לא תהיה בעלת תשובה מכרעת. 
    
    כמו שכתבתי בעבודה בלשון שמעולם לא הגשתי, ב־28 במרץ 2024, הורשע סם בנקמן־פריד (מעתה ואילך יקרא "SBF") בהונאות משקיעים בעולם הקריפטו בסכומים של מיליארדי דולרים, ונאסר ל־25 שנה. לאורך המשפט, SBF טען שהשתמש בכוחו כדי להפוך את העולם למקום יותר טוב. ע''פ עדויות כאלו ואחרות (זה לא נאמר ישירות, אלא נובע מתמיכתו בתורות תועלתניות מסוימות) הועלו ספקולציות (דבר אמיתותן לא רלוונטי לעבודה זו, שמנסה לתאר שיפוט של אירוע תיאורטי גרידא), הובן כי SBF תמך בגישות תועלתניות מסויימות שגרמו לו לחשוב כי הבעיה המרכזית של האנושות היא המקום בכדו"א והסיכון שהאנושות תשמיד אותו, וכי הוא מכיר בבעיה ועל כן מוסרי עליו לפעול בדרכים לא חוקיות, ששוללות את הזכות של אנשים לבחירה חופשית על כיצד ישתמשו בממונם, כדי להשקיע בפתרון הבעיה.
    
    משום שרוב מה שאני עומד לדבר עליו עומד בסתירה מוחלטת לערך העליון בעיני כל הדמויות ב''אנטיגונה'' – הוא חוק האלים, ורוב תתי הערכים והפירושים של אותו החוק (ששים בהעדפה דברים כמו משפחה, הממלכה, וכו'), בעוד תחת המערכת שאני הגדרתי הם נובעים בצורה יותר עקיפה (פירטתי על כך בעבודה הקודמת) שההכרעה בגינה תלויה פר־מקרה, אני לא רואה ערך בלדבר עליהם בהקשר הזה. לכן אני אקשר את זה להקשר אחר – 
    
    ב''אנטיגונה'', שני הגיבורים הטרגיים חוטאים בחטא ההיבריס. על כן, הם בטוחים בצדקתם שלהם ונמנעים מלהקשיב לחוכמת ההמונים. המקהלה מייצגת את ההפך הגמור – היא מעמידה את קריאון על טעותו (בעיקיפין, לדוגמה באמרה כי אולי יד האלים בקבורת פולינקס) וכן את אנטיגונה (הם מציינים כי אנטיגונה לא יכולה להשוות את עצמה לאלה, מיד לאחד שעושה זאת). בהקשר הזה, היא מייצגת צדק מורכב יותר, חכם יותר, שלא רק מניח את מה שנוח לו להניח (קל היה להם לחשוב על אנטיגונה, שהעם כולו והמקהלה חיבבו, כעל נעלה יותר מאשר היא באמת) אלא פועל לפי עקרונות מוסר יותר בסיסיים (לפיהם, לא משנה כמה אוהבים את אנטיגונה, היא איננה אלה). 
    
    עתה, ננתח את אותו האירוע תוך שימוש באמור לעיל: 
    \begin{itemize}
        \item מצד אחד, SBF מנסה להציג עמדה שנשמעת די סבירה – אם מטרתי היא שהאנושות תשרוד לזמן ארוך ככל האפשר (אפשר ששבמקרה האמיתי זה נובע כמסקנה מהנחות אחרות, ואינו נלקח כאקסיומה, אך כמו שצוין תרגיל זה הוא תיאורטי בלבד ולכן אין זה רלוונטי), אזי עלי להגן עליה מפני איומים כמו נשק גרעיני, שינויי אקלים, פגיעת מטאור וכו' – אירועים כטסטרופיים, שחרף היותם לא סבירים במיוחד – יכולים להשמיד כליל את כל האנושות, ולכן יש להמנע מהם בכל מכיר. דרך מתבקשת להמנע מאותם האיומים היא ריחוק למקומות בלתי תלויים בכדו''א, כלומר, פלנטות אחרות. מכאן SBF מסיק (לכאורה) שהדבר המתבקש הוא שבכל מחיר עלינו להגיע לפלנטה אחרת, והמחיר יכול להיות הונם של אנשים רבים שלא הסכימו לכך. 
        \item ארצה להקביל את SBF וטיעוניו להיבריס של קריאון ואנטיגונה – השתקעות עצמית כל כך מרובה, כך שאינו יכול לדמיין מצב בו דעתו שלו שגויה. לשם כך, אצביע על שלושה כשלים בטיעוניו: 
        \begin{enumerate}
            \item ראשית, הטיעון מחיל בתוכו הנחות לא מבוססות רבות. דוגמאות להנחות כאלו, הוא שהסיכון הגדול ביותר שעומד על האנושות הוא אותם אירועים כטסטרופיים שישמידו את הפלנטה. אפשר ויש בעיות קצרות טווח יותר שמעמידות אותנו בסכנה, ו־SBF כלל לא מתייחס לאירוע כזה. באופן דומה הוא מסיק שבריחה לפלנטה אחרת תציל אותנו מאותו הגורל. הוא אינו יכול להניח, לדוגמה, שאת תתפשט מלחמה כזו היא אינה תגיע באופן מהיר ולפנטות אחרות (ואף אם היה באופן איטי, די בכך כדי שהוא יאבד את הטיעון האסימפטוטי מאחורי טענותיו). 
            
            במילים אחרות, הוא מניח הנחות שנשמעות סבירות אך למעשה מתייחסות למקרים מאוד ספציפיים. 
            \item שנית, הוא מסרב להקשיב לעקרון חוכמת ההמונים – יתכן שהוא טועה. סטטיסטית, הוא טועה. אם היה מאפשר למערכות דמוקרטיות חוקיות להגיע למסקנה דומה – אם היה רק מקשיב למקהלה – אולי היה מגיע לתוצאות אחרות לחלוטין, שסטטיסטית (עד כמה שאפשר למדוד זאת) יהיו עדיפות לכולם. 
            
            \item אחרונה, הוא מנסה לנתח מערכת כאוטית ומורכת לאין שיעור כמו התנהגות האנושות והטבע, באמצעות כלים ולוגיקה ``פשוטים'' – טיעונים אלו הם נפנופי ידיים גמורים שבינם לבין המציאות אין קשר, מהסיבות המתוארות בפסקה השלישית. במקום זאת, היה עליו להשתמש בנימוקים העוסקים בצורה סטטיסטית באופן ההתנהגות של האנושות או של הטבע. 
        \end{enumerate}
    \end{itemize}
    
    (מצטער, לא הספקתי לעשות הגהה רצינית, ויש סיכוי שממש התאמצתי כדי שזה יכנס לדף אחד. מקווה שזה בסדר שאני מגיש באיחור של 5 דקות, חזרתי די מאוחר ממיונים של הרובוטיקה ואתמול היה לי אודיסאה, לא קלטתי שהייתי צריך לסיים את זה בשלישי)
    
\end{document}