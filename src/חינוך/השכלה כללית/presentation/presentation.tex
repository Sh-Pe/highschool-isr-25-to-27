%! ~~~ Packages Setup ~~~
\documentclass[]{beamer}
\usetheme{Boadilla}
\usepackage{ragged2e}
\apptocmd{\frame}{}{\justifying}{}
\setbeamertemplate{navigation symbols}{}



% Math packages
\usepackage{ifxetex,ifluatex,amssymb,amsmath,mathrsfs,amsthm,witharrows,mathtools,mathdots}
\usepackage{amsmath}
\WithArrowsOptions{displaystyle}
\renewcommand{\qedsymbol}{$\blacksquare$} % end proofs with \blacksquare. Overwrites the defualts. 
\usepackage{cancel,bm}
\usepackage[thinc]{esdiff}


% Design
\usepackage[labelfont=bf]{caption}
\usepackage{graphicx}
\graphicspath{ {./} }

% Hebrew initialzing
\usepackage[bidi=basic]{babel}
\usepackage{multicol}
\PassOptionsToPackage{no-math}{fontspec}
\babelprovide[main, import, Alph=letters]{hebrew}
\babelprovide[import]{english}
\babelfont[hebrew]{rm}{David CLM}
\babelfont[hebrew]{sf}{David CLM}
%\babelfont[english]{tt}{Monaspace Xenon}

% Language Shortcuts
\newcommand\en[1] {\begin{otherlanguage}{english}#1\end{otherlanguage}}
\newcommand\he[1] {\she#1\sen}
\newcommand\sen   {\begin{otherlanguage}{english}}
    \newcommand\she   {\end{otherlanguage}}
\newcommand\del   {$ \!\! $}



%! ~~~ Document ~~~

\author{שחר פרץ}
\title{\textit{חושבים וכותבים ערכים}}
\begin{document}
    \maketitle
    \begin{frame}{פעילות חרדים 2}
        \begin{itemize}
            \item הפעילות עצמה ומהלכה
            \item הערכים שנבחרו: 
            \begin{itemize}
                \item תקשורת והבנה של האחר
                \item לאומיות
                \item אחדות
                \item מעורבות חברתית ועזרה לאחרים
                \item משפחתיות
            \end{itemize}
        \end{itemize}
    \end{frame}
    
    \begin{frame}{ביטוי אומנותי – האגס 1 $\backslash$ אהוד בנאי}
        \begin{multicols}{2}
             רחוב האגס אחד \\
            מעל לחנות הירקות,\\
            הבית ריק עכשיו\\
            וחשופים הם הקירות,\\
            אבל ספוגים הם\\
            זכרונות של חג,\\
            ריחות של יסמין\\
            וניגון ישן שלסעודה מזמין. \\
            \vspace{0.2cm}
            רחוב האגס אחד\\
            ליד מורטות העופות,\\
            ספרי קודש ישנים\\
            מצהיבים בארונות,\\
            ואין עצה ואין תבונה\\
            כנגד הזמן,\\
            הולך לו הגדול\\
            מגיע הקטן.\\
            \vspace{0.2cm}
            אסדר לסעודתא\\
            בצפרא דשבתא,\\
            ואזמין בה השתא\\
            עתיקא קדישא.\\
            \vspace{0.2cm}
            נהוריה ישרי בה\\
            בקידושא רבא,\\
            ובחמרא טבא\\
            דבה תחדי נפשא. 
        \end{multicols}
    \end{frame}
    
    \begin{frame}{הקשר של היבטוי האומנותי לערכים הנבחרים}
        בשיר, אהוד בנאי מתייחס לזכרונותיו בתוך חברה דתית. חברה שקהל היעד שלו לא מכיר, והוא מתגעגע אליה. זה מתקשר הדוקות לערך המשפחתיות והאחדות. 
    \end{frame}
    
    \begin{frame}{סיכום אישי}
        \begin{itemize}
            \item דברים שלמדתי במהלך הפעילות
            \item דברים שלמדתי בעקבות העבודה
            \item סיכום ודגשים
        \end{itemize}
    \end{frame}
    
\end{document}