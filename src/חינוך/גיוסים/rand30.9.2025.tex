%! ~~~ Packages Setup ~~~ 
\documentclass[]{article}
\usepackage{lipsum}
\usepackage{rotating}


% Math packages
\usepackage[usenames]{color}
\usepackage{forest}
\usepackage{ifxetex,ifluatex,amssymb,amsmath,mathrsfs,amsthm,witharrows,mathtools,mathdots,centernot}
\usepackage{amsmath}
\WithArrowsOptions{displaystyle}
\renewcommand{\qedsymbol}{$\blacksquare$} % end proofs with \blacksquare. Overwrites the defualts. 
\usepackage{cancel,bm}
\usepackage[thinc]{esdiff}


% tikz
\usepackage{tikz}
\usetikzlibrary{graphs}
\newcommand\sqw{1}
\newcommand\squ[4][1]{\fill[#4] (#2*\sqw,#3*\sqw) rectangle +(#1*\sqw,#1*\sqw);}


% code 
\usepackage{algorithm2e}
\usepackage{listings}
\usepackage{xcolor}

\definecolor{codegreen}{rgb}{0,0.35,0}
\definecolor{codegray}{rgb}{0.5,0.5,0.5}
\definecolor{codenumber}{rgb}{0.1,0.3,0.5}
\definecolor{codeblue}{rgb}{0,0,0.5}
\definecolor{codered}{rgb}{0.5,0.03,0.02}
\definecolor{codegray}{rgb}{0.96,0.96,0.96}

\lstdefinestyle{pythonstylesheet}{
	language=Java,
	emphstyle=\color{deepred},
	backgroundcolor=\color{codegray},
	keywordstyle=\color{deepblue}\bfseries\itshape,
	numberstyle=\scriptsize\color{codenumber},
	basicstyle=\ttfamily\footnotesize,
	commentstyle=\color{codegreen}\itshape,
	breakatwhitespace=false, 
	breaklines=true, 
	captionpos=b, 
	keepspaces=true, 
	numbers=left, 
	numbersep=5pt, 
	showspaces=false,                
	showstringspaces=false,
	showtabs=false, 
	tabsize=4, 
	morekeywords={as,assert,nonlocal,with,yield,self,True,False,None,AssertionError,ValueError,in,else},              % Add keywords here
	keywordstyle=\color{codeblue},
	emph={var, List, Iterable, Iterator},          % Custom highlighting
	emphstyle=\color{codered},
	stringstyle=\color{codegreen},
	showstringspaces=false,
	abovecaptionskip=0pt,belowcaptionskip =0pt,
	framextopmargin=-\topsep, 
}
\newcommand\pythonstyle{\lstset{pythonstylesheet}}
\newcommand\pyl[1]     {{\lstinline!#1!}}
\lstset{style=pythonstylesheet}

\usepackage[style=1,skipbelow=\topskip,skipabove=\topskip,framemethod=TikZ]{mdframed}
\definecolor{bggray}{rgb}{0.85, 0.85, 0.85}
\mdfsetup{leftmargin=0pt,rightmargin=0pt,innerleftmargin=15pt,backgroundcolor=codegray,middlelinewidth=0.5pt,skipabove=5pt,skipbelow=0pt,middlelinecolor=black,roundcorner=5}
\BeforeBeginEnvironment{lstlisting}{\begin{mdframed}\vspace{-0.4em}}
	\AfterEndEnvironment{lstlisting}{\vspace{-0.8em}\end{mdframed}}


% Design
\usepackage[labelfont=bf]{caption}
\usepackage[margin=0.6in]{geometry}
\usepackage{multicol}
\usepackage[skip=4pt, indent=0pt]{parskip}
\usepackage[normalem]{ulem}
\forestset{default}
\renewcommand\labelitemi{$\bullet$}
\usepackage{graphicx}
\graphicspath{ {./} }

\usepackage[colorlinks]{hyperref}
\definecolor{mgreen}{RGB}{25, 160, 50}
\definecolor{mblue}{RGB}{30, 60, 200}
\usepackage{hyperref}
\hypersetup{
	colorlinks=true,
	citecolor=mgreen,
	linkcolor=black,
	urlcolor=mblue,
	pdftitle={Document by Shahar Perets},
	%	pdfpagemode=FullScreen,
}
\usepackage{yfonts}
\def\gothstart#1{\noindent\smash{\lower3ex\hbox{\llap{\Huge\gothfamily#1}}}
	\parshape=3 3.1em \dimexpr\hsize-3.4em 3.4em \dimexpr\hsize-3.4em 0pt \hsize}
\def\frakstart#1{\noindent\smash{\lower3ex\hbox{\llap{\Huge\frakfamily#1}}}
	\parshape=3 1.5em \dimexpr\hsize-1.5em 2em \dimexpr\hsize-2em 0pt \hsize}



% Hebrew initialzing
\usepackage[bidi=basic]{babel}
\PassOptionsToPackage{no-math}{fontspec}
\babelprovide[main, import, Alph=letters]{hebrew}
\babelprovide[import]{english}
\babelfont[hebrew]{rm}{David CLM}
\babelfont[hebrew]{sf}{David CLM}
%\babelfont[english]{tt}{Monaspace Xenon}
\usepackage[shortlabels]{enumitem}
\newlist{hebenum}{enumerate}{1}

% Language Shortcuts
\newcommand\en[1] {\begin{otherlanguage}{english}#1\end{otherlanguage}}
\newcommand\he[1] {\she#1\sen}
\newcommand\sen   {\begin{otherlanguage}{english}}
	\newcommand\she   {\end{otherlanguage}}
\newcommand\del   {$ \!\! $}

\newcommand\npage {\vfil {\hfil \textbf{\textit{המשך בעמוד הבא}}} \hfil \vfil \pagebreak}
\newcommand\ndoc  {\dotfill \\ \vfil {\begin{center}
			{\textbf{\textit{שחר פרץ, 2025}} \\
				\scriptsize \textit{קומפל ב־}\en{\LaTeX}\,\textit{ ונוצר באמצעות תוכנה חופשית בלבד}}
	\end{center}} \vfil	}

\newcommand{\rn}[1]{
	\textup{\uppercase\expandafter{\romannumeral#1}}
}

\makeatletter
\newcommand{\skipitems}[1]{
	\addtocounter{\@enumctr}{#1}
}
\makeatother


%! ~~~ Document ~~~

\author{שחר פרץ}
\title{\textit{הכנה לצו ראשון}}
\begin{document}
	\maketitle
	
	\section{מבוא}
	
	\subsection*{צעד קדימה}
	טל': 03-5344435
	
	אימייל: office@stepforward.co.il
	
	אתר: stepforward.co.il
	
	נדבר על מבחנים פסיכוטכניים (דפ''ר), סוגי השאלות בהם וכו'. נדבר על הכלים שאפשר לצבור לקראת הראיון האישי, ועל עוד משהו שפספסתי. 
	
	\subsection*{צו ברשת}
	
	לפני שמתגייסים לצו הראשון צריך למלא ``צו ברשת'' באתר מתגייסים שקורס בהתמדה פעם ביום כל יום. בצו ברשת מכניסים: 
	\begin{itemize}
		\item שאלו אימות נתונים (כתובת, שמות הורים וכו')
		\item שאלון רפואי (תוצאות בדיקות שתן וראייה – רופאי המשפחה מכירים את זה)
		\item שליחת מסמכים. כל מה שקשור לצו הראשון ולצבא לאחר מכן: 
		\begin{itemize}
			\item הבחנות רפואיות
			\item הבחנות פסיכיאטריות
			\item לקויות למידה (הפרעות קשב וריכוז)
		\end{itemize}
		כדאי להביא את אותם המסמכים ללשכה. 
	\end{itemize}
	במכתב הזימון יש משתמש אישי כדי להכנס לאתר. 
	
	אם לא מילאתם צו ברשת – בצו הראשון יחיזרו אותכם לבית למלא את זה. 
	
	\subsection*{תהליך הגיוס}
	תהליך הגיוס כדלהמן: 
	
	\begin{enumerate}
		\item צו ראשון – ידובר בהרחבה. דפ''ר רואים עוד באותו היום או מעט מאוחר יותר. פרופיל באופן מיידי. 
		\item קבלת תאריך גיוס – תאריך הגיוס ששולחים לאחר הצו הראשון ``פיקטיבי ולא אמיתי''. 
		\item 	יום המא''ה – היום לא נדבר על יום המא''ה, אבל באופן כללי יש שם מבחנים נוספים שצריך לעשות. הם נעשי םמהבית בזום. 
		\item 	שאלון העדפות – בשאלון ההעדפות ישנה רשימה של תפקידים אליהם נוכל להתגייס. אפשר לדרג אותה כיצד שנרצה, אך כמובן בסוף הצבא מגייס לאן שהוא צריך. 
		\item מיונים גיבושים וימי סיירות, שיבוץ חזוי, וגיוס לצה''ל. 
	\end{enumerate}
	המיונים לטייס ומודיעין מתחילים בי''א אחרי יום המא''ה, כי הם אורכים יותר זמן. 
	
	אם עוברים שבוע־שבועיים ולא מקבלים תוצאות לצו ראשון – כדאי להתקשר למיטב. 
	
	אז מה נקבע בצו הראשון? 
	\begin{itemize}
		\item השכלה
		\item דפ''ר
		\item ראיון אישי
		\item סימול עברית (בין 5-8), נדבק תוך כדי הראיון. אם אתם יודעים לדבר עברית שותף הכל יהיה נורמלי. אין מה להתכונן. 
		\item קשב מתמשך (לא משוקלל בדפ''ר, למרות שעושים אותו ביחד עם הדפ''ר)
	\end{itemize}
	
	\subsection*{פרופילים}
	להלן הציונים של פרופילים: 
	\begin{itemize}
		\item 97 = כשיר ליחידות מובחרות
		\item 82 = כשיר לקיבוץ בחי''ר ושדה
		\item 72 = כשיר לשיבוץ בקרבי מלבד לחי''ר
		\item 64 = כשיר לשיבוץ בתומך לחימה, יחידות עורפיות ובחלק מיחידות הלוחמה
		\item 45 = כשיר לשיבוץ ביחידות עורפיות
		\item 24 = בלתי קשיר ארעית
		\item 21 = בלתי כשיר תמידית, אפשרות להתדב
	\end{itemize}
	
	אם מישהו רוצה להגיע לפרופיל גבוהה, לא מומלץ להסתיר דברים. בסוף תתקעו ביחידה עם בעיות. בכל עדכון במצב הרפואי יש לעדכן את לשכת הגיוס בצירוף המסמכים המתאימים. פרופיל 97 או 82 כנראה ישלח לקרבי. 
	
	הצו הראשון מורכב מכמה תחנות: 
	\begin{itemize}
		\item בדיקה רפואית
		\item המדור הפסיכוטכני, מבחני דפ''ר
		\item הראיון האישי
	\end{itemize}
	זה יכול לקחת שלוש שעות, יכול להיות שתגיעו בשמונה ותצאו בשלוש. תלוי בעומסים. אל תסמכו על הקיוסק בתל השומר ותביאו אוכל, מים, תעודה מזהה, תעודת זהות ודרכון. 
	
	עתה נדבר בעיקר על הדפ''ר והראיון האישי. 
	
	\section{דפ''ר}
	הדפ''ר מוחלק למספר חלקים: 
	\begin{itemize}
		\item \textbf{הבנת הנקרא} – בעיקר מכוון לעולים חדשים, לא רלוונטי לרוב האנשים. 
		\item \textbf{קשב מתמשך} – בעצם שייך ליום המא''ה רק שמבצעים אותו יחד עם הדפ''ר. 
		\item עוד כמה שנדבר עליהם בהמשך
	\end{itemize}
	המבחן אדפטיבי. הוא מתאים את הרמה של עצמו בהתאם לתשובות שלחם. המשמעות – אי אפשר לענות על שאלות אחורה. באין ברירה, עדיף לנחש, תשובה שגויה מהווה טעות. אלו לא שאלות קשות, אלא כאלו שנועדו לבלבל. הדפ''ר בדומה לתיאוריה, ניעזר במחשב עם חצצים בינהם. הגבלת הזמן היא פר שאלה, לא פר פרק! הזמן משתנה בין שאלה לשאלה, בין דקה לדקה וחצי. הזמן לא נצבר, לכן לפעמים לא חייבים להזדרז. 
	
	מקבלים כמה עזרים: 
	\begin{itemize}
		\item עפרון ודף טיוטה, מותר לרשום משני הצדדים. לא בודקים אותו. 
		\item מחשבון פשוט, לא מדעי (גם אין צורך בכזה). 
		\item אוזניות לבידוד רעשים/הקראה. לא חובה. 
		\item מבחן בשפה זרה אפשרי, למי שזה לרוונטי עבורו. זמין גם באמהרית. לבקש מראש. 
	\end{itemize}
	
	כל מה שקשור ללקויות למידה, והפרעות קשב וריכוז – מכבדים את בצו הראשון. זה מוריד פרופיל. מי שיש לו לקות חמורה בחשבון (דיסלקוליה) יוכל להבחן להאם אשכרה יש לו את זה לפני הדפ''ר. האבחונים – צריכים להיעשות לאחר סיום כיתה ו'. האבחון חייב לכלול שם, ת''ז, מועד ביצוע, היכן בוצע וע''י מי. מי שנוטל ריטלין וכיו''ב שיביא מרשם. 
	
	ציון הדפ''ר בין 10-90, בקפיצות של 10. ניתן להגיש ערעור על ציון דפ''ר של 80 (כולל) ומטה (כלומר לא 90). ניתן להגיש ערעור רק אחרי יום המא''ה. לוקחים את הציון הגבוהה מבין השניים. טיפים לדפ''ר: 
	
	\begin{itemize}
		\item להגיע רעננים, אחרי שנת לילה טובה. 
		\item להביא מים וארוחה קלה מהבית. 
		\item להגיע בזמן! תשכת תל השומר היא המקום הכי פקוק במדינה. הפגנות חרדים + מלחמות + פקקים של ת''א. תשתדלו להקדים בשעה. אפשר להכנס לפני (באופן כללי). 
		\item תרגלו את כלל התחומים, ברמות המשתנות. 
	\end{itemize}
	
	יש 10-15 שאלות בפרק, כלומר בין שעה לשעה ורבע מבחן. 
	
	\subsection{אנולוגיות מילוליות}
	יש כל מני סוגי אנלוגיות, לדוגמה מילים נרדפות (שמש:חמה, מתנה:שי), הפכים, סיבה ותוצאה (אימון:כושר, חולי:חולשה), פריט וקטגוריה (יונה:ציפור, סכין:כלי מטבח) וכו'. 
	
	דוגמה ראשונה: נתון פרק:סדרה. מבין ארבעת התשובות: פסקה:טקסט, רכיב ושלם. שימו לב ששחקנים:תיאטרון או מספרים:קלמר, לא ממש עובד כי מספריים/תיאטרון מורכבים מאוד דברים. שימו לב – היחס לא בהכרח סימטרי. 
	
	דוגמה יותר מעניינת: נתון תרימת:יושרה, והאפשרויות ארגון:סדר, בדידות:שמחה, נדבה:אנוכיות, מוטיבציה:עייפות, אושר:לחץ. התשובה – נדבה:אנוכיות, היחס של בדידות:שמחה לא ממש מתאים כי הוא לא מתאר פעולות. 
	
	
	\subsection{אנלוגיות צורניות}
	אין לי ממש מה לצייר כאן. לרוב זה הכפלה, סיבוב ושינוי צבע. שימו לב לתשובות דומות. לפעמים ישנו סתם מאפיין אקראי בלי פואנטה. 
	
	\subsection{חשיבה כמותית}
	\begin{itemize}
		\item לקרוא את השאלה במלואה ולהסביר במילים שלנו
		\item לאתר נתונים חשובים בשאלה ולסדר אותם בדף הטיוטה
		\item לעבוד בשלבים ובמידת הצורך להעיזר במחשבון
		\item פחות לחשב, יותר לחשוב
		\item יש מגוון רחב של שאלות, לכן הכי חשוב לתרגל כאן
	\end{itemize}
	הרמה הטכנית לא מאוד גבוהה, הבעיה העקרית היא זמן. תוך דקה עד דקה וחצי צריך להספיק לקרוא את השאלה ולהקיא תשובה. השאלה ברמת כיתה ז'-ח'. 
	
		
	\subsection{חשיבה צורנית}
	הפרק האחד לפני האחרון בדפ''ר. מאוד מזכיר את האנלוגיות הצורניות. מטריצות – בשורות ועמודות, לא באלכסון. (מראה דוגמה באלכסון. זה יכול להיות באלכסונים. זה גם יכול להיות מודוגלרי). 
	
	זה בערך הכל בריבועים ואז קורים דברים. 
	
	\subsection{הבנת הוראות}
	``הוא בעיקר נועד לעצבן, גם ילדים ביסודי יכולים לפתור אותו כשאין להם לחץ והם לא מבולבלים''. 
	
	\subsection{קשב מתמשך}
	מבחן מתמשך הבוחן את היכולת להגיב בעקבויות למטרה משעממת וחוזרת על עצמה (לדוגמה, מה שנעם עושה, או תצפיתניות)
	\begin{itemize}
		\item לא משוקלל לדפ''ר
		\item לא ניתן לראות כמה זמן נותר לסיום כל שאלה
		\item אין התאמות במבחן, אפילו אם יש בדפ''ר
	\end{itemize}
	
	
	\section{ראיון}
	המטרה: 
	\begin{itemize}
		\item להכיר באופן אישי של המלש''בים
		\item להעריך התאמה למסגרת הצבאית
		\item לזהות פוטנציאל הצלחה במגוון תפקידים
	\end{itemize}
	השיקולים של צה''ל לאיזה מיונים יפתחו: כישורים ויכולות, צרכי המערכת, והעדפות אישיות. לא כדאי להגיד דברים שאתם לא מעוניינים בו או מסוגלים אליו. ''בגדול זה דו־שיח נעים``. המראיינים בערך בגיל שלנו (18-20) והם ''מדברים איתכם בגובה העיניים``. החדר קטן, 1:1. אין זמן אחיד לראיון – חצי שעה, שעה, יכול להיות שעה וחצי. הראיון לא קובע את התפקיד. בנות מתראיינות רק אצל מראיינות, לבנים יכולים להיות מראיינים משני המינים. המראיינים נותנים ציון וחוו''ד שנשמרים במערכת ואינם חשופים למשל''ב. 
	
	קחו בחשבון: 
	\begin{itemize}
		\item מנח גוף ושפת גוף
		\item קשר עין
		\item הבעות פנים
		\item מחוות ידיים
		\item קול
	\end{itemize}
	
	
	
	
	כניסה למפתח: 30102010. יש לנו שלושה חודשים באפליקצייה. אפשר להקפיא. 
	
	
	
	\ndoc
\end{document}